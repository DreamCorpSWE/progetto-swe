\documentclass[12pt]{article}

\newcommand{\documento}{Verbale 2019-02-01}
\usepackage{fancyhdr}
\usepackage{titling}
\usepackage{caption}
\usepackage{multirow}
\usepackage{tabularx}
\usepackage[T1]{fontenc}
\usepackage[utf8]{inputenc}
\usepackage[italian]{babel}
\usepackage{lastpage}
\usepackage{nopageno}
\usepackage{graphicx}
\usepackage [colorlinks=true,urlcolor=blue, linkcolor=black]{hyperref}
\newcommand{\subtitle}[1]{%
    \posttitle{%
        \par\end{center}
        \begin{center}\LARGE#1\end{center}

    \vskip0.5em}%
}

\setlength{\oddsidemargin}{0in}
\setlength{\evensidemargin}{0in}
\setlength{\topmargin}{0in}
\iffalse \setlength{\headsep}{-.25in}\fi
\setlength{\textwidth}{6.5in}
\setlength{\textheight}{8.5in}

\font\myfont=cmr12 at 40pt

\pagestyle{fancy}
\fancyhf{}
\rhead{\leftmark}
\lhead{\includegraphics[width = 20mm]{../logo.png}}
\rfoot{Pagina \thepage \space di \pageref{LastPage}}
\lfoot{Studio di fattibilità}
\renewcommand{\footrulewidth}{0.4pt}
\newcommand{\red}{\daL}
\newcommand{\verp}{\pie}
\newcommand{\res}{\mar}
\newcommand{\version}{Versione 1.0.0}
\newcommand{\use}{Interno}
\title{\fontsize{40}{40}\selectfont Verbale interno 22/02/2019}
\author{Dream Corp.}
\date{22/02/2019}

\begin{document}
\maketitle
\begin{center}
	\hspace{5em}
	\includegraphics[width =70mm]{logo.png}\newline 
	\\G\&B
	\begin{table}[!htpb]
		\centering
		\begin{tabular}{r|l}
			\multicolumn{2}{c}{Informazioni sul documento}\\
			\hline
			Versione & \version \\
			Responsabile & \res\\
			Redattori & \red \\
			Verificatori & \verp\\
			Uso & \use\\
			
			Destinatari & Dream Corp. \\
			& Prof. Tullio Vardanega\\
			& Prof. Riccardo Cardin\\
			& Zucchetti SpA\\
		\end{tabular}
	\end{table}
\end{center}
\newpage

~\newline


\section{Riunione}
    \subsection{Informazioni generali}
    \begin{itemize}
        \item \textbf{Motivo della riunione}: È stata indetta questa riunione per organizzarsi con lo sviluppo della Technology Baseline.
        \item \textbf{Luogo e Data}: LabTA Torre Archimede, 20/02/2019;
        \item \textbf{Orario}: 10:00-13:00;
        \item \textbf{Partecipanti}: Tutti i membri del gruppo.
        \newpage
        \section{Ordine del giorno}
        \begin{itemize}
        \item Programmazione incontro esterno con l'azienda Zucchetti venerdì (22/02/2019) pomeriggio alle 14:30.
        \item Discussione sullo sviluppo del test.
        \end{itemize}
        
        \newpage
    \begin{enumerate}
\section{Risultati}
    \subsection{Ordine del giorno}
        \item \textbf{Technology Baseline:} Sono state scelte le tecnologie per i test (Jest, Coveralls) e sono state definite le prime slide per i colloqui della Technology Baseline.
        \item \textbf{Incontro con l'azienda Zucchetti :} Per mostrare i risultati dello sviluppo del plugin il gruppo ha deciso di richiedere un colloquio con la proponente Zucchetti, confermato per il 22/02/2019
    \end{enumerate}
    \subsection{Tracciamento delle decisioni}
    \begin{table}[!h] % h! serve per posizionarla relativamente
            \centering
            \renewcommand{\arraystretch}{2}
            \rowcolors{2}{gray!25}{white} %colori alternati, grigio 25% e bianco 100%
            \begin{tabular}{|c|c|p{6cm}|l|l|} % p{dimensione desiderata}
                \rowcolor{orange!50} %colore intestazione
        		\hline
        		\textbf{Codice} & \textbf{Decisione}\\
                \hline
                VER-2019-02-20.1 & Scelta delle tecnologie per i test\\
                \hline
                VER-2019-02-20.2 & Sviluppo prime slide per Technology Baseline.\\
                \hline
        \end{tabular}
        \caption{Tracciamento delle modifiche} %descrizone a fine tabella
        \label{tab:Tracciamento delle modifiche}
        \end{table}
    \end{itemize}




\end{document}
