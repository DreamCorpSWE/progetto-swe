\documentclass[12pt]{article}

\newcommand{\documento}{Verbale 2019-02-01}
\usepackage{fancyhdr}
\usepackage{titling}
\usepackage{caption}
\usepackage{multirow}
\usepackage{tabularx}
\usepackage[T1]{fontenc}
\usepackage[utf8]{inputenc}
\usepackage[italian]{babel}
\usepackage{lastpage}
\usepackage{nopageno}
\usepackage{graphicx}
\usepackage [colorlinks=true,urlcolor=blue, linkcolor=black]{hyperref}
\newcommand{\subtitle}[1]{%
    \posttitle{%
        \par\end{center}
        \begin{center}\LARGE#1\end{center}

    \vskip0.5em}%
}

\setlength{\oddsidemargin}{0in}
\setlength{\evensidemargin}{0in}
\setlength{\topmargin}{0in}
\iffalse \setlength{\headsep}{-.25in}\fi
\setlength{\textwidth}{6.5in}
\setlength{\textheight}{8.5in}

\font\myfont=cmr12 at 40pt

\pagestyle{fancy}
\fancyhf{}
\rhead{\leftmark}
\lhead{\includegraphics[width = 20mm]{../logo.png}}
\rfoot{Pagina \thepage \space di \pageref{LastPage}}
\lfoot{Studio di fattibilità}
\renewcommand{\footrulewidth}{0.4pt}
\newcommand{\red}{\mic}
\newcommand{\verp}{\mat}
\newcommand{\res}{\mic}
\newcommand{\version}{Versione 1.0.0}
\newcommand{\use}{Interno}
\title{\fontsize{40}{40}\selectfont Verbale interno 04/02/2019}
\author{Dream Corp.}
\date{04/02/2019}

\begin{document}
\maketitle
\begin{center}
	\hspace{5em}
	\includegraphics[width =70mm]{logo.png}\newline 
	\\G\&B
	\begin{table}[!htpb]
		\centering
		\begin{tabular}{r|l}
			\multicolumn{2}{c}{Informazioni sul documento}\\
			\hline
			Versione & \version \\
			Responsabile & \res\\
			Redattori & \red \\
			Verificatori & \verp\\
			Uso & \use\\
			
			Destinatari & Dream Corp. \\
			& Prof. Tullio Vardanega\\
			& Prof. Riccardo Cardin\\
			& Zucchetti SpA\\
		\end{tabular}
	\end{table}
\end{center}
\newpage

~\newline


\section{Riunione}
    \subsection{Informazioni generali}
    \begin{itemize}
        \item \textbf{Motivo della riunione}: È stata indetta questa riunione per riorganizzarsi e capire i problemi sorti in seguito all'entrata in RR.   \pedice.
        \item \textbf{Luogo e Data}: LabTA Torre Archimede, 01/02/2019;
        \item \textbf{Orario}: 9:00-14:00;
        \item \textbf{Partecipanti}: Tutti i membri del gruppo eccetto \daG, assente poiché indisposto e Michele Clerici che ha preso parte alla riunione tramite videochat Hangouts per problemi di trasporto.
        \newpage
        \section{Ordine del giorno}
        \begin{itemize}
        \item Decisione su cambio editor di testo Latex;
        \item Aggiornamento Norme di Progetto;
        \item Programmazione nuove riunioni;
        \item Programmazione incontro esterno con l'azienda Zucchetti mercoledì (07/02/2019) pomeriggio alle 14:30.
        \end{itemize}
        
        \newpage
    \begin{enumerate}
\section{Risultati}
    \subsection{Ordine del giorno}
        \item \textbf{Cambio editor Latex :} Dopo un attenta ponderazione sui pro e contro il gruppo ha deciso di migrare verso l'editor online Overleaf al posto di TeXStudio, usato fino ad'ora;
        \item \textbf{Deadline interna :} Il manager, per poter affrontare al meglio il periodo che va dall'RR all'RP, ha deciso di imporre una deadline al gruppo che consiste nel terminare le modifiche atte alla sistemazione dei documenti entro il 10 Febbraio, potendo così poi concentrarsi sulle fasi dedicate alla Revisione di Progetto.
        \item \textbf{Incontro con l'azienda Zucchetti :} Per risolvere diversi dubbi sorti recentemente il gruppo ha deciso di richiedere un colloquio con la proponente Zucchetti, confermato per il 07/02/2019
    \end{enumerate}
    \subsection{Tracciamento delle decisioni}
    \begin{table}[!h] % h! serve per posizionarla relativamente
            \centering
            \renewcommand{\arraystretch}{2}
            \rowcolors{2}{gray!25}{white} %colori alternati, grigio 25% e bianco 100%
            \begin{tabular}{|c|c|p{6cm}|l|l|} % p{dimensione desiderata}
                \rowcolor{orange!50} %colore intestazione
        		\hline
        		\textbf{Codice} & \textbf{Decisione}\\
                \hline
                VER-2019-02-01.1 & Passaggio da TexStudio a Overleaf come editor per Latex\\
                \hline
                VER-2019-02-01.2 & Revisione di tutti i documenti\\
                \hline
        \end{tabular}
        \caption{Tracciamento delle modifiche} %descrizone a fine tabella
        \label{tab:Tracciamento delle modifiche}
        \end{table}
    \end{itemize}




\end{document}
