\section{Informazioni sul documento}
    \subsection{Scopo del documento}
    	Questo documento si prefigge lo scopo di garantire a tutti i membri del gruppo un modo comune di lavorare al fine di aumentare l'efficienza\pedice. Verranno descritte le scelte architetturali e i vari software scelti. 
    \subsection{Aggiornamenti}
        Il documento non è da considerarsi in versione finale e verrà aggiornato in modo appropriato a seconda delle esigenze che sorgeranno durante le varie fasi di sviluppo del progetto. Ogni attività verrà quindi normata prima che venga eseguita.
    \subsection{Il prodotto}
    	Il prodotto ha lo scopo di fornire un sistema "smart" di monitoraggio dei sistemi in modo da garantire e migliorare i servizi erogati dall'azienda ai terzi. L'applicativo sarà un'estensione scritta in Javascript\pedice per il software Grafana\pedice, si lavorerà inoltre con le reti bayesiane\pedice.
    \subsection{Glossario}
    	Data la presenza di diversi elementi con significato ambiguo è stato necessario l'utilizzo di un glossario volto a disambiguare tali elementi col loro preciso significato. Questi termini verranno contrassegnati all'interno dei documenti con la lettera \textbf{G} a pedice e in grassetto.
\subsection{Riferimenti}
    \subsubsection{Normativi}
	    \begin{itemize}
	        \item Standard ISO/IEC 12207:1995 \newline \url{https://www.math.unipd.it/~tullio/IS-1/2009/Approfondimenti/ISO_12207-1995.pdf}
	        \item Capitolato C3 \newline \url{https://www.math.unipd.it/~tullio/IS-1/2018/Progetto/C3.pdf}
	    \end{itemize}
    \subsubsection{Informativi}
	    \begin{itemize}
	        \item \PdP;
	        \item \PdQ;
	        \item UML 2.0;
	        \item Airbnb Javascript\pedice Style Guide\newline
	        \url{https://github.com/airbnb/javascript};
            \item Slide del corso Ingegneria del Software \newline
            \url{https://www.math.unipd.it/~tullio/IS-1/2018/}
            \item PDCA\newline
            \url{https://it.wikipedia.org/wiki/Ciclo_di_Deming}
	    \end{itemize}
