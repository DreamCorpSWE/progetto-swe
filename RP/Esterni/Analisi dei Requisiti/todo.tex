TO DO 1

SCRIVO SUL LOG LE MODIFICHE (per aggiunte future)
METTO NOMI CORRETTI SUL LOG (per aggiunte future)
inserire i riferimenti dei sotto casi d'uso nello scenario principale del caso d'uso principale


------------------------------------------------------------------------------------------------------
TO DO DA CORREZIONE

UC6 e UC7: quali sono le informazioni richieste per configurare un alert? 

tracciamento una volta conclusa la sistemazione dei requisiti
Tabella riepilogo tracciamento
Rilettura casi d'uso estensioni

--- BORDIN

Fig. 3: UC2 non può essere presente nel diagramma che lo descrive graficamente. Il
medesimo problema ricorre anche per altri casi d’uso (UC6,UC7, ...). 

RIfare i diagrammi UML dei casi d'uso secondo quelli nuovi

--- CONSIDERAZIONI GENERALI ---

I requisiti funzionali hanno la medesima granularità dei casi d’uso. Poiché quest’ultimi
non raggiungono un livello di dettaglio sufficientemente elevato, anche i
requisiti devono essere dettagliati maggiormente.

Il documento analizza le funzionalità del prodotto a un livello di dettaglio del
tutto insufficiente. 
Dovete spingervi in maggior dettaglio, sia nei casi d’uso che nei requisiti. 
Nel complesso, documento da rivedere.