\section{Tracciamento dei Requisiti}

\subsection{Tracciamento Casi d'Uso e Requisiti Funzionali}
Per una maggiore comprensione dei requisiti verranno rappresentate, in forma tabellare, le relazioni tra requisiti funzionali e casi d'uso. Il tracciamento diretto tra casi d'uso e requisiti renderà la comprensione del sistema più facile e veloce.

\begin{table}[!htbp] % h! serve per posizionarla relativamente
            \centering
            \renewcommand{\arraystretch}{2} % dimensione verticale delle righe
            \rowcolors{2}{gray!25}{white} %colori alternati, grigio 25% e bianco 100%
            \begin{tabular}{|c|c|} % p{dimensione desiderata}
                \rowcolor{orange!50} %colore intestazione
        		\hline
        		\textbf{Codice Caso d'Uso} & \textbf{Codice Requisito} \\
                \hline
                UC1 & RFC1\\
                \hline
                UC1(A) & RFC1A\\
                \hline
                UC1(A).1 & RFC1A.1\\
                \hline
                UC1(B) & RFC1B\\
                \hline
                UC2 & RFC2\\
                \hline
                UC2.1 & RFC2.1\\
                \hline
                UC2.2 & RFC2.2\\
                \hline
                UC2.3 & RFC2.3\\
                \hline
                UC2.3.1 & RFC2.3.1\\
                \hline
                UC2.3.2 & RFC2.3.2\\
                \hline
                UC3 & RFC3\\
                \hline
                UC3.1 & RFC3.1\\
                \hline
                UC3.2 & RFC3.2\\
                \hline
                \end{tabular}
        \end{table}
        \begin{table}[!htbp] % h! serve per posizionarla relativamente
            \centering
            \renewcommand{\arraystretch}{2} % dimensione verticale delle righe
            \rowcolors{2}{gray!25}{white} %colori alternati, grigio 25% e bianco 100%
                \begin{tabular}{|c|c|} % p{dimensione desiderata}
                \rowcolor{orange!50} %colore intestazione
        		\hline
        		\textbf{Codice Caso d'Uso} & \textbf{Codice Requisito} \\
                \hline
                UC3.2(A)& RFC3.2A\\
                \hline
                UC3.2(A).1 & RFC3.2A.1\\
                \hline
                UC3.2(A).2 & RFC3.2A.2\\
                \hline
                UC3.2(A).3 & RFC3.2.3\\
                \hline
                UC3.2(B) & RFC3.2B\\
                \hline
                UC3.2(B).1 & RFC3.2B.1\\
                \hline
                UC3.2(B).2 & RFC3.2B.2\\
                \hline
                UC4 & RFC4\\
                \hline
                UC4.1 & RFC4.1\\
                \hline
                UC4.1(A) & RFC4.1A\\
                \hline
                UC4.1(B) & RFC4.1B\\
                \hline
                UC4.2 & RFC4.2\\
                \hline
                UC5 & RFO5\\
                \hline
                UC5.1 & RFO5.1\\
                \hline
                UC5.2 & RFO5.2\\
                \hline
                UC5.2.1 & RFO5.2.1\\
                \hline
                UC5.2.2 & RFO5.2.2\\
                \hline
                UC6 & RFC6\\
                \hline
                \end{tabular}
        \end{table}
        \begin{table}[!htbp] % h! serve per posizionarla relativamente
            \centering
            \renewcommand{\arraystretch}{2} % dimensione verticale delle righe
            \rowcolors{2}{gray!25}{white} %colori alternati, grigio 25% e bianco 100%
                \begin{tabular}{|c|c|} % p{dimensione desiderata}
                \rowcolor{orange!50} %colore intestazione
        		\hline
        		\textbf{Codice Caso d'Uso} & \textbf{Codice Requisito} \\
                \hline
                UC7 & RFC7\\
                \hline
                UC7.1 & RFC7.1\\
                \hline
                UC7.2 & RFC7.2\\
                \hline
                UC7.3 & RFC7.3\\
                \hline
                UC7.4 & RFC7.4\\
                \hline
                UC7.5 & RFC7.5\\
                \hline
                UC7.6 & RFC7.6\\
                \hline
                UC7.7 & RFC7.7\\
                \hline
                UC7.8 & RFC7.8\\
                \hline
                UC8 & RFC8\\
                \hline
                UC8.1 & RFC8.1\\
                \hline
                UC8.1.1 & RFC8.1.1\\
                \hline
                UC8.1.2 & RFC8.1.2\\
                \hline
                UC9 & RFC9\\
                \hline
                UC9.1 & RFC9.1\\
                \hline
                UC10 & RFC10\\
                \hline
                UC10.1 & RFC10.1\\
                \hline
            \end{tabular}
        \end{table}
        \begin{table}[!htbp] % h! serve per posizionarla relativamente
            \centering
            \renewcommand{\arraystretch}{2} % dimensione verticale delle righe
            \rowcolors{2}{gray!25}{white} %colori alternati, grigio 25% e bianco 100%
                \begin{tabular}{|c|c|} % p{dimensione desiderata}
                \rowcolor{orange!50} %colore intestazione
        		\hline
        		\textbf{Codice Caso d'Uso} & \textbf{Codice Requisito} \\
                \hline
                UC10.2 & RFC10.2\\
                \hline
                UC11 & RFC11\\
                \hline
                UC12 & RFC12\\
                \hline
                UC13 & RFC13\\
                \hline
                UC14 & RFC14\\
                \hline
                UC15 & RFC15\\
                \hline
                UC16 & RFC16\\
                \hline
                UC17 & RFC17\\
                \hline
                UC18 & RFC18\\
                \hline
                UC19 & RFC19\\
                \hline
        \end{tabular}
        \caption{Tracciamento Casi d'Uso - Requisiti Funzionali} %descrizione a fine tabella
\end{table}

\subsection{Tracciamento Requisiti per Tipologia e Priorità}
Per facilitare la lettura e la visualizzazione verranno presentate delle tabelle indicizzate in modo specifico:
\begin{itemize}
    \item Tracciamento Priorità-Requisito
    \item Tracciamento Tipologia-Requisito
\end{itemize}

\subsubsection{Tracciamento: Priorità-Requisito}
Di seguito viene riportata una tabella riassuntiva che descrive le priorità dei requisiti e i requisiti che vi appartengono per rendere di più facile lettura la distribuzione per priorità.
Per brevità non viene riportata la descrizione del requisito ma semplicemente il suo codice univoco in quando era già stato descritto in dettaglio nelle sezioni precedenti.   

\begin{table}[!htbp] % h! serve per posizionarla relativamente
            \centering
            \renewcommand{\arraystretch}{2} % dimensione verticale delle righe
            \rowcolors{2}{gray!25}{white} %colori alternati, grigio 25% e bianco 100%
            \begin{tabular}{|c|p{2cm}|} % p{dimensione desiderata}
                \rowcolor{orange!50} %colore intestazione
        		\hline
        		\textbf{Priorità Requisito} & \textbf{Codice Requisiti} \\
                \hline
                Compulsory & RFC1, RFC1A, RFC1A.1, RFC1B, RFC2, RFC2.1, RFC2.2, RFC2.3, RFC2.3.1, RFC2.3.2, RFC3, RFC3.1, RFC3.2, RFC3.2A, RFC3.2A.1, RFC3.2A.2, RFC3.2.3, RFC3.2B, RFC3.2B.2, RFC4, RFC4.1, RFC4.1A, RFC4.1B, RFC4.2, RFC6, RFC7, RFC7.1, RFC7.2, RFC7.3, RFC7.4, RFC7.5, RFC7.6, RFC7.7, RFC7.8\\
                \hline
            \end{tabular}
\end{table}
\begin{table}[!htbp] % h! serve per posizionarla relativamente
            \centering
            \renewcommand{\arraystretch}{2} % dimensione verticale delle righe
            \rowcolors{2}{gray!25}{white} %colori alternati, grigio 25% e bianco 100%
            \begin{tabular}{|c|p{2cm}|} % p{dimensione desiderata}
                \rowcolor{orange!50} %colore intestazione
        		\hline
        		\textbf{Priorità Requisito} & \textbf{Codice Requisiti} \\
                \hline
                & RFC8, RFC8.1, RFC8.1.1, RFC8.1.2, RFC9, RFC9.1, RFC10, RFC10.1, RFC10.2, RFC11, RFC12, RFC13, RFC14, RFC15, RFC16, RFC17, RFC18, RFC19\\
                \hline
                Optional & RFO5, RFO5.1, RFO5.2, RFO5.2.1, RFO5.2.2\\
                \hline
        \end{tabular}
        \caption{Tracciamento Priorità-Requisito} %descrizione a fine tabella
\end{table}
\newpage
\subsubsection{Tracciamento: Tipologia-Requisito}
Di seguito viene riportata una tabella riassuntiva che descrive le tipologie di requisiti e i requisiti che vi appartengono per rendere di più facile lettura la distribuzione per tipologia.
Per brevità non viene riportata la descrizione del requisito ma semplicemente il suo codice univoco in quando era già stato descritto in dettaglio nelle sezioni precedenti.

\begin{table}[!htbp] % h! serve per posizionarla relativamente
            \centering
            \renewcommand{\arraystretch}{2} % dimensione verticale delle righe
            \rowcolors{2}{gray!25}{white} %colori alternati, grigio 25% e bianco 100%
            \begin{tabular}{|c|p{2cm}|} % p{dimensione desiderata}
                \rowcolor{orange!50} %colore intestazione
        		\hline
        		\textbf{Tipologia Requisito} & \textbf{Codice Requisiti} \\
                \hline
                Funzionale & RFC1, RFC1A, RFC1A.1, RFC1B, RFC2, RFC2.1, RFC2.2, RFC2.3, RFC2.3.1, RFC2.3.2, RFC3, RFC3.1, RFC3.2, RFC3.2A, RFC3.2A.1, RFC3.2A.2, RFC3.2.3, RFC3.2B, RFC3.2B.2, RFC4, RFC4.1, RFC4.1A, RFC4.1B, RFC4.2, RFO5, RFO5.1, RFO5.2, RFO5.2.1, RFO5.2.2, RFC6, RFC7, RFC7.1, RFC7.2, RFC7.3 \\
                \hline
            \end{tabular}
\end{table}
\begin{table}[!htbp] % h! serve per posizionarla relativamente
            \centering
            \renewcommand{\arraystretch}{2} % dimensione verticale delle righe
            \rowcolors{2}{gray!25}{white} %colori alternati, grigio 25% e bianco 100%
            \begin{tabular}{|c|p{2cm}|} % p{dimensione desiderata}
                \rowcolor{orange!50} %colore intestazione
        		\hline
        		\textbf{Tipologia Requisito} & \textbf{Codice Requisiti} \\
                \hline
                & RFC7.4, RFC7.5, RFC7.6, RFC7.7, RFC7.8, RFC8, RFC8.1, RFC8.1.1, RFC8.1.2, RFC9, RFC9.1, RFC10, RFC10.1, RFC10.2, RFC11, RFC12, RFC13, RFC14, RFC15, RFC16, RFC17, RFC18, RFC19 \\
                \hline
                Vincolo & RVO1, RCV2, RCV3, RCV4, RCV5\\
                \hline
                Qualità & RQC1, RQC2, RQO3, RQC4\\
                \hline
        \end{tabular}
        \caption{Tracciamento Tipologia-Requisito} %descrizione a fine tabella
\end{table}

\newpage
\subsection{Riepilogo}
Di seguito viene riportata una tabella riassuntiva a doppia entrata con la tipologia di requisito per colonna e la priorità per riga. Viene riportato come valore il numero per ogni tipologia e priorità al fine di avere un quadro generale ed immediato della distribuzione degli stessi.

\begin{table}[!htbp] % h! serve per posizionarla relativamente
            \centering
            \renewcommand{\arraystretch}{2} % dimensione verticale delle righe
            \rowcolors{2}{gray!25}{white} %colori alternati, grigio 25% e bianco 100%
            \begin{tabular}{|c|c|c|c|c|c|} % p{dimensione desiderata}
                \rowcolor{orange!50} %colore intestazione
        		\hline & \textbf{Funzionali} & \textbf{Vincolo} & \textbf{Qualità} & \textbf{Prestazionali} & \textbf{Tot. Priorità}\\
                \hline
                \textbf{Compulsory} & 52 & 4 & 3 & 0 & 59\\
                \hline
                \textbf{Optional} & 5 & 1 & 1 & 0 & 7\\
                \hline
                \textbf{Tot. Tipologia} & 53 & 5 & 4 & 0 & 66\\
                \hline
        \end{tabular}
        \caption{Riepilogo Distribuzione Requisiti} %descrizione a fine tabella
\end{table}
