
I casi d'uso possono essere riutilizzati (per esempio le conferme o qualche errore) in tal caso conviene fare un caso d'uso a parte
ATTENZIONE alle pre e post

Da discutere UC3 intero. In base alla mia breve progettazione non è nemmeno necessario porsi il problema della duplice associazione per l'utente. noi dobbiamo garantirla ma all'utente sarà totalmente oscuro cosa succede. 
Basterà che nel caso d'uso UC2 (quello di associazione nodi) non si mostrino, per una certa rete, i nodi che sono già stati assegnati ad un flusso dati. Così si evita di rompere la proprietà di biunivocità e per i casi d'uso è molto semplificato e pulito.

Propongo di creare un caso UC2 che sia:
UC2: configurazione associazione nodo-flusso dati
In cui all'interno sia presente una scelta.
Dopo aver selezionato il flusso dati, dopo aver selezionato la rete bayesiana,...
1) C'è già un'associazione --> ci mostra il nodo selezionato in precedenza
2) Non c'è già un'associazione --> ci fa scegliere da una lista i nodi possibili.
Se ne avevamo già uno, possiamo rimuoverlo, con il pulsante "Dissocia".
Una volta fatto "dissocia" si torna al caso in cui nessun nodo è stato associato al flusso dati e quindi il secondo caso della scelta


UCx modifica dati del panel in modalità edit.--->UC3
    uc x.1 titolo
    uc x.2 descrizione
    uc x.3 salvataggio dashboard

fare caso d'uso per entrare in modalitá modifica


le pre condizioni descrivono la situazione in quel momento quindi es. "é disponibile la finestra di selezione" quindi in modo impersonale e che descrive solo la situazione attuale
casomai anche la schermata in cui ci si trova