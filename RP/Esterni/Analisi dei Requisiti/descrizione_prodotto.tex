\section{Descrizione prodotto}
		\subsection{Scopo del prodotto}			
Si devono realizzare dei sistemi che possano applicare metodi di intelligenza artificiale\pedice al flusso dei dati raccolti, al fine non solo di monitorare la situazione del sistema ma anche per consigliare gli interventi o quanto meno le zone di intervento alla linea di produzione del software.
Il presente capitolato ha per oggetto l'affidamento della fornitura per la
realizzazione di un plug-in per lo strumento di monitoraggio Grafana\pedice che
applichi reti Bayesiane\pedice al flusso dei dati ricevuti per allarmi o segnalazioni tra gli operatori del servizio Cloud e la linea di produzione del software.


		\subsection{Funzioni del prodotto}
Il plug-in deve poter leggere la definizione di una rete Bayesiana da un file in formato json. Avverrà successivamente l'associazione dei flussi di monitoraggio con i nodi della rete Bayesiana inserita. I nodi della rete mapperanno gli stati dei flussi di monitoraggio e il cambio di stato avverrà nel momento in cui si rileverà una variazione di valore nel flusso dati. Questa variazione del valore verrà catturata dal sistema di Grafana, dopo la definizione di un alert opportuno per quel flusso di monitoraggio. Un alert è un messaggio che viene inviato da Grafana, riportante lo stato del nodo monitorato e il suo valore, quando quest'ultimo viola una delle condizioni poste. L'alert può essere modificato e rimosso a discrezione dell'utente.
Nel momento in cui un alert viene lanciato il sistema avvia il ricalcolo delle probabilità dell'intera rete grazie al supporto della libreria jsbayes. 
I valori ricalcolati delle probabilità dei nodi della rete a cui non è stato associato un flusso di monitoraggio verranno rappresentati con dei grafici, detti "panel", grazie al supporto di Grafana.
Su questi valori probabilistici verranno a loro volta applicati opzionalmente degli alert, i quali si attiveranno quando le stime su queste percentuali supereranno una certa soglia decisa dall'utente.


		\subsection{Tipologia di utenti}

Il prodotto è rivolto a tutti coloro che gestiscono i sistemi di raccolta e collezione di dati in modo da poterli monitorare ed intervenire qualora sia necessario, grazie a degli allarmi forniti dal plug-in.
Gli utenti devono avere familiarità  con il tool di monitoraggio Grafana.


		\subsection{Vincoli di progettazione}
			\subsubsection{Requisiti obbligatori}
				\begin{itemize}
					\item Leggere la definizione della rete Bayesiana da un file in formato json;
					\item Associare i nodi della rete, letta dal file json, ad un flusso di dati presente in Grafana;
					\item Applicare il ricalcolo delle probabilità della rete secondo regole temporali prestabilite;
					\item Fornire nuovi dati al sistema di Grafana derivati dai nodi della rete non collegati al flusso di monitoraggio;
					\item Rendere disponibili i dati al sistema di creazione di grafici e dashboard per la loro visualizzazione.
		        	\end{itemize}
			\subsubsection{Requisiti opzionali}
				\begin{itemize}
					\item Possibilità di definire "alert\pedice" in base a livelli di soglia raggiunti dai nodi non collegati al flusso dei dati;
					\item Possibilità di disegnare la rete Bayesiana con un piccolo editor grafico specializzato;
					\item Possibilità di applicare più reti Bayesiane in oggetti di monitoraggio diversi;
					\item Possibilità di creare una rete Bayesiana a partire dai dati raccolti sul campo anziché svilupparla con la collaborazione degli esperti del settore;
					\item Identificare altri metodi di Intelligenza Artificiale oltre alla rete Bayesiana che siano applicabili all'analisi del flusso di dati di monitoraggio.
		        	\end{itemize}
		        \subsubsection{Requisiti opzionali scelti da implementare}
				\begin{itemize}
					\item Possibilità di definire "alert" in base a livelli di soglia raggiunti dai nodi non collegati al flusso dei dati;
					\item Possibilità di applicare più reti Bayesiane in oggetti di monitoraggio diversi.
	        		\end{itemize}
				
				
			\subsection{Vincoli generali}			
Essendo il prodotto un plug-in per Grafana (versione 5.4.3), occorre aver scaricato e configurato Grafana (versione 5.4.3 o superiore) in modo da poter installare il plug-in.


\newpage
