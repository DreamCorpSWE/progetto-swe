\section{A}
    \subsection*{AirBnb}
Insieme di regole che indicano come un codice Javascript dovrebbe essere organizzato e scritto in modo da avere un codice più leggibile e corretto.
    \subsection*{Alert}
Segnale che avvisa, tramite una notifica, una possibile situazione critica da tenere sotto controllo.
    \subsection*{Attori}
Chi o cosa interagisce con il sistema, possono essere sia umani che altri software.
\newpage

\section{B}
    \subsection*{Branch}
Ramo di lavoro che favorisce uno sviluppo non lineare e incrementale, offrendo la possibilità di sviluppare degli incrementi sui branch che verranno poi uniti al branch principale chiamato master.
\newpage

\section{C}
    \subsection*{Capitolato}
Il capitolato è un documento tecnico in linguaggio naturale al quale si fa riferimento per definire le specifiche tecniche di un problema che si vuole risolvere e le caratteristiche richieste dal prodotto.
    \subsection*{Casi d’uso}
Descrive una sequenza di interazioni tra il sistema e un utente che può essere un umano o un altro software, chiamato attore, che devono essere svolte per ottenere un risultato. Rappresentano quindi i vari scenari che si possono incontrare nell’ utilizzo di un prodotto software.
    \subsection*{Code Coverage}
La percentuale di linee di codice del progetto che sono state eseguite dai test dopo un’esecuzione.
    \subsection*{Commit}
Un insieme di modifiche o incrementi effettuati ai file contenuti in un repository, richiede una breve descrizione su quanto aggiunto o modificato.
\newpage

\section{D}
    \subsection*{Dashboard}
Insieme di Panels organizzati e disposti in una o più righe.
    \subsection*{DevOps}
DevOps (Development and Operations) è un insieme di pratiche finalizzate ad aumentare l’automatismo tra processi per lo sviluppo di software in modo da poter effettuare le build, implementare i test e rilasciare software in modo più veloce e sicuro.
    \subsection*{Diagramma di Gantt}
Permette la rappresentazione grafica di un calendario di attività,utile al fine di pianificare, coordinare e tracciare specifiche attività in un progetto dando una chiara illustrazione dello stato d’avanzamento del progetto rappresentato.
\newpage

\section{E}
    \subsection*{Efficienza}
Indica la capacità di raggiungere l’obiettivo prefissato impiegando le risorse minime indispensabili.
\newpage
    
\section{F}
    \subsection*{Flusso di monitoraggio}
Flusso di dati provenienti da un database interrogato da una query effettuata ad intervalli temporali prestabiliti.
    \subsection*{Framework}
E’ un’ architettura logica di supporto (spesso un’ implementazione logica di un particolare design pattern) su cui un software può essere progettato e realizzato,spesso facilitandone lo sviluppo da parte del programmatore.
\newpage
    
\section{G}
    \subsection*{GitHub}
Software di controllo di versione, permette di aggiornare un file senza dover sovra-scrivere le versioni precedenti.
    \subsection*{Gmail}
Gmail è un servizio di email gratuito sviluppato da Google nel 2004, è possibile accedervi sia tramite web che tramite un’ apposita applicazione.
    \subsection*{Google Drive}
Google Drive è un servizio web, in ambiente cloud computing, di memorizzazione e sincronizzazione online introdotto da Google nel 2012. Comprende il file hosting, il file sharing e la modifica collaborativa di documenti, da a disposizione fino a 15 GB di spazio gratuito estendibili fino a 30 TB in totale.
    \subsection*{Grafana}
Grafana è un’ applicazione web open-source che permette il monitoraggio di flussi di dati provenienti da diversi tipi di database tramite dashboard e grafici.
    \subsection*{Gulpease index}
L’indice Gulpease è un indice di leggibilità di un testo tarato sulla lingua italiana.  Rispetto ad altri ha il vantaggio di utilizzare la lunghezza delle parole in lettere anziché in sillabe, semplificandone il calcolo automatico.
    \subsection*{Gunning Fog index}
L’indice Gunning Fox è stato progettato per misurare la facilità di lettura e di comprensione di un testo.
\newpage
    
\section{I}
    \subsection*{Intelligenza artificiale}
Abilità di un sistema tecnologico di risolvere problemi o svolgere compiti e attività tipici della mente e dell’abilità umana.
    \subsection*{ISO/IEC 15504}
Anche conosciuta come SPICE (Software Process improvement and Capability Determination), è un insieme di nove documenti di standard tecnici relativi ai processi di sviluppo del software e relative funzioni di business e, in particolare, alla loro valutazione.
    \subsection*{ISO/IEC9126}
Con la sigla ISO/IEC 9126 si individua una serie di normative e linee guida,sviluppate dall’ISO (Organizzazione internazionale per la normazione) in collaborazione con l’IEC (Commissione Elettrotecnica Internazionale), preposte a descrivere un modello di qualità del software.
    \subsection*{Issue Tracking System}
E’ un software che registra e mantiene una lista delle issues, ovvero un’ insieme di passi da compiere per migliorare un sistema.
\newpage
    
\section{J}
    \subsection*{Javascript}
E’ un linguaggio dinamico di alto livello, lato client, interpretato dal browser e spesso utilizzato nel web per descrivere il comportamento delle pagine.
    \subsection*{Json}
(JavaScript Object Notation) Viene usato per lo scambio di dati tra il browser e il server, è basato sul modello della programmazione ad oggetti di Javascript.
\newpage

\section{L}
    \subsection*{Latex}
Latex è un linguaggio di markup per la preparazione di testi, basato sul programma di composizione tipografica TEX.
\newpage

\section{M}
    \subsection*{Machine Learning}
Rappresenta un insieme di metodi statistici per migliorare progressivamente la performance di un algoritmo nell’identificare pattern nei dati.
    \subsection*{Magnitudo}
Nell’analisi dei rischi è calcolata come il prodotto dell’impatto di un rischio e della probabilità che esso accada. Descrive quindi il rischio reale.
    \subsection*{Mean Time Between Failures}
(MTBF)Il tempo medio fra i guasti (in inglese meantime between failures, spesso abbreviato in MTBF), è un parametro di affidabilità applicabile a dispositivi meccanici, elettrici ed elettronici e ad applicazioni software.
    \subsection*{Milestone}
Il termine milestone viene tipicamente utilizzato nella pianificazione e gestione di progetti complessi per indicare il raggiungimento di obiettivi stabiliti in fase di definizione del progetto stesso.
\newpage

\section{O}
    \subsection*{Open-Source}
Software di cui gli autori rendono pubblico il codice sorgente, favorendone il libero studio e permettendo a programmatori indipendenti di apportarvi modifiche ed estensioni.
    \subsection*{Overleaf}
Overleaf è un editor online Latex che permette la collaborazione real-time e la compilazione online.
\newpage

\section{P}
    \subsection*{Panel}
Pannello che mostra il risultato di una query sottoforma di grafico, tabella o testo.
    \subsection*{PDCA}
Il ciclo di Deming (o ciclo di PDCA, acronimo dall’inglese Plan–Do–Check–Act, in italiano "Pianificare - Fare - Verificare - Agire") è un metodo di gestione iterativo in quattro fasi utilizzato per il controllo e il miglioramento continuo dei processi e dei prodotti.
    \subsection*{Plugin}
Il plugin in campo informatico è un programma non autonomo che interagisce con un altro programma per ampliarne o estenderne le funzionalità originarie.
\newpage

\section{Q}
    \subsection*{Query}
Interrogazione da parte di un utente di un database per compiere determinate operazioni sui dati come selezione, inserimento, cancellazione e aggiornamento.
\newpage

\section{R}
    \subsection*{Repository}
Ambiente di un sistema informativo in cui vengono gestiti i metadati attraverso tabelle relazionali. L’insieme di tabelle, regole e motori di calcolo tramite cui si gestiscono i metadati prende il nome di metabase.
    \subsection*{Requisiti}
Requisiti che un sistema deve soddisfare, possono essere obbligatori se stabiliti dal cliente oppure opzionali se implementati a discrezione dello sviluppatore.
    \subsection*{Rete Bayesiana}
E’ un grafo aciclico orientato in cui i nodi rappresentano le variabili e gli archi rappresentano le relazioni di dipendenza statistica tra le variabili e le distribuzioni locali di probabilità dei nodi figlio rispetto ai valori dei nodi padre. Una rete bayesiana rappresenta la distribuzione della probabilità congiunta di un insieme di variabili.
\newpage

\section{S}
    \subsection*{Simple measure of Goobledygook}
(SMOG) L’indice SMOG è una misura di leggibilità che stima gli anni di educazione necessaria per comprendere una porzione di testo.
    \subsection*{Singular responsabilty principle}
Afferma che ogni elemento di un programma (classe,metodo, variabile) deve avere una sola responsabilità, e che tale responsabilità debba essere interamente incapsulata dall’elemento stesso. Tutti i servizi offerti dall’elemento dovrebbero essere strettamente allineati a tale responsabilità.
    \subsection*{Slack}
Applicazione di messaggistica istantanea pensata per la collaborazione tra i membri di uno o più gruppi di lavoro.  Offre la possibilità di creare canali privati, gruppi privati e chat dirette. Tutto ciò che è presente in Slack può essere cercato, che sia un file, una conversazione o le persone stesse. Inoltre può essere esteso mediante l’ uso di applicazioni di terze parti.
    \subsection*{Stakeholders}
Tutti i soggetti, individui od organizzazioni, attivamente coinvolti in un pro-getto il cui interesse è influenzato dal risultato e la cui azione o reazione a sua volta influenza le fasi o il completamento del progetto.  Ad esempio lo possono essere il cliente, il fornitore, i membri del team di progetto, i fruitori dei risultati in uscita dal progetto e i finanziatori.
    \subsection*{Strumento di monitoraggio}
Strumento che monitora costantemente lo stato di un sistema notificando l’amministratore se uno o più valori superano una fissata sogliacritica.
\newpage

\section{T}
    \subsection*{Tecnology Baseline}
Configurazione di software, hardware e processi stabiliti e documentati da utilizzare come riferimento. E’ una base di appoggio alla quale non si può retrocedere.
    \subsection*{TexStudio}
E’ un editor open-source per fogli di tipo Latex, tra le sue caratteristiche include un correttore dello spelling, un sistema auto completamento, ed un visualizzatore di  PDF integrato.
\newpage

\section{U}
    \subsection*{UML}
UML (Unified Modeling Language) è un linguaggio di modellazione in ambito di progettazione di software object oriented. Permette di rappresentare i diagrammi delle classi, dei casi d’uso, diagrammi di stato e di sequenza.
\newpage

\section{Z}
    \subsection*{Zucchetti}
Zucchetti è un’azienda italiana che produce soluzioni software, hardware e servizi per aziende, banche, assicurazioni, professionisti e associazioni di categoria.
\newpage