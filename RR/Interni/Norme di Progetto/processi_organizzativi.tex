\section{Processi organizzativi}
    \subsection{Comunicazione}
        In questa sezione vengono le presentate le norme per la comunicazione tra le parti che aderiscono al progetto.
        \subsubsection{Comunicazioni interne}
            Vengono di seguito esposte le metodologie di comunicazione all'interno del team DreamCorp.
            \newline
            Per le comunicazioni all'interno del gruppo è stato utilizzato \textit{Slack}, un'applicazione di messaggistica multipiattaforma adatto per i gruppi di lavoro. Attraverso questo strumento è possibile suddividere lo spazio di comunicazione in canali tematici, ricercare messaggi, condividere file di grosse dimensioni e tracciare i messaggi.
            \newline
            All'interno di \textit{Slack} sono stati predisposti dei canali tematici al fine di rendere più efficiente lo scambio di messaggi:
            \begin{itemize}
                \item \textbf{general}: Canale utilizzato per le comunicazioni e l'organizzazione di carattere generale;
                \item \textbf{github-notificaton} : Canale a cui è stato aggiunto un bot di GitHub il quale permette di inviare un messaggio nel momento in cui un membro del team apre/chiude issues, effettua un merge o una \textit{push};
                \item \textbf{automatismi}: Dove si discute dei software da utilizzare per creare automatismi;
                \item{random}: Canale generico dove discutere di tutto ciò che non è inerente al progetto.    
            \end{itemize}
            \newline
            Inoltre sono stati predisposti dei canali per la redazione dei documenti, uno per ogni documento, in modo da facilitare la discussione degli stessi:
            \begin{itemize}
                \item \textbf{norme_di_progetto}: Utilizzato per discutere delle norme di progetto da seguire durante lo sviluppo;
                \item \textbf{piano_progetto}: Canale riservato alla redazione del documento \textit{Piano di progetto};
                \item \textbf{studio_di_fattibilita}: Canale per la raccolta delle considerazioni sui capitolati e per la scrittura del documento;
                \item \textbf{piano_di_qualifica}: Utilizzato per discutere del raggiungimento della qualità e per l'organizzazione della redazione del documento;
                \item \textbf{analisi_requisiti}: Per discutere dei casi d'uso, requisiti e utilizzo del software per la redazione del documento \textit{Analisi_dei_requisiti v 1.0.0}; 
                \item \textbf{glossario}: Dove viene confrontato il contenuto dei glossari personali.
            \end{itemize}
            \newline
            I membri del team sono tenuti a comunicare rispettando i topic, nel caso si volesse portare all'attenzione tutti i membri del gruppo sono presenti i comandi:
            \begin{itemize}
                \item \textit{@everyone} per notificare tutti i membri del gruppo;
                \item \textit{@channel} per notificare i membri di un specifico canale.
            \end{itemize}
            
            \newline
            Altre modalità di comunicazione previste sono:
            \begin{itemize}
                \item Comunicazioni orali informali: per parlare di qualsiasi problematica, strategia, utilizzo di strumenti, consigli, dubbi;
                \item Riunioni: Incontri di persona precedentemente organizzati.
            \end{itemize}        
        \subsubsection{Comunicazioni esterne}
            Questa sezione è inerente alle modalità di comunicazione da seguire con i membri esterni al gruppo di lavoro.
            \newline
            Allo scopo di garantire un costante miglioramento della qualità di prodotto, sono stati individuati le seguenti entità esterne:
            \begin{itemize}
                \item La proponente \textbf{Zucchetti S.p.a}, rappresentata da Gregorio Piccoli, board member e CTO dell'azienda, con il quale si vuole stabilire un rapporto di collaborazione per definire bisogni e requisiti per la realizzazione del prodotto;
                \item I Committenti \textbf{Prof. Tullio Vardanega} e \textbf{Prof. Riccardo Cardin}, ai quali verrà consegnata tutta la documentazione in ciascuna fase di revisione, con i quali si vuole stabilire un rapporto utile al miglioramento dei processi e delle strategie.
            \end{itemize}
            \paragraph{Comunicazioni esterne scritte} Devono essere effettuate tassativamente attraverso l'uso dell'indirizzo mail del gruppo:
            \newline
            dreamcorp.swe@gmail.com   **********DA MODIFCARE***********
            \newline
            Ogni email ricevuta a questo indirizzo verrà inoltrata automaticamente alla casella postale di ogni membro del gruppo attraverso l'utilizzo dei filtri di Gmail, inoltre il testo delle mail verrà riportate come messaggio su \textit{Slack} attraverso l'uso di un bot.
            \paragraph{Composizione email}
            Viene di seguito presentata la modalità di scrittura delle email rivolte a soggetti esterni.
            \begin{itemize}
                \item \textbf{Email verso la Proponente Zucchetti S.p.a} 
                \item \textbf{Email verso i Committenti}: ci si propone di utilizzare un oggetto che descriva in modo più accurato possibile il contenute del messaggio, ai committenti ci si rivolgerà con il Voi e con il Lei.
            \end{itemize}
            Nella possibilità in cui un messaggio debba essere mandato a più persone, nel campo "A:" va indicato il principale destinatario, mentre nel campo "Cc:" vanno indicati tutti gli altri.
            Viene posta particolare attenzione nella modalità di risposta e di inoltro delle email. In questi casi sono da utilizzare i tasti "Rispondi a tutti" e "Inoltra a" che formatteranno in automatico il corpo del messaggio includendo le clausole "Re:" e "I:".
            \subsection{Riunioni}
                In questa sottosezione vengono presentate le modalità di riunione, sia interne che esterne.
                Allo scopo di far rispettare l'ordine del giorno, redigere il \textit{Verbale di riunione} e prendere nota degli argomenti trattati e delle decisioni prese, all'interno di ogni riunione dovrà essere nominato a turno tra i membri del gruppo DreamCorp un Segretario.
                \subsubsection{Verbali di riunione}
                   Nello svolgimento delle riunioni è compito del Segretario redigere il documento \textit{Verbale di riunione} secondo il seguente schema:
                    \begin{itemize}
                        \item \textbf{Verbale Esterno DATA}, incluso nel frontespizio. Per DATA si intende la data in cui è stata effettuata la riunione, scritta secondo le \textit{Norme Tipografiche};
                        \item \textbf{Informazioni sulla riunione}: sezione contente:
                        \begin{itemize}
                            \item \textbf{Motivo}: Descrizione sintetica del motivo che ha portato ad indirre una riunione;
                            \item \textbf{Luogo e Data}: ad esempio \textit{Padova, Giovedì 23 Febbraio 2019}; 
                            \item \textbf{Ora inizio}: nel formato ventiquattro ore;
                            \item \textbf{Ora fine}: nel formato ventiquattro ore;
                            \item \textbf{Partecipanti}: vengono elencati i partecipanti della riunione, iniziando dal Proponente/Committente nel caso la riunione sia esterna. Nel caso un membro sia costretto a lasciare la riunione prima del previsto, deve essere annotata l'ora di uscita e il motivo.
                        \end{itemize}
                        \item \textbf{Ordine del giorno}: Un elenco puntato degli argomenti discussi; 
                        \item \textbf{Resoconto}: sezione contente le annotazioni circa le decisioni prese con motivazioni e gli argomenti discussi e non.
                    \end{itemize}
                    \paragraph{Nomenclatura} 
                        I verbali esterni dovranno avere nome:
                        VER-DATA
                        \newline
                        I verbali interni saranno nominati:
                        VIR-DATA
                        \newline
                        Viene usata questa nomenclatura al fine di avere una facile consultazione.
                        \newline
                    \paragraph{Conservazione dei verbali}
                        I verbali interni ed esterni verranno redatti in \LaTeX e caricati nel repository del progetto.
                \subsubsection{Riunioni interne}
                    La partecipazione delle riunioni interne è permessa ai soli membri del gruppo DreamCorp.
                    \newline
                    \paragraph{Compiti del Responsabile di Progetto}
                        Di seguito vengono elencati i compiti che deve eseguire il Responsabile di Progetto:
                        \begin{itemize}
                            \item Fissare la data delle riunioni interne, previa discussione con i membri all'interno del canale \textit{general} su \textit{Slack};
                            \item Stabilire l'ordine del giorno;
                            \item Valutare le richieste di riunione dei membri del gruppo e decidere se accettarle;
                            \item Comunicare data e ordine del giorno attraverso il comando \textbf{@everyone} all'interno di \textit{Slack} con almeno un giorno di anticipo;
                            \item Verificare ed approvare il verbale;
                            \item Confermare spostare od annullare le riunioni sempre con le modalità di notifica a tutti i membri del gruppo.
                         \end{itemize}
                     \paragraph{Compiti dei partecipanti}
                        \begin{itemize}
                            \item Comunicare tempestivamente assenze e/o ritardi; 
                            \item Presentarsi puntualmente alle riunioni;
                            \item Partecipare attivamente alle discussioni;
                            \item Mantenere un atteggiamento educato.
                        \end{itemize}
                    \paragraph{Approvazione delle decisioni}
                        Le decisioni vengono approvate con la regola della maggioranza dei partecipanti alla riunione.
                        Affinché una riunione sia valida devono essere presenti almeno cinque membri del gruppo.
                        
                  \subsubsection{Riunioni esterne}
                  
            \subsection{Ruoli di progetto}
                Nonostante il progetto sia collaborativo, all'interno del gruppo sono stati assegnati dei ruoli corrispondenti alle figure aziendali, essi sono:
                \begin{itemize}
                    \item \textbf{Responsabile di progetto}
                \end{itemize}
                       
                     
                
            
    
     
            
            