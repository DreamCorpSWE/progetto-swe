\section{Processi primari}
    \subsection{Accordo di fornitura}
    	In questo paragrafo vengono documentate le norme che i membri devono seguire affinché il gruppo possa diventare committente dei professori Vardanega e Cardin e diventare fornitori dell'azienda Zucchetti.
    \subsubsection{Studio di fattibilità}
        Dopo la presentazione dei capitolati il gruppo si è riunito per discutere qual'era la più consona. Dopo aver risolto i dubbi interni si è optato per la scelta del capitolato \textit{numero 3} ed infine è stato redatto lo studio di fattibilità, documento, \textit{in versione 1.0.0},  atto a valutare i pro e i contro di ogni progetto permettendo così una più attenta valutazione. \newline
        I punti chiave dell'analisi sono i seguenti:
    	\begin{itemize}
		   \item \textbf{Introduzione:} Viene fatta una breve introduzione del contesto in cui applicare la soluzione;
		   \item \textbf{Finalità:} Viene descritto in modo sintetico lo scopo finale da raggiungere a lavoro completato;
		   \item \textbf{Tecnologie in uso:} Si descrivono genericamente i software che la proponente intende usare;
		   \item \textbf{Conclusioni:} Rappresenta il motivo per cui un capitolato è stato scelto oppure scartato.
	    \end{itemize}
    \subsubsection{Documentazione fornita}
	    Al fine di assicurare massima trasparenza e qualità alla proponente ed ai committenti verrano elencati i documenti forniti con una breve descrizione del loro contenuto:
    	\begin{itemize}
	        \item \textbf{Piano di progetto:} descrive la pianificazione, la consegna e il suo completamento;
	        \item \textbf{Analisi dei requisiti:} viene definita l'analisi dei casi d'uso e dei requisiti del gruppo
	        \item \textbf{Piano di qualifica:} verifica, validazione e garanzia della qualità dei processi e di prodotto.
	    \end{itemize}
    \newpage    
\subsection{Sviluppo}
    \subsubsection{Analisi dei requisiti}
    	L'Analisi dei Requisiti viene scritta dagli Analisti che hanno il compito di valutare in modo accurato ogni aspetto del progetto. In particolare questo documento è redatto con lo scopo di:
	    \begin{itemize}
	    	\item Descrivere scopo e funzionalita' del prodotto;
	    	\item Fornire i requisiti e vincoli concordati col cliente;
	    	\item Descrivere gli elementi princpali che hanno un ruolo chiave nello sviluppo del prodotto;
	    	\item Definire tutti i casi d'uso;
	    	\item Tracciare in modo dettagliato tutti i requisiti.\newline
	    \end{itemize}
    	L'Analisi dei Requisiti seguira' le specifiche descritte in seguito.\newline \newline
    	\textbf{Classificazione casi d'uso} Sono elencati in ordine dal più generico al più dettagliato ed è stato scelto il seguente criterio per la loro classificazione: \newline
	    \begin{center}
	    	UCX.Y
	    \end{center}
	    \begin{itemize}
	    	\item \textbf{Codice X:} E' il codice identificativo del caso d'uso generico che potrebbe suddividersi
	    	in casi d'uso più specifici. Nel caso non ci siano questi ultimi, il codice risulta univoco.
	    	\item \textbf{Codice Y:} E' un codice identificativo univoco per il caso d'uso. E' presente solo nel
	    	caso in cui il caso d'uso UCX abbia dei sotto casi d'uso più specifici.
	    \end{itemize}
	    X e Y sono numeri progressivi che stanno a indicare la specificità all'interno dei casi d'uso.\newline
	    Ogni caso d'uso è inoltre definito secondo la seguente struttura:\newline
	    \begin{itemize}
	    	\item \textbf{ID:} il codice del caso d'uso secondo la convenzione specificata poco sopra;
	    	\item \textbf{Nome:} titolo del caso d'uso;
	    	\item \textbf{Descrizione:} breve descrizione del caso d'uso;
	    	\item \textbf{Precondizione:} condizioni assunte come vere prima del verificarsi degli eventi del caso d'uso;
	    	\item \textbf{Postcondizione:} condizioni assunte come vere dopo il verificarsi degli eventi del caso d'uso;
	    	\item \textbf{Attori:} attori principali e secondari (se presenti) del caso d'uso;
	    	\item \textbf{Scenario Principale:} flusso degli eventi rappresentato attraverso una lista numerata.\newline
	    \end{itemize}
	    Nella \textit{Figura 1} viene riportato un esempio di caso d'uso:\newline \newline
    
	    \begin{figure}[!htbp]
	    	\centering
	    	\includegraphics{casoduso.png}
	    	\caption{Esempio di caso d'uso}
	    \end{figure}
    
	    \textbf{Classificazione dei requisiti} Tutti i requisiti ottenuti dopo una profonda analisi degli Analisti possono essere ricavati da tre diverse fonti:\newline
	    \begin{itemize}
	    	\item \textbf{Interno:} il requisito proviene da una decisione del gruppo DreamCorp, generalmente emersa durante un incontro e riportata in un verbale;
	    	\item \textbf{Capitolato:} il requisito proviene dalle richieste del capitolato;
	    	\item \textbf{Esterno:} il requisito proviene da un incontro con la proponente.\newline
	    \end{itemize}
	    Il codice utilizzato per indicizzare univocamente i requisiti è il seguente:\newline
	    \begin{center}
	    	\textbf{R+(F|Q|V|P)+(C|O)+(X(.Y)*)}
	    \end{center}
	    \begin{itemize}
	    	\item \textbf{R:} Requisito;
	    	\item \textbf{F|Q|V|P:}
	    	\begin{itemize}
	    		\item F: Requisito funzionale che descrive nel dettaglio i servizi che verranno forniti dal sistema agli attori;
	    		\item Q: Requisito di qualità;
	    		\item V: Requisito di vincolo;
	    		\item P: Requisito prestazionale;
	    	\end{itemize}
	    	\item \textbf{C|O:}
	    	\begin{itemize}
	    		\item C: Compulsory (obbligatorio);
	    		\item O: Optional (opzionale);
	    	\end{itemize}
	    	\item \textbf{X.Y:} Numeri naturali concatenati con un punto per descrivere un sottorequisito.\newline
	    \end{itemize}
	    I requisiti di vincolo, di qualità e prestazionali fanno parte dei requisiti non funzionali che descrivono i vincoli sul sistema e sul suo processo di sviluppo.
	    Ad ogni requisito verranno infine associate la sua priorità, una breve descrizione e le sue fonti come nella \textit{Figura 2}.\newline
	    
	    \begin{figure}[!htbp]
	    	\centering
	    	\includegraphics{requisiti.png}
	    	\caption{Esempio di requisito}
	    \end{figure}
	    
	    \textbf{Tracciamento} Infine, per facilitare la lettura e la visualizzazione dei requisiti, questi verranno indicizzati in due modalità specifiche:
	    \begin{itemize}
	    	\item \textbf{Tracciamento Priorità-Requisito:} il focus è orientato sulla priorità;
	    	\item \textbf{Tracciamento Tipologia-Requisito:}il focus è orientato sulla tipologia.\newline
	    \end{itemize}
	    Per una lettura immediata non sono riportate le descrizioni per le quali si rimanda alle sezioni apposite nel documento "Analisi dei Requisiti".
	    Infine viene riportata una tabella riassuntiva che permette di avere un quadro generale della distribuzione dei requisiti.\newline
	    \subsubsection{Progettazione}
	    L'attività di Progettazione consiste nel descrivere una soluzione al problema che sia soddisfacente per tutti gli stakeholders\pedice. Ciò serve a garantire che il prodotto sviluppato soddisfi le qualità, le proprietà e i bisogni nell'attività di analisi permettendo cosi di:
	    \begin{itemize}
	        \item Garantire la qualità del prodotto;
	        \item Ripartire il problema originale in maniera ricorsiva facilitando cosi la codifica delle componenti;
	        \item Ottimizzare.
	    \end{itemize}	
	    \subsubsection{Uso di diagrammi}
	    Al fine di essere il più comprensibili possibile sarà necessario far uso su larga scala di diagrammi UML\pedice 2.0