\section*{Informazioni sul documento}

\section{Introduzione}
    \subsection{Scopo del documento}
    	Questo documento si prefigge lo scopo di garantire a tutti i membri del gruppo un modo comune di lavorare al fine di aumentare l'efficenza\pedice. Verranno descritte le scelte architetturali e i vari software scelti.
    \subsection{Il prodotto}
    	Il prodotto ha lo scopo di fornire un sistema "smart" di monitoraggio dei sistemi in modo da garantire e migliorare i servizi erogati dall'azienda ai terzi. L'applicativo sarà un estensione scritta in Javascript per il software Grafana, verranno inoltre utilizzate le reti bayesiane.
    \subsection{Glossario}
    	Data la presenza di diversi elementi con significato ambiguo è stato necessario l'utilizzo di un glossario volto a disambiguare tali elementi col loro preciso significato.
\subsection{Riferimenti}
    \subsubsection{Normativi}
	    \begin{itemize}
	        \item Standard ISO/IEC 12207:1995 \newline \url{https://www.math.unipd.it/~tullio/IS-1/2009/Approfondimenti/ISO_12207-1995.pdf}
	        \item Capitolato C3 \newline \url{https://www.math.unipd.it/~tullio/IS-1/2018/Progetto/C3.pdf}
	    \end{itemize}
    \subsubsection{Informativi}
	    \begin{itemize}
	        \item Piano di Progetto v. 1.0
	        \item Piano di Qualifica v. 1.0
	        \item GitHub
	        \item Javascript
	    \end{itemize}