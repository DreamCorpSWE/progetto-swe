\section{Processi di Supporto}

	\subsection{Documentazione}
		\subsubsection{Descrizione}
			In questo capitolo sono presenti le norme adottate per 					redigere, verificare e approvare la documentazione 						ufficiale prodotta da DreamCorp. Tutti documenti sono 					elencati nella sezione \hyperref[3.1.5]{\textit{3.1.5}} denominata "Lista documenti".
		\subsubsection{Divisione dei documenti}
			Per una maggiore formalità e classificazione, ogni 						documento è stato diviso in Interno o Esterno in base alle 				seguenti caratteristiche:
			\begin{itemize}
				\item \textbf{Interno:} ha utilità interna al team;
				\item \textbf{Esterno:} condiviso anche con i Committenti e la Proponente.
			\end{itemize}
		\subsubsection{Nomenclatura}
			I documenti formali seguono uno standard preciso per la 				loro nomenclatura. Di seguito un esempio esplicativo: 					\newline 
			\begin{center}
				DocumentoDiEsempio.for
			\end{center}
			\begin{itemize}
				\item \textbf{DocumentoDiEsempio:} indica il nome del documento che viene scritto utilizzando la lettera maiuscola per ogni parola presente senza spazi al suo interno;
				\item \textbf{.for:} indica il formato del documento. Quest'ultimo sara' \textit{.tex} dalla fase di creazione fino alla sua approvazione e da quel momento in poi verra' creato il file \textit{.pdf} contenente la versione definitiva di quel documento.
			\end{itemize}
			\textbf{Codice per il versionamento} La versione del documento e' specificata all' interno del documento stesso in una tabella dove sono segnate in modo preciso tutte le modifiche fatte fino ad arrivare alla major release in formato PDF. Il sistema utilizzato verra' spiegato in modo piu' dettagliato successivamente nella sezione \hyperref[3.2.2]{\textit{3.2.2}} di questo documento.
		\subsubsection{Ciclo di vita documentazione}
			Per ogni documento formale sono previste tre fasi obbligatorie:
			\begin{itemize}
			\item \textbf{Redazione:} questa fase dura dalla creazione del documento fino alla scrittura completa di tutti i contenuti previsti. Durante questo processo, il Responsabile di Progetto assegna ai Redattori i contenuti da aggiungere e una volta che quest'ultimi saranno esauriti potra' approvare il passaggio alla fase successiva;
			\item \textbf{Verifica:} in questa fase il documento viene controllato in ogni sua parte attraverso le opportune procedure dai Verificatori che una volta ultimato il lavoro daranno il loro feedback al Responsabile di Progetto; questo puo' dunque approvare il documento che passa quindi alla fase successiva oppure puo' farlo tornare alla fase di Redazione per correggere eventuali imperfezioni riscontrate dai Verificatori;
			\item \textbf{Approvato:} In questa fase il documento e'pronto per essere rilasciato e la sua versione viene aggiornata ad una major.
			\end{itemize}
		\subsubsection{Lista documenti}
		\label{3.1.5}
			\begin{itemize}
				\item \textbf{Analisi dei Requisiti:} \newline
				Classificato come documento Esterno, il suo scopo e' di analizzare ed esporre i requisiti del progetto. Contiene l'analisi dei casi d'uso che riguardano il prodotto in fase di sviluppo e i diagrammi di interazione con l'utente a prodotto ultimato;
				\item \textbf{Norme di Progetto:} \newline
				
				\item \textbf{Piano di Progetto:} \newline
				
				\item \textbf{Piano di Qualifica:} \newline
				
				\item \textbf{Studio di Fattibilita':} \newline
				
				\item \textbf{Glossario:} \newline
				
			\end{itemize}
		\subsubsection{Norme tipografiche}

		\subsubsection{Struttura dei documenti}

		\subsubsection{Strumenti di supporto}

	\subsection{Versionamento}
		\subsubsection{Comandi di base}

		\subsubsection{Numerazione della versione}
		\label{3.2.2}

	\subsection{Verifica}
		\subsubsection{Analisi statica}

		\subsubsection{Analisi dinamica}

		\subsubsection{Strumenti di verifica}