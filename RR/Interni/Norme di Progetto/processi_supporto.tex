\section{Processi di Supporto}

	\subsection{Documentazione}
		\subsubsection{Descrizione}
			In questo capitolo sono presenti le norme adottate per 	redigere, verificare e approvare la documentazione 			ufficiale prodotta da DreamCorp. Tutti documenti sono 		elencati nella sezione \hyperref[3.1.5]{\textit{\underline{3.1.5}}} denominata "Lista documenti".
		\subsubsection{Divisione dei documenti}
			Per una maggiore formalità ogni documento è stato classificato in Interno o Esterno in base alle seguenti caratteristiche:
			\begin{itemize}
				\item \textbf{Interno:} ha utilità interna al team, ovvero contiene tutte le informazioni significative per componenti del gruppo in fase di sviluppo;
				\item \textbf{Esterno:} condiviso anche con i Committenti e la Proponente, espone le informazioni utili per spiegare il metodo di lavoro seguito per lo sviluppo del progetto.
			\end{itemize}
		\subsubsection{Nomenclatura}
			I documenti formali seguono uno standard preciso per la loro nomenclatura. Di seguito un esempio esplicativo: 					\newline 
			\begin{center}
				DocumentoDiEsempio.for
			\end{center}
			\begin{itemize}
				\item \textbf{DocumentoDiEsempio:} indica il nome del documento che viene scritto utilizzando la lettera maiuscola per ogni parola presente senza spazi al suo interno;
				\item \textbf{.for:} indica il formato del documento. Quest'ultimo sarà \textit{.tex} dalla fase di creazione fino alla sua approvazione e da quel momento in poi verrà creato il file \textit{.pdf} contenente la versione definitiva di quel documento.
			\end{itemize}
			\textbf{Codice per il versionamento} La versione del documento e' specificata all'interno del documento stesso in una tabella dove sono segnate in modo preciso tutte le modifiche fatte fino ad arrivare alla major release in formato PDF. Il sistema utilizzato verrà spiegato in modo più dettagliato successivamente nella sezione \hyperref[3.2.2]{\textit{\underline{3.2.2}}} di questo documento.
		\subsubsection{Ciclo di vita documentazione}
			Per ogni documento formale sono previste tre fasi obbligatorie:
			\begin{itemize}
			\item \textbf{Redazione:} questa fase dura dalla creazione del documento fino alla scrittura completa di tutti i contenuti previsti. Durante questo processo, il Responsabile di Progetto assegna ai Redattori i contenuti da aggiungere e una volta che quest'ultimi saranno esauriti potrà approvare il passaggio alla fase successiva;
			\item \textbf{Verifica:} in questa fase il documento viene controllato in ogni sua parte attraverso le opportune procedure dai Verificatori che una volta ultimato il lavoro daranno il loro feedback al Responsabile di Progetto; questo può dunque approvare il documento che passa quindi alla fase successiva oppure può farlo tornare alla fase di Redazione per correggere eventuali imperfezioni riscontrate dai Verificatori;
			\item \textbf{Approvato:} In questa fase il documento e' pronto per essere rilasciato e la sua versione viene aggiornata ad una major.
			\end{itemize}
		\subsubsection{Lista documenti}
		\label{3.1.5}
			\begin{itemize}
				\item \textbf{Analisi dei Requisiti: }[Esterno] \newline
				Il suo scopo è di analizzare ed esporre i requisiti del progetto. Contiene l'analisi dei casi d'uso che riguardano il prodotto in fase di sviluppo e i diagrammi di interazione con l'utente a prodotto ultimato;
				\item \textbf{Norme di Progetto: }[Interno] \newline
				Contiene tutte le regole e le convenzioni adottate dai membri del gruppo DreamCorp per uno sviluppo metodico del progetto;
				\item \textbf{Piano di Progetto: }[Esterno] \newline
				Il suo scopo è quello di descrivere gli obiettivi del progetto e gli elementi necessari per il loro raggiungimento. Inoltre vien presentato come il gruppo DreamCorp amministra il proprio tempo e i membri al suo interno;
				\item \textbf{Piano di Qualifica: }[Esterno] \newline
				Consiste in una spiegazione dettagliata di come il team DreamCorp vuole soddisfare i requisiti di qualità del progetto;
				\item \textbf{Studio di Fattibilità: }[Interno] \newline
				Il suo obiettivo è di mostrare come è stato analizzato ogni capitolato elencando per ognuno punti a favore e sfavore in modo da evidenziare tutti i dubbi sorti in fase di decisione.
				\item \textbf{Glossario: }[Esterno] \newline
				Contiene tutti i termini presenti nei documenti formali che il gruppo DreamCorp ha ritenuto opportuno avessero bisogno di una spiegazione o chiarimento per facilitarne la comprensione. E' unico per tutti i documenti.
			\end{itemize}
		\subsubsection{Norme tipografiche}
			Le norme scritte in questa sezione devono essere seguite in tutti i documenti prodotti dal team DreamCorp:
			\begin{itemize}
				\item \textbf{Date:} sono scritte seguendo la regola anno-mese-giorno (DD-MM-YYYY);
				\item \textbf{Voci nel Glossario:} per segnalare una parola che si trova nel glossario viene posta una \textbf{G} a pedice della parola;
				\item \textbf{Link interni:} i collegamenti che rimandano a una sezione interna al documento sono scritti in corsivo e sottolineati;
				\item \textbf{Link esterni:} i collegamenti che rimandano a una pagina web esterna al documento sono scritti in colore blu;
				\item \textbf{Elenchi puntati:} ogni elemento della lista è caratterizzato in generale da una prima parola o serie di parole in grassetto seguite da due punti \textit{":"} e successivamente dal suo contenuto, ma può anche essere formato solamente dal contenuto nei casi in cui non sia necessaria una particolare specifica iniziale. Alla fine è posto un punto e virgola \textit{";"} tranne per l'ultimo elemento che è concluso da un punto \textit{"."};
				\item \textbf{Citazioni:} le citazioni sono scritte in \textit{corsivo}.
			\end{itemize}
		\subsubsection{Struttura dei documenti}
			Tutti i documenti utilizzano una struttura di base contenuta nel file \textit{stile.tex} in modo da non ripetere ogni volta gli elementi comuni.
			\newline \newline \textbf{Frontespizio} E' la prima pagina del documento dove si trovano:
			\begin{itemize}
				\item \textbf{Logo:} immagine identificativa del gruppo;
				\item \textbf{Nome del progetto:} G\&B;
				\item \textbf{Nome del documento};
				\item \textbf{Data:} data in cui è stato approvata la versione corrente in cui è stato approvato il documento;
				\item \textbf{Versione:} versione corrente del documento;
				\item \textbf{Nome del gruppo:} DreamCorp.
			\end{itemize}
			In seguito in forma tabellare sono riportati:
			\begin{itemize}
				\item \textbf{Responsabile:} persona a capo del progetto nel momento in cui è stato approvato;
				\item \textbf{Redattori:} persone incaricate nella stesura del documento;
				\item \textbf{Verificatori:} persone incaricate di controllare la qualità del documento;
				\item \textbf{Uso:} Interno o Esterno;
				\item \textbf{Destinatari:} persone alle quali è destinate il documento.
			\end{itemize}
		\textbf{Diario delle modifiche}  Si trova a partire dalla seconda pagina del documento e riporta sotto forma di tabella tutte le modifiche apportate al documento. La tabella è composta da quattro colonne che contengono rispettivamente descrizione della modifica, autore della modifica, data della modifica e versione a cui è arrivato il documento.
		\newline \newline \textbf{Indice}  E' posto sempre dopo il Diario delle Modifiche e contiene l'elenco dei capitoli e sottocapitoli in cui è diviso il documento.
		\newline \newline \textbf{Elenco delle Tabelle} Contiene la lista delle tabelle presenti nel documento; è presente solo quando necessaria, ovvero se nel documento si trova almeno una tabella.
		\newline \newline \textbf{Elenco delle Figure} Contiene la lista delle immagini presenti nel documento; è presente solo se nel documento c'è almeno un'immagine.
		\newline \newline \textbf{Contenuto}  Il documento è completato dal suo contenuto. Ogni pagina è composta come di seguito:\newline
		\begin{itemize}
			\item \textbf{Logo del team:} in alto a sinistra nell'intestazione;
			\item \textbf{Capitolo:} in alto a destra nell'intestazione;
			\item \textbf{Contenuto:} occupa l'intera pagina;
			\item \textbf{Nome Documento:} in basso a sinistra nel piè di pagina;
			\item \textbf{Numero Pagina:} in basso a destra nel piè di pagina. \newline
		\end{itemize}
		L'intestazione e il piè di pagina sono separate dal resto del documento attraverso due righe nere orizzontali. Ogni nuovo capitolo (identificato in \LaTeX ~con il comando \textit{\textbackslash{}section}) comincia sempre in una nuova pagina.
		\subsubsection{Strumenti di supporto}
		\label{3.1.8}
			Il formato scelto per la stesura della documentazione è il  \LaTeX\pedice ~che permette una maggiore precisione e uniformità fra tutti i membri del gruppo. L'ambiente di sviluppo principale scelto è \textbf{TeXstudio\pedice} il quale è risultato il più completo in termini di funzioni.
			E' stato anche utilizzato l'editor \LaTeX ~online Overleaf\pedice essendo molto comodo per modifiche rapide e la condivisione dei file tra tutti i membri del gruppo.
	\subsection{Versionamento}
		E' stata scelta la tecnologia Git per le parti del progetto e della documentazione che necessitano di un versionamento e il servizio utilizzato è GitHub. Per tutte le altre parti lo strumento di condivisione di cui si serve il gruppo DreamCorp è una cartella su Google Drive\pedice.
		\subsubsection{Comandi di base}
			In seguito vengono descritti i comandi principali di cui si è servito il team per l'uso di Git all'interno del progetto:
			\begin{itemize}
				\item \textbf{git clone:} crea una copia in locale del repository\pedice;
				\item \textbf{git pull:} aggiorna la cartella in locale col contenuto del repository in remoto;
				\item \textbf{git checkout:} permette di cambiare branch\pedice su cui si sta lavorando;
				\item \textbf{git add:} aggiunge uno o più file a seconda del comando alla lista dei file tracciati da Git. Aggiungendo il simbolo "." si possono aggiungere tutti i file modificati con un solo comando;
				\item \textbf{git commit -m "\textit{Messaggio}":} fa il commit\pedice delle modifiche effettuate in locale di tutti i file che sono stati aggiunti attraverso il comando \textit{git add}. Il messaggio da inserire non è opzionale e deve descrivere lo scopo del commit;
				\item \textbf{git push:} carica in remoto le modifiche effettuate in locale;
				\item \textbf{git status:} permette di consultare  lo stato del repository locale mostrando i file tracciati e non. \newline
			\end{itemize}
			Inoltre, per migliorare il workflow generale è stato utilizzato GitFlow che permette una gestione più accurata dei branch.
		\subsubsection{Numerazione della versione}
		\label{3.2.2}
			La versione di tutti i documenti è rappresentata nella forma X.Y.Z seguendo le seguenti regole:
			\begin{itemize}
				\item \textbf{X:} è il numero di una major release che avviene una volta che il documento passa allo stato di Approvato;
				\item \textbf{Y:} numero di una release normale nella quale sono state apportate modifiche o aggiunte sostanziali al documento;
				\item \textbf{Z:} è il numero di una minor release nella quale vengono apportate piccole modifiche o correzioni.
			\end{itemize}
		L'aumento di una cifra comporta l'azzeramento di quelle alla sua destra. \newline
		
	\subsection{Verifica}
		\subsubsection{Analisi statica}
			L'analisi statica della documentazione e del codice è divisa in due fasi, Walkthrough e Inspection:
			\begin{itemize}
				\item \textbf{Walkthrough:} questa fase consiste nella lettura di tutto il documento da esaminare e nella creazione di una lista degli errori trovati da utilizzare nella fase successiva.
				\item \textbf{Inspection:} questa fase consiste nella rilettura del documento attraverso la lista creata precedentemente per un'analisi mirata. \newline
			\end{itemize}
		Gli strumenti utilizzati per l'Analisi statica si possono trovare nella sezione \hyperref[3.1.8]{\textit{\underline{3.1.8}}} di questo documento.\newline
		\subsubsection{Analisi dinamica}
			Essendo applicato solo al codice del software, il processo di Analisi dinamica deve ancora essere definito in modo definitivo; in generale consisterà nella creazione e esecuzione di vari test.\newline
		\subsubsection{Strumenti di verifica del software}
		Gli strumenti di verifica del software verranno definiti in fase di programmazione e codifica del software stesso.\newline
		
	\subsection{Metriche per la Qualità di Processo}
	Verranno  utilizzate  le  seguenti  metriche  per  valutare  l’efficienza  e  l’efficacia  dei processi.
	 	\subsubsection{Schedule Variance (SV)} Indica se si è in linea, in anticipo o in ritardo, rispetto alla schedulazione delle
attività di progetto pianificate nella baseline.
	 	E' un indicatore di efficacia soprattutto nei confronti del Cliente.
	 	Se il valore SV ottenuto è positivo significa che il progetto sta procedendo con una
maggiore velocità rispetto a quanto pianificato viceversa se negativo.\newline
	 	\textbf{Misurazione:}
\newline
	 	\[
	 		SV = BCWP-BCWS \newline
	 	\]
 		Dove:
 		\begin{itemize}
 			\item \textbf{BCWP (Budgeted Cost of Work Performed):} E' il valore (in giorni
o Euro) delle attività realizzate alla data corrente. Rappresenta il valore
prodotto dal progetto ossia la somma di tutte le parti completate e di tutte
le porzioni completate delle parti ancora da terminare;
 			\item \textbf{BCWS (Budgeted Cost of Work Scheduled):} E' il costo pianificato (in giorni o Euro) per realizzare le attività di progetto alla data corrente. \newline
 		\end{itemize}
	 	\subsubsection{Budget Variance (BV)} Indica se alla data corrente si è speso di più o di meno rispetto a quanto previsto
a budget alla data corrente.
	 	E' un indicatore che ha un valore unicamente contabile e finanziario.
	 	Se il valore BV ottenuto è positivo significa che il progetto sta spendendo il proprio
budget con minor velocità di quanto pianificato, viceversa se negativo.\newline
	 	\textbf{Misurazione:}
\newline
	 	\[
	 		BV = BCWS-ACWP \newline
	 	\]
	 	Dove:
	 	\begin{itemize}
	 		\item \textbf{BCWS (Budgeted Cost of Work Scheduled):} E' il costo pianificato (in Euro) per realizzare le attività di progetto alla data corrente;

	 		\item \textbf{ACWP (Actual Cost of Work Performed):} E' il costo effettivamente sostenuto (in Euro) alla data corrente.
\newline
	 	\end{itemize}
	 	\subsubsection{Code Coverage} Verranno usati i seguenti criteri per poter avere una misura di codice testato e verificato:\newline
	 		\begin{itemize}
	 			\item \textbf{Line coverage:} primitiva rispetto alle successive, fornisce un’idea generale del codice. Verificare se ogni linea è stata utilizzata;
	 			\item \textbf{Functional coverage:} Verificare che ogni funzione sia stata chiamata;
	 			\item \textbf{Path coverage:} Verificare che ogni percorso indipendente nel programma sia eseguito almeno una volta;
	 			\item \textbf{Condition coverage:} Verificare che ogni percorso di ogni espressione booleana sia coperto dai test; 
	 			\item \textbf{Branch coverage:} Verificare se tutti i possibili branch (derivanti da if e case statement) sono stati eseguiti.
	 		\end{itemize}
		 \subsubsection{Servizi esterni non raggiungibili} Numero totale di giorni in cui siano stati offline o bloccati servizi usati.\newline
		 \textbf{Misurazione:} Indice numerico incrementato partendo da zero per ogni giorno
in cui i servizi utilizzati dal gruppo siano risultati totalmente offline per la maggior parte del giorno.\newline
		\subsubsection{Rischi non calcolati} Indice numerico indica la quantità di rischi esterni a quelli presenti nell’attività di
analisi dei rischi rilevati nella corrente fase di progetto.
\newline
		\textbf{Misurazione:} Indice numerico incrementato partendo da 0 per ogni rischio che
si manifesta senza essere stato individuato precedentemente nella lista di rischi.
		Viene resettato all’inizio di ogni nuova fase di progetto.
		\newline
	\subsection{Metriche per la Qualità di Prodotto}
		Verranno  utilizzate  le  seguenti  metriche  per  valutare  l’efficienza  e  l’efficacia  dei prodotti. \newline Per i documenti sono usate le seguenti metriche:
		\subsubsection{Gulpease Index} L’indice Gulpease\pedice è un indice di leggibilità del testo tarato sulla
		lingua italiana. Differentemente da indici di lingua straniera, ha il vantaggio di controllare
la lunghezza delle parole anzichè il numero di sillabe per parola, semplificandone il calcolo
automatico.
		Nel complesso, l’indice Gulpease considera la lunghezza delle parole, il numero delle frasi
ed il numero delle parole totali, venendo calcolato con la seguente formula:\newline
		\[
		 	89+\frac{300 \times (numero ~delle ~frasi) - 10 \times (numero ~delle ~lettere)}{(numero ~delle ~parole)}
		\]\newline	
		Il valore risultante è compreso tra 0 e 100, dove un indice più alto corrisponde ad un
indice di leggibilità più semplice.
		Le soglie dei valori dell’indice di leggibilità Gulpease sono:
		\begin{itemize}
			\item inferiore a 80, il documento è difficile da leggere per chi ha la licenza elementare;
			\item  inferiore a 60, il documento è difficile da leggere per chi possiede la licenza media;
			\item inferiore a 40, il documento è difficile da leggere per chi ha un diploma superiore.
			\newline
		\end{itemize}
		\subsubsection{Errori Ortografici} Gli errori ortografici possono essere identificati tramite lo strumento ‘Controllo
ortografico’ presente in TexStudio. Sarà poi compito dei Verificatori correggerli.
\newline
		\subsubsection{Gunning Fog Index} E' un indice per misurare la facilità di lettura e di comprensione di un testo. Il numero risultante è un indicatore del numero di anni di educazione formale della quale una persona necessita al fine di leggere il testo con facilità e si misura con la seguente formula:\newline
		\[
		GFI=0.4\times((\frac{numero ~parole}{numero ~frasi})+100\times(\frac{numero ~parole ~complesse}{numero ~parole}))
		\]\newline
		Più l'indice risultante è alto più di difficile comprensione è il testo; in generale un indice minore di 12 si riferisce a un testo destinato a un vasto pubblico.
		\subsubsection{Simple Measure Of Gobbledygook (SMOG)} E' una misura di leggibilità che stima gli anni di educazione necessari per la comprensione di un testo. Viene calcolato attraverso la seguente formula: \newline
		\[
		SMOG=1.0430\times\sqrt{numero ~di ~polisillabi\times\frac{30}{numero ~di ~frasi}}+3.1291
		\]\newline
		Dove per $polissilabi$ si intendono tutte le parole con tre o più sillabe.
		\newline \newline \newline
		Per il software sono usate le seguenti metriche:\newline
		\subsubsection{Percentuale requisiti fondamentali soddisfatti} Indica la percentuale dei requisiti obbligatori coperti dall’implementazione. La sua
		formula di misurazione è la seguente:\newline
		\[
		PRFS=(\frac{N_{RFS}}{N_{RF}}) \times 100
		\]	
		Dove $N_{RFS}$ è il numero di requisiti fondamentali soddisfatti e $N_{RF}$ è il numero totale
		dei requisiti fondamentali.
		\subsubsection{Percentuale requisiti opzionali soddisfatti} Indica la percentuale dei requisiti opzionali coperti dall’implementazione. La sua formula di misurazione è la seguente:\newline
		\[
		PROS=(\frac{N_{ROS}}{N_{RO}}) \times 100
		\]	
		Dove $N_{ROS}$ è il numero di requisiti opzionali soddisfatti e $N_{RO}$ è il numero totale
		dei requisiti opzionali.
		\subsubsection{Mean Time Between Failures (MTBF)} E' il valore medio calcolato in unità di tempo tra una failure e la successiva. Viene calcolato attraverso la seguente formula:\newline
		\[
		MTBF=\frac{\sum(inizio ~del ~Downtime - ~inizio ~dell'Uptime)}{numero ~di ~failures}
		\]\newline
		$MTBF$ è un indice molto utile per quantificare l'affidabilità di un sistema.
		\subsubsection{Blocco operazioni non corrette} Indica la percentuale di funzionalità in grado di gestire correttamente i fault che
potrebbero verificarsi. La sua formula di misurazione è la seguente:
\newline
		\[
		BNC=(\frac{N_{FE}}{N_{ON}}) \times 100
		\]	
		Dove $N_{FE}$ è il numero di failure evitati durante i test effettuati e $N_{ON}$ è il numero
di test-case eseguiti che prevedono l’esecuzione di operazioni non corrette, causa
		di possibili failure.
		\subsubsection{Test conclusi in failure} Indica la percentuale di testing che si sono concluse in failure. La sua formula di
misurazione è la seguente:
 \newline
		\[
		TCF=(\frac{N_{FR}}{N_{TE}}) \times 100
		\]	
		Dove $N_{FR}$ è il numero di failure rilevati durante l'attività di testing e $N_{TE}$
è il numero di test-case eseguiti.
		\subsubsection{Tempo di risposta} Indica il tempo medio che intercorre fra la richiesta software di una determinata
funzionalità e la restituzione del risultato all’utente. La sua formula di misurazione
è la seguente:\newline
		\[
		TR=\frac{\sum_{i=1}^{n}T_i}{n}
		\]	
		Dove $T_i$ è il tempo intercorso fra la richiesta \textit{i} di una funzionalità ed il comportamento delle operazioni necessarie a restituire un risultato a tale richiesta.
		\subsubsection{Impatto nuove aggiunte} Indica la percentuale di nuove aggiunte effettuate in risposta a failure che hanno portato
all’introduzione di nuove failure in altre componenti del sistema. La sua formula
		di misurazione è la seguente:
\newline
		\[
		INA=(\frac{N_{FRF}}{N_{FR}}) \times 100
		\]
		Dove $N_{FRF}$ è il numero di failure risolte con l’introduzione di nuove failure e $N_{FR}$
		è il numero di failure risolte.
	