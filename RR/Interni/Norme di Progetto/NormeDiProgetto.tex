% !TeX spellcheck = en_US
\documentclass[a4paper,12pt]{article}

\usepackage[italian]{babel}
\usepackage[T1]{fontenc}
\usepackage[utf8]{inputenc}

\title{\myfont Norme di Progetto}
\font\myfont=cmr12 at 40pt
\date{\myfont 04-12-2018}
\begin{document}

  \maketitle
  \begin{center}
  \huge Versione 1.00 
  \\G\&B
  \end{center}
  \newpage
  \tableofcontents
  \newpage

\section*{Informazioni sul documento}


\section{Introduzione}
\subsection{Scopo del documento}
Questo documento si prefigge lo scopo di garantire a tutti i membri del gruppo un modo comune di lavorare al ifne di aumentare l'efficenza. Verranno descritte le scelte architetturali e i vari software scelti
\subsection{Il prodotto}
Il prodotto ha lo scopo di fornire un sistema "smart" di monitoraggio dei sistemi in modo da garantire e migliorare i servizi erogati dall'azienda ai terzi. L'applicativo sarà un estensione scritta in Javascript per il software Grafana, verranno inoltre utilizzate le reti bayesiane.
\subsection{Glossario}
Data la presenza di diversi elementi con significato ambiguo è stato necessario l'utilizzo di un glossario volto a disambiguare tali elementi col loro preciso significato.
\subsection{Riferimenti}

\newpage

\section{Processi primari}
\subsection{Accordo di fornitura}
In questo paragrafo vengono documentate le norme che i membri devono seguire affinchè il gruppo possa diventare committente dei professori Tullio e Cardin e diventare fornitori dell'azienda Zucchetti.
\subsubsection{Studio di fattibilità}
Dopo la presentazione dei capitolati il gruppo si è riunito per discutere qual'era la più consona. Dopo aver risolto i dubbi interni si è optato per la scelta del capitolato \textit{numero 3} ed infine parte del gruppo ha steso lo studio di fattibilità, documento, \textit{in versione 1.0.0},  atto a valutare i pro e i contro di ogni progetto permettendo così una più attenta valutazione. \newline
I punti chiave dell'analisi sonon i seguenti:
\begin{itemize}
	\item \textbf{Introduzione:} Viene fatta una breve introduzione del contesto in cui applicare la soluzione; 
	\item \textbf{Tecnologie in uso:} Si descrivono genericamente i software che la proponente intende usare;
	\item \textbf{Conclusioni:} Rappresenta il motivo per cui un capitolato è stato scelto oppure scartato.
\end{itemize}
\subsubsection{Documentazione fornita}
Al fine di assicurare massima trasparenza e qualità alla proponente ed ai committenti verrano elencati i documenti forniti con una breve descrizione del loro contenuto:
\begin{itemize}
\item \textbf{Piano di progetto:} descrive la pianificaziona, la consegna e il suo completamento;
\item \textbf{Analisi dei requisiti:} viene definita l'analisi dei casi d'uso e dei requisiti del gruppo
\item \textbf{Piano di qualifica:} verifica e validazione e garanzia della qualità dei processi e di prodotto.
\end{itemize}
 
\subsection{Sviluppo}
\subsubsection{Analisi dei requisiti}
Con l'analisi dei requisiti è stato preso in analisi ogni capitolato valutandone pro, contro e dando una valutazione finale. Questo documento è stato redatto con lo scopo di:
\begin{itemize}
\item 
\item 
\end{itemize}
\subsubsection{Progettazione}
L'attività di Progettazione consiste nel descrivere una soluzione al problema che sia soddisfacente per tutti gli stackeholders. Cio serve a garantire che il prodotto sviluppato soddisfi le qualità, le proprieta e i bisogni nell'attvita di analisi permettendo cosi di:
\begin{itemize}
\item Garantire la qualita del prodotto ;
\item Ripartire il problema originale in maniera ricorsiva facilitando cosi la codifica delle componenti;
\item Ottimizzare.
\end{itemize}	
\subsubsection{Uso di diagrammi}
Al fine di essere il piu comprensibili possibile sara necessario far uso su larga scala di diagrammi uml 2.0
\end{document}
