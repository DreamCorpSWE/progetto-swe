% !TeX spellcheck = en_US
\documentclass[12pt]{article}

\usepackage[italian]{babel}

\title{\myfont Norme di Progetto}
\font\myfont=cmr12 at 40pt
\date{ \myfont 04-12-2018}


\begin{document}
  \maketitle
  \begin{center}
  \huge Versione 1.00 
  \\G\&B
  \end{center}
  \newpage
  \tableofcontents
  \newpage

\section{Informazioni sul documento}


\section{Introduzione}
\subsection{Scopo del documento}
Questo documento si prefigge lo scopo di garantire a tutti i membri del gruppo un modo comune di lavorare al ifne di aumentare l'efficenza. Verranno descritte le scelte architetturali e i vari software scelti
\subsection{Il prodotto}
Il prodotto ha lo scopo di fornire un sistema "smart" di monitoraggio dei sistemi in modo da garantire e migliorare i servizi erogati dall'azienda ai terzi. L'applicativo sarà un estensione scritta in Javascript per il software Grafana, verranno inoltre utilizzate le reti bayesiane.
\subsection{Glossario}
Data la presenza di diversi elementi con significato ambiguo è stato necessario l'utilizzo di un glossario volto a disambiguare tali elementi col loro preciso significato.
\subsection{Riferimenti}

\newpage

\section{Processi primari}
\subsection{Accordo di fornitura}
In questo paragrafo vengono documentate le norme che i membri devono seguire affinchè il gruppo possa diventare committente dei professori Tullio e Cardin e diventare fornitori dell'azienda Zucchetti.
\subsection{Studio di fattibilità}
Dopo la presentazione dei capitolati il gruppo si è riunito per discutere qual'era la più consona. Dopo aver risolto i dubbi interni si è optato per la scelta del capitolato \textit{numero 3} ed infine parte del gruppo ha steso lo studio di fattibilità, documento, \textit{in versione 1.0.0}, che permette di valutare più approfonditamente le scelte e l'analisi dei rischi che ci ha portato a questa scelta.

\end{document}