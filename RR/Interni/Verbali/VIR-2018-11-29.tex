\documentclass[12pt]{article}

\newcommand{\documento}{Verbale 2018-11-22}
\usepackage{fancyhdr}
\usepackage{titling}
\usepackage{caption}
\usepackage{multirow}
\usepackage{tabularx}
\usepackage[T1]{fontenc}
\usepackage[utf8]{inputenc}
\usepackage[italian]{babel}
\usepackage{lastpage}
\usepackage{nopageno}
\usepackage{graphicx}
\usepackage [colorlinks=true,urlcolor=blue, linkcolor=black]{hyperref}
\newcommand{\subtitle}[1]{%
    \posttitle{%
        \par\end{center}
        \begin{center}\LARGE#1\end{center}

    \vskip0.5em}%
}

\setlength{\oddsidemargin}{0in}
\setlength{\evensidemargin}{0in}
\setlength{\topmargin}{0in}
\iffalse \setlength{\headsep}{-.25in}\fi
\setlength{\textwidth}{6.5in}
\setlength{\textheight}{8.5in}

\font\myfont=cmr12 at 40pt

\pagestyle{fancy}
\fancyhf{}
\rhead{\leftmark}
\lhead{\includegraphics[width = 20mm]{../logo.png}}
\rfoot{Pagina \thepage \space di \pageref{LastPage}}
\lfoot{Studio di fattibilità}
\renewcommand{\footrulewidth}{0.4pt}
\newcommand{\red}{TBD}
\newcommand{\verp}{TBD}
\newcommand{\vers}{TBD}
\newcommand{\res}{TBD}
\newcommand{\version}{Versione TBD}
\newcommand{\use}{Interno}
\title{\myfont Verbale interno 2018-11-29}
\author{Dream Corp.}
\date{\myfont 2018-11-30}

\begin{document}
	\maketitle
	\begin{center}
		\includegraphics[width = 70mm]{../../logo.png}\newline
		\huge \version 
		\\G\&B
		
		\begin{table}[h!]
			\centering
			\begin{tabular}{r|l}
					\multicolumn{2}{c}{Informazioni sul documento}\\
			        \hline
        			Redazione & \pie\\
        			Verifica & Marco Davanzo\\
        			Responsabile & Davide Ghiotto\\
        			Uso & Interno\\
        			Destinatari & Dream Corp. \\
        			& Zucchetti SpA\\
        			& Prof. Tullio Vardanega\\
        			& Prof. Riccardo Cardin\\
			\end{tabular}
		\end{table}
		
\end{center}


\section{Riunione}
    \subsection{Informazioni generali}
    \begin{itemize}
        \item \textbf{Motivo della riunione}: È stata indetta questa riunione per decidere il nome e il logo del gruppo, per definire i principali strumenti organizzativi e per scegliere definitivamente il capitolato.
        \item \textbf{Luogo e data}: Aula 1AD100 Torre Archimede, Giovedì 29 Novembre 2018;
        \item \textbf{Orario}: 12:30-14:30;
        \item \textbf{Partecipanti}: Tutti i membri del gruppo.

    \end{itemize}
        \newpage
\section{Ordine del giorno}
    \begin{itemize}
        \item Logo del gruppo;
        \item Nome del gruppo;
        \item Scelta degli strumenti organizzativi;
        \item Scelta del capitolato.

    \end{itemize}
    
\newpage
\section{Resoconto}
    \subsection{Argomenti}
        Di seguito sono riportati i punti del giorno discussi:
    \begin{enumerate}
        \item \textbf{Nome del gruppo:} il nome che è stato scelto per il gruppo è "DreamCorp" che rimanda a "Dream corporation" cioè "Società dei sogni". Il logo del gruppo è stato progettato da \daG;
        \item \textbf{Strumenti organizzativi:} Sono stati scelti i seguenti strumenti organizzativi: 
        \begin{itemize}
            \item \textbf{E-mail} dreamcorp.swe@gmail.com per la comunicazione con i destinatari del progetto: Zucchetti Spa \pedice , Prof. Tullio Vardanega, Prof. Riccardo Cardin; 
            \item \textbf{Slack} per la comunicazione tra i membri del gruppo. Sono stati creati tanti canali quanti i documenti, oltre ad uno generale, dover poter scrivere in base alle esigenze;
            \item \textbf{Github} \pedice per la redazione dei documenti. Sono stati predisposti dei canali per Github \pedice, uno per ogni documento.
        \end{itemize}
        \item \textbf{Scelta del capitolato:} è stato scelto il capitolato C3 per i motivi descritti nello \textit{Studio di fattibilità}. 
    \end{enumerate}

\end{document}
