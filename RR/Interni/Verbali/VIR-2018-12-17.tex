\documentclass[12pt]{article}

\newcommand{\documento}{Verbale 2018-11-22}
\usepackage{fancyhdr}
\usepackage{titling}
\usepackage{caption}
\usepackage{multirow}
\usepackage{tabularx}
\usepackage[T1]{fontenc}
\usepackage[utf8]{inputenc}
\usepackage[italian]{babel}
\usepackage{lastpage}
\usepackage{nopageno}
\usepackage{graphicx}
\usepackage [colorlinks=true,urlcolor=blue, linkcolor=black]{hyperref}
\newcommand{\subtitle}[1]{%
    \posttitle{%
        \par\end{center}
        \begin{center}\LARGE#1\end{center}

    \vskip0.5em}%
}

\setlength{\oddsidemargin}{0in}
\setlength{\evensidemargin}{0in}
\setlength{\topmargin}{0in}
\iffalse \setlength{\headsep}{-.25in}\fi
\setlength{\textwidth}{6.5in}
\setlength{\textheight}{8.5in}

\font\myfont=cmr12 at 40pt

\pagestyle{fancy}
\fancyhf{}
\rhead{\leftmark}
\lhead{\includegraphics[width = 20mm]{../logo.png}}
\rfoot{Pagina \thepage \space di \pageref{LastPage}}
\lfoot{Studio di fattibilità}
\renewcommand{\footrulewidth}{0.4pt}
\newcommand{\red}{TBD}
\newcommand{\verp}{TBD}
\newcommand{\vers}{TBD}
\newcommand{\res}{TBD}
\newcommand{\version}{Versione TBD}
\newcommand{\use}{Interno}
\title{\myfont Verbale interno 2018-12-17}
\author{Dream Corp.}
\date{\myfont 2018-12-18}

\begin{document}
	\maketitle
	\begin{center}
		\includegraphics[width = 70mm]{logo.png}\newline
		\huge \version 
		\\G\&B
		
		\begin{table}[h!]
			\centering
			\begin{tabular}{r|l}
					\multicolumn{2}{c}{Informazioni sul documento}\\
			        \hline
        			Redazione & Matteo Bordin\\
        			Verifica & Gianluca Pegoraro\\
        			Responsabile & \pie\\
        			Uso & Interno\\
        			Destinatari & Dream Corp. \\
        			& Zucchetti SpA\\
        			& Prof. Tullio Vardanega\\
        			& Prof. Riccardo Cardin\\
			\end{tabular}
		\end{table}
		
\end{center}


\section{Riunione}
    \subsection{Informazioni generali}
    \begin{itemize}
        \item \textbf{Motivo della riunione}: È stata indetta questa riunione per scambiarsi informazioni sulle funzionalità di Grafana   \pedice.
        \item \textbf{Luogo e Data}: LabTA Torre Archimede, Lunedì 17 Dicembre 2018;
        \item \textbf{Orario}: 14:00-16:30;
        \item \textbf{Partecipanti}: Tutti i membri del gruppo fino alle 15:00. Successivamente hanno proseguito \daL, \daG, \mat.
        \newpage
        \section{Ordine del giorno}
        \begin{itemize}
        \item Condivisione delle funzionalità di Grafana \pedice .
        \end{itemize}
        
        \newpage
    \begin{enumerate}
    \section{Resoconto}
        \item \textbf{Funzionalità di Grafana \pedice:} Dopo uno studio autonomo, sono state condivise diverse informazioni sulle funzionalità di Grafana \pedice utili per la futura stesura dell'\textit{analisi dei requisiti}.
    \end{enumerate}

    \end{itemize}




\end{document}
