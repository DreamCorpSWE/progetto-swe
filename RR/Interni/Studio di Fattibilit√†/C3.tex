\section{Capitolato scelto C3}
	\subsection{Descrizione}
		Il progetto G\&B consiste nel sviluppare un plug-in di Grafana, un software per il monitoraggio di sistemi, al fine di non solo monitorare i livelli di risorse utilizzate dai suddetti sistemi, ma anche di dare direttive sugli interventi da eseguire.
	\subsection{Finalità}
		Gli obiettivi principali del progetto sono: implementare un plug-in di Grafana che legge una rete Bayesiana e che a intervalli regolari o in continuità applica il ricalcolo delle probabilità della rete e fornisce i nuovi dati alla dashboard attraverso grafi.
	\subsection{Tecnologie utilizzate}
		\begin{itemize}
			\item \textbf{Javascript} : linguaggio di scripting orientato agli oggetti ed agli eventi;
			\item \textbf{Reti Bayesiane} : modello grafico probabilistico che rappresenta un insieme di variabili stocastiche con le loro dipendenze condizionali (Consigliata la libreria jsbayes : \url{https://github.com/vangj/jsbayes});
			\item \textbf{Grafana} : software open source per il monitoraggio di sistemi \url{https://grafana.com/}.
		\end{itemize}
	\subsection{Conclusioni}
		L'argomento è stato trovato molto interessante, soprattutto per quanto riguarda l'uso delle reti baiesiane. Questo progetto ci permette di interfacciarci con nuove funzionalità e tecnologie a noi sconosciute e che potremmo utilizzare nel corso della nostra carriera.
		Il progetto si presenta molto impegnativo in termini di studio personale per quanto riguarda le funzionalità di Grafana.
		Inoltre la disponibilità dell'azienda ha fatto preferire questo capitolato agli altri.