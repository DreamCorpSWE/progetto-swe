\section{Capitolato C2}
	\subsection{Descrizione}
		Il progetto Colletta prevede lo sviluppo di un servizio web o un'applicazione mobile per la raccolta di dati indiretta. Questi verrano utilizzati per istruire un'intelligenza artificiale.
	\subsection{Finalità}
		L'obiettivo del progetto è quello di predisporre degli esercizi di grammatica (ad esempio analisi grammaticale) che potranno essere svolti da utenti e docenti registrati al servizio.
		In seguito allo svolgimento degli esercizi, i dati raccolti attraverso quest'ultimi saranno utilizzati per istruire un'intelligenza artificiale tramite tecniche di apprendimento automatico supervisionato, in modo che essa possa eseguirli.
	\subsection{Tecnologie utilizzate}
		Viene dato ampio margine alla scelta delle tecnologie, anche se ne vengono suggerite alcune :
		\begin{itemize}
			\item \textbf{Firebase} : piattaforma di sviluppo mobile e web, consigliato per la raccolta dati;
			\item \textbf{Hunpos} : software opensource per il pos-tagging;
			\item \textbf{FreeLing} : altro software per il pos-tagging.
		\end{itemize} 
	\subsection{Conclusioni}
		L'ampia libertà di scelta delle tecnologie è stata molto apprezzata. Anche il dover interfacciarsi con l'intelligenza artificiale risulta molto interessante, ma la richiesta in sé non lo è. L'assenza di profondità del capitolato ci ha portati a scartarlo.