\section{Capitolato C4}
	\subsection{Descrizione}
		Il progetto MegAlexa prevede di sviluppare un sistema Workflow(shortcut) per la creazione di skill custom.
	\subsection{Finalità}
		L'obiettivo del progetto è quello di creare una skill per Amazon Alexa in grado di avviare workflow creati dagli utenti attraverso un'interfaccia web o mobile.
		L'utente, all'interno della piattaforma, potrà inserire delle microfunzioni all'interno del workflow, eseguite tramite controllo vocale.
	\subsection{Tecnologie}
		\begin{itemize}
			\item \textbf{Amazon Web Services}
				\begin{itemize}
					\item \textbf{API Gateway} : servizio che semplifica le attività svolte su API su qualsiasi scala;
					\item \textbf{Lambda} : permette esecuzione di codice senza badare al server;
					\item \textbf{DynamoDB} : database non relazionale che fornisce prestazioni affidabili su qualsiasi scala.
				\end{itemize}
			\item \textbf{Node.js} : piattaforma Open source event-driven per l'esecuzione di codice JavaScript;
			\item \textbf{HTML, CSS3, Javascript} : linguaggi per il servizio web;
			\item \textbf{Bootstrap} : framework responsive;
			\item \textbf{Swift o Kotlin} : linguaggi di programmazione utilzzate per sviluppare rispettivamente applicazioni per iOS e Android.
		\end{itemize}
	\subsection{Conclusioni}
		Il progetto utilizza tecnologie attuali ed interessanti. Pensiamo che possano essere anche molto utili per il futuro. Tuttavia non è stato motivo di ispirazione per il gruppo.