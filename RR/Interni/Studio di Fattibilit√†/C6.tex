\section{Capitolato: C6}
	\subsection{Descrizione}
		Il capitolato "Soldino" prevede la realizzazione di una piattaforma wep per la gestione automatizzata delle VAT.
	\subsection{Finalità}
		L'obiettivo è la creazione di una serie di DApps che lavorano sulla EVM per la gestione automatica della VAT.
		Viene richiesto lo sviluppo di un'applicazione web o di un programma ad interfaccia grafica composta da una o più sezioni per le funzionalità del Governo, delle imprese, e dei cittadini. Inoltre vengono richieste delle sezioni dove le varie entità possono eseguire transazioni tra di loro.
	\subsection{Tecnologie utilizzate}
		\begin{itemize}
			\item \textbf{HTML, CSS, Javascript} : linguaggi di programmazione comunemente utilizzati per la creazione di contenuti web;
			\item \textbf{Blockchain} : registro digitale aperto e distribuito, in grado di memorizzare record di dati (solitamente, denominati "transazioni") in modo sicuro, verificabile e permanente;
			\item \textbf{Ethereum} : piattaforma creata per permettere agli utenti di creare in modo semplice applicazioni decentralizzate (applicazioni distribuite dove ogni parte è in grado di svolgere il proprio compito senza dipendere dalle altre) che usano la tecnologia blockchain;
			\item \textbf{Truffle} : ambiente di sviluppo, framework di test e pipeline di asset per blockchain che utilizzano Ethereum Virtual Machine (EVM);
			\item \textbf{React/Redux} : framework che aiuta a scrivere applicazioni Javascript che si comportino in modo consistente, funzionino in diversi ambienti e siano facili da testare.
		\end{itemize}
	\subsection{Conclusioni}
		La tecnologia Blockchain è risultata a tutti interessante e innovativa, in quanto costituisce il futuro delle transazioni, ma prevede uno studio dedicato a cui nessun membro del team è particolarmente interessato. Seguendo il trend degli altri capitolati, non andremo ad usare direttamente la tecnologia in questione ma dovremo solo capire come "interfacciarsi" con quest'ultima, che per quanto interessante non è fonte di ispirazione per il nostro lavoro.