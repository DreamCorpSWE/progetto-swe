\section{Introduzione}
	\subsection{Obiettivo del documento}
		Lo scopo del documento è di motivare la scelta del capitolato\pedice C3 \textit{"G\&B"}, e di presentare le considerazioni che ci hanno portato a scartare gli altri.
	\subsection{Glossario}
		Allo scopo di evitare ambiguità a lettori esterni al gruppo, si specifica che all'interno del documento verranno inseriti dei termini con un carattere G come pedice, questo significa che il significato inteso in quella situazione è stato inserito nel Glossario.
		\subsection{Riferimenti}
			\subsubsection{Normativi}
				\begin{enumerate}
					\item \textit{Norme di progetto}.
				\end{enumerate}
			\subsubsection{Informativi}
				\begin{enumerate}
					\item \textbf{Capitolato 1 : Butterfly} - Monitor per processi CI/CD;
						\newline
						\url{ https://www.math.unipd.it/~tullio/IS-1/2018/Progetto/C1.pdf}
					\item \textbf{Capitolato 2 : Colletta} - Piattaforma raccolta dati di analisi di test;
						\newline
						\url{https://www.math.unipd.it/~tullio/IS-1/2018/Progetto/C2.pdf}
					\item \textbf{Capitolato 3 : G\&B} - Monitoraggio intelligente di processi DevOps\pedice;
						\newline
						\url{https://www.math.unipd.it/~tullio/IS-1/2018/Progetto/C3.pdf}
					\item \textbf{Capitolato 4 : MegAlexa} - Arrichitore di skill di Amazon Alexa;
						\newline
						\url{https://www.math.unipd.it/~tullio/IS-1/2018/Progetto/C4.pdf}
					\item \textbf{Capitolato 5 : P2PCS} - Piattaforma di peer-to-peer car sharing; 
						\newline 
						\url{https://www.math.unipd.it/~tullio/IS-1/2018/Progetto/C5.pdf}
					\item \textbf{Capitolato 6 : Soldino} - Piattaforma Ethereum per pagamenti IVA.
						\newline 
						\url{https://www.math.unipd.it/~tullio/IS-1/2018/Progetto/C6.pdf}
            \end{enumerate}
