\documentclass{article}
\usepackage{titling}
\usepackage{caption}
\usepackage{multirow}
\usepackage{tabularx}
\usepackage[T1]{fontenc}
\usepackage[utf8]{inputenc}
\usepackage[italian]{babel}
\usepackage [colorlinks=true,urlcolor=blue]{hyperref}
\newcommand{\subtitle}[1]{%
  \posttitle{%
    \par\end{center}
    \begin{center}\LARGE#1\end{center}
    \vskip0.5em}%
}

\title{\textbf{STUDIO DI FATTIBILITÀ}
\\{\Large Versione 1.0.0 in data }}
\author{(Dreamcorp - Progetto DevOps)}

\begin{document}

\maketitle



\begin{tabularx}{\textwidth}{|X|X|X|X|X|}
	\hline
	\textbf{Versione} & \textbf{Data} & \textbf{Descrizione} & \textbf{Autore} & \textbf{Ruolo}\\
	\hline
	0.0.3 & 5/12/2018 & Introduzione & Pietro Casotto & Responsabile \\
	\hline
	0.0.2 & 5/12/2018 & Analisi dei Rischi & Davide Ghiotto & Analista \\
	\hline
	0.0.1 & 4/12/2018 & Creazione struttura del documento & Davide Ghiotto & Analista  \\
	\hline
\end{tabularx}
\newpage

\section{Introduzione}
	\subsection{Obiettivo del documento}
		Lo scopo del documento è di motivare la scelta del capitolato C3 \textit{"G\&B"} e di presentare le considerazioni che ci hanno portato a scartare gli altri.
	\subsection{Glossario}
		Al fine di evitare ambiguità è stato redatto un documento chiamato Glossario contente tutti i termini che a seconda del contesto possono necessitare di un ulteriore spiegazione per chiarificarne il significato. ((((((  *****   DA CONCORDARE METODO DI SEGNALAZIONE AL GLOSSARIO ******   ))))) 
	\subsection{Riferimenti}
		\subsubsection{Normativi}
			\begin{enumerate}
				\item \textit{Norme di progetto}
			\end{enumerate}
		\subsubsection{Informativi}
			\begin{enumerate}
				\item \textbf{Capitolato 1 : Butterfly} - monitor per processi CI/CD \newline 	\url{ https://www.math.unipd.it/~tullio/IS-1/2018/Progetto/C1.pdf}
				\item \textbf{Capitolato 2 : Colletta} - piattaforma raccolta dati di analisi di test \newline \url{https://www.math.unipd.it/~tullio/IS-1/2018/Progetto/C2.pdf}
				\item \textbf{Capitolato 3 : G\&B} - monitoraggio intelligente di processi DevOps. \newline \url{https://www.math.unipd.it/~tullio/IS-1/2018/Progetto/C3.pdf}
				\item \textbf{Capitolato 4 : MegAlexa} - Arrichitore di skill di Amazon Alexa. \newline \url{https://www.math.unipd.it/~tullio/IS-1/2018/Progetto/C4.pdf}
				\item \textbf{Capitolato 5 : P2PCS} - Piattaforma di peer-to-peer car sharing \newline \url{https://www.math.unipd.it/~tullio/IS-1/2018/Progetto/C5.pdf}
			    \item \textbf{Capitolato 6 : Soldino} - Piattaforma Ethereum per pagamenti IVA \newline \url{https://www.math.unipd.it/~tullio/IS-1/2018/Progetto/C6.pdf}
			\end{enumerate}
			
\newpage
    \section{Capitolato scelto: C3}
        \subsection{Descrizione}
        Il progetto G\&B consiste nel sviluppare un plug-in di Grafana, un software per il monitoraggio di sistemi, al fine di non solo monitorare i livelli di risorse utilizzate dai suddetti sistemi, ma anche di dare direttive sugli interventi da eseguire.
        \subsection{Finalità}
        	Gli obiettivi principali del progetto sono: Implementare un plug-in di Grafana che legge una rete Bayesiana e che a intervalli regolari o in continuità applica il ricalcolo delle probabilità della rete e fornisce i nuovi dati alla dashboard attraverso grafi.
        \subsection{Tecnologie utilizzate}
            \begin{itemize}
                \item \textbf{Javascript} : linguaggio di scripting orientato agli oggetti ed agli eventi.
                \item \textbf{Reti Bayesiane} : modello grafico probabilistico che rappresenta un insieme di variabili stocastiche con le loro dipendenze condizionali (Consigliata la libreria jsbayes : \url{https://github.com/vangj/jsbayes})
                \item \textbf{Grafana} : software open source per il monitoraggio di sistemi \url{https://grafana.com/}
            \end{itemize}
	\subsection{Conclusioni}
		L'argomento è stato trovato molto interessante, soprattutto per quanto riguarda l'uso delle reti baiesiane. Questo progetto ci permette di interfacciarci con nuove funzionalità e tecnologie a noi sconosciute e con cui potremmo interfacciarci nel corso della nostra carriera.
		Il progetto si presenta molto impegnativo in termini di studio personale per per quanto riguarda le funzionalità di Grafana.
		Inoltre la disponibilità dell'azienda ha fatto preferire questo capitolato agli altri.
\newpage

	\section{Captiolato C1}
		\subsection{Descrizione}
			Il progetto Butterfly prevede lo sviluppo di applicativi per la raccolta di messaggi provenienti da diversi servizi, il raggruppamento di essi all'interno di topic, ed infine l'invio ai sistemi di messaggistica degli utenti desiderati.
		\subsection{Finalità}
			Butterfly si articola in quattro diversi tipi di applicativi :
			\begin{itemize}
				\item \textbf{Producers} : componenti che hanno il compito di recuperare le segnalazioni e pubblicarle sotto forma di messaggi all'interno dei topic adeguati, per il progetto viene richiesto lo sviluppo di almeno due componenti.
				\item \textbf{Broker} : strumento per gestire i topic, implementato attraverso un DockerFile.
				\item \textbf{Consumers} : componenti che si abboneranno ai topic adeguati e invieranno i messaggi ai destinatari finali, per il progetto vengono richiesti almeno due di questi componenti.
				\item \textbf{Gestore Personale} : un componente custom specifico che attraverso il sitema di Consumer/Producer reindirizza i messaggi alla persona più appropriata secondo i canali di comunicazione desiderati.
			\end{itemize}
		\subsection{Tecnologie utilizzate}
			\begin{itemize}
				\item \textbf{Linguaggi consigliati}
					\begin{itemize}
						\item \textbf{Java} : linguaggio di programmazione ad alto livello orientato agli oggeti, a tipizzazione statica.
						\item \textbf{Python} : liguaggio ad alto livello, orientato agli oggetti, adatto a scripting, applicazioni distribuite e system testing.
						\item \textbf{Node.js} : piattaforma Open source event-driven per l'esecuzione di codice JavaScript Server-side.
				\end{itemize}	
			\item \textbf{Apache Kafka} : piattaforma open source di stream processing.
			\item \textbf{Docker} : progetto open-source che automatizza il deployment di applicazioni all'interno di contenitori software.
			\end{itemize}
		\subsection{Conclusioni}
			Questo progetto è stimolante perchè ci introduce all'apprendimento di tecnologie non ancora viste che sono molto utilizzate in ambito lavorativo che riteniamo fondamentali da avere in un curriculum. 
			D'altro canto non conosciamo la complessità degli argomenti da studiare quindi siamo un po scettici.
		
\newpage
		\section{Capitolato C2}
			\subsection{Descrizione}
				Il progetto Colletta prevede lo sviluppo di un servizio web o un'applicazione mobile per la raccolta di dati indiretta. Questi verrano utilizzati per istruire un'intelligenza artificiale.
			\subsection{Finalità}
				L'obiettivo del progetto è quello di predisporre degli esercizi di grammatica (ad esempio analisi grammaticale) che potranno essere svolti da utenti e docenti registrati al servizio.
				In seguito allo svolgimento degli esercizi, i dati raccolti attraverso quest'ultimi saranno utilizzati per istruire un'intelligenza artificiale, tramite tecniche di apprendimento automatico supervisionato, in modo che essa possa eseguirli.
			\subsection{Tecnologie utilizzate}
				Viene dato ampio margine alla scelta delle tecnologie, anche se ne vengono suggerite alcune :
				\begin{itemize}
					\item \textbf{Firebase} : piattaforma di sviluppo mobile e web, consigliato per la raccolta dati.
					\item \textbf{Hunpos} : software opensource per il pos-tagging.
					\item \textbf{FreeLing} : alto software per il pos-tagging.
				\end{itemize} 
			\subsection{Conclusioni}
				L'ampia libertà di scelta delle tecnologie è stata molto apprezzata. Anche il dover interfacciarsi con l'intelligenza artificiale risulta molto interessante, ma la richiesta in sè non lo è. L'assenza di profondità del capitolato ci ha portati a scartarlo.
\newpage
	\section{Capitolato C4}
		\subsection{Descrizione}
			Il progetto MegAlexa prevede di sviluppare un sistema Workflow(shortcut) per skill custom.
		\subsection{Finalità}
			L'obiettivo del progetto è quello di creare una skill per Amazon Alexa in grado di avviare workflow creati dagli utenti attraverso un'interfaccia web o mobile.
			L'utente, all'interno della piattaforma, potrà inserire delle microfunzioni all'interno del workflow, eseguito tramite controllo vocale.
		\subsection{Tecnologie}
			\begin{itemize}
				\item \textbf{Amazon Web Services}
					\begin{itemize}
						\item \textbf{API Gateway} : servizio che semplifica le attività svolte su API su qualsiasi scala.
						\item \textbf{Lambda} : permette esecuzione di codice senza badare al server.
						\item \textbf{DynamoDB} : database non relazionale che fornisce prestazioni affidabili su qualsiasi scala.
					\end{itemize}
				\item \textbf{Node.js} : piattaforma Open source event-driven per l'esecuzione di codice JavaScript.
				\item \textbf{HTML, CSS3, Javascript} : linguaggi per il servizio web.
				\item \textbf{Bootstrap} : framework responsive.
				\item \textbf{Swift o Kotlin} : linguaggi di programmazione utilzzate per sviluppare rispettivamente applicazioni per iOS e Android.
			\end{itemize}
		\subsection{Conclusioni}
			Il progetto utilizza tecnologie attuali ed interessanti. Pensiamo che possano essere anche molto utili per il futuro.
			
\newpage
\section{Capitolato: C5}
\subsection{Descrizione}
Il capitolato "P2PCS" prevede l'arricchimento delle funzionalita' dell'applicazione che gestisce un car sharing condominiale
\subsection{Finalità}
avere alcune delle funzionalita' del car sharing in funzione all'interno dell'applicazione
\subsection{Tecnologie utilizzate}
\begin{itemize}
	\item \textbf{NodeJS} : linguaggi di scripting comunemente utilizzati per la creazione di contenuti web.
	\item \textbf{GoogleCloud} : registro digitale aperto e distribuito, in grado di memorizzare record di dati (solitamente, denominati "transazioni") in modo sicuro, verificabile e permanente
\end{itemize}
\subsection{Conclusioni}
Il progetto è troppo vago e non si capisce esattamente cosa ci è richiesto ne le tecnologie da utilizzare.
Non condividiamo l'idea di base dell'applicazione. Sicuramente non sarà il progetto scelto
\newpage

\newpage
\section{Capitolato: C6}
\subsection{Descrizione}
Il capitolato "Soldino" prevede la realizzazione di una piattaforma wepb er la gestione automatizzata delle VAT
\subsection{Finalità}
la creazione di una serie di DApps che lavorano sulla EVM per la gestione automaticha della VAT
\subsection{Tecnologie utilizzate}
\begin{itemize}
	\item \textbf{HTML, CSS, Javascript} : linguaggi di programmazione comunemente utilizzati per la creazione di contenuti web.
	\item \textbf{Blockchain} : registro digitale aperto e distribuito, in grado di memorizzare record di dati (solitamente, denominati "transazioni") in modo sicuro, verificabile e permanente
	\item \textbf{Ethereum} : piattaforma creata per permettere agli utenti di creare in modo semplice applicazioni decentralizzate (applicazioni distribuite dove ogni parte è in grado di svolgere il prorpio compito senza dipendere dalle altre) che usano la tecnologia blockchain.
	\item \textbf{Truffle} : ambiente di sviluppo, framework di test e pipeline di asset per blockchain che utilizzano Ethereum Virtual Machine (EVM).\newline \url{https://truffleframework.com/docs/truffle/overview}
	\item \textbf{React/Redux} : framework che aiuta a scrivere applicazioni Javascript che si comportino in modo consistente, funzionino in diversi ambienti e siano facili da testare.\newline \url{https://redux.js.org/}
\end{itemize}
\subsection{Conclusioni}
La tecnologia Blockchain è risultata a tutti interessante e innovativa in quanto costituisce il futuro delle transazioni, ma prevede uno studio dedicato a cui nessun membro del team è particolarmente interessato. Seguendo il trend degli altri capitolati, non andremo ad usare direttamente la tecnologia in questione ma dovremo solo capire come "interfacciarsi" con quest'ultima, che per quanto interessante non è fonte di ispirazione per il nostro lavoro.
\newpage			
									
		
\end{document}