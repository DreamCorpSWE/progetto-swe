\section{Introduzione}
		\subsection{Scopo del documento}			
Questo documento si pone come obiettivo quello di effettuare un' analisi dei requisiti per la progettazione e lo sviluppo del capitolato \pedice "G\&B: monitoraggio intelligente di processi DevOps \pedice" (Capitolato C3) proposto dall'azienda Zucchetti \pedice.


		\subsection{Obiettivo del prodotto}

Il sistema da realizzare sarà  un plug-in \pedice di Grafana \pedice, scritto in linguaggio JavaScript\pedice, che leggerà da un file json\pedice contenente la definizione della rete Bayesiana \pedice e quindi permetterà di associare ad alcuni nodi della rete data un flusso di monitoraggio \pedice.
Ad intervalli predefiniti, verranno eseguiti i calcoli previsti dalla rete Bayesiana, modificando le probabilità dei nodi derivati in base ai dati rilevati dal campo.


		\subsection{Note esplicative}

Allo scopo di evitare ambiguità a lettori esterni al gruppo, si specifica che all'interno del documento verranno inseriti dei termini con un carattere \pedice come pedice, questo significa che il significato inteso in quella situazione è stato inserito nel Glossario.


		\subsection{Riferimenti }

		\subsubsection{Riferimenti Normativi}
		\begin{itemize}
			\item Norme di progetto v 1.0.0
			\item Capitolato C3: G\&B: monitoraggio intelligente di processi DevOps.\newline
			\url{https://www.math.unipd.it/~tullio/IS-1/2018/Progetto/C3.pdf}
			\item Grafana:\newline
			\url{https://grafana.com/}
			\item Libreria Open Source per reti bayesiane:\newline
			\url{https://github.com/vangj/jsbayes}
			\item Slide lezioni utilizzate durante il corso di Ingegneria del Software:\newline
			\url{https://www.math.unipd.it/~tullio/IS-1/2018/}
		\end{itemize}

		\subsubsection{Riferimenti Informativi}
			\begin{itemize}
			\item Piano di qualifica v 1.0.0
			\item Slide del corso di "Ingegneria del Software"\newline
			\url{https://www.math.unipd.it/~tullio/IS-1/2018/}
		\end{itemize}
				

\newpage
