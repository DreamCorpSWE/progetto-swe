% da include in stile.tex per poter colorare le tabelle
% \usepackage[table]{xcolor}

\begin{table}[h!] % h! serve per posizionarla relativamente
            \centering
            \renewcommand{\arraystretch}{2} % dimensione verticale delle righe
            \rowcolors{2}{gray!25}{white} %colori alternati, grigio 25% e bianco 100%
            \begin{tabular}{|l|c|p{8cm}|c|} % p{dimensione desiderata}
                \rowcolor{orange!50} %colore intestazione
        		\hline
        		\textbf{Col1} & \textbf{Col2} & \textbf{Col3} & \textbf{Col4}\\
                \hline
                Value1 & Value2 & Value3 & Value4\\
        \end{tabular}
        \caption{Caption} %descrizone a fine tabella
        \label{tab:my_label}
\end{table}
