
\section{Tracciamento dei Requisiti}
Per facilitare la lettura e la visualizzazione verranno presentate delle tabelle indicizzate in modo specifico:
\begin{itemize}
    \item Tracciamento Priorità-Requisito
    \item Tracciamento Tipologia-Requisito
\end{itemize}

\newpage
\subsection{Tracciamento: Priorità-Requisito}
Di seguito viene riportata una tabella riassuntiva che descrive le priorità dei requisiti e i requisiti appartenenti a quella priorità per rendere di più facile lettura la distribuzione per priorità.
Per brevità non viene riportata la descrizione del requisito ma semplicemente il suo codice univoco in quando era già stato descritto in dettaglio nella sezioni precedenti.   

\begin{table}[!htbp] % h! serve per posizionarla relativamente
            \centering
            \renewcommand{\arraystretch}{2} % dimensione verticale delle righe
            \rowcolors{2}{gray!25}{white} %colori alternati, grigio 25% e bianco 100%
            \begin{tabular}{|c|p{2cm}|} % p{dimensione desiderata}
                \rowcolor{orange!50} %colore intestazione
        		\hline
        		\textbf{Priorità Requisito} & \textbf{Codice Requisiti} \\
                \hline
                Compulsory & RFC1, RFC1.1, RFC1.2, RFC2, RFC3.2, RFC4, RFC5, RFC9, RFC10, RFC10.1, RFC10.2, RFC10.3, RCV2, RCV3, RCV4, RCV5, RQC1, RQC2, RQC4\\
                \hline
                Optional & RFO3, RFO3.1, RFO6, RFCO7, RFO8, RVO1, RQO3\\
                \hline
        \end{tabular}
        \caption{Tracciamento Priorità-Requisito} %descrizone a fine tabella
\end{table}

\newpage
\subsection{Tracciamento: Tipologia-Requisito}
Di seguito viene riportata una tabella riassuntiva che descrive le tipologie di requisiti e i requisiti appartenenti a quella tipologia per rendere di più facile lettura la distribuzione per tipologia.
Per brevità non viene riportata la descrizione del requisito ma semplicemente il suo codice univoco in quando era già stato descritto in dettaglio nella sezioni precedenti.

\begin{table}[!htbp] % h! serve per posizionarla relativamente
            \centering
            \renewcommand{\arraystretch}{2} % dimensione verticale delle righe
            \rowcolors{2}{gray!25}{white} %colori alternati, grigio 25% e bianco 100%
            \begin{tabular}{|c|p{2cm}|} % p{dimensione desiderata}
                \rowcolor{orange!50} %colore intestazione
        		\hline
        		\textbf{Tipologia Requisito} & \textbf{Codice Requisiti} \\
                \hline
                Funzionale & RFC1, RFC1.1, RFC1.2, RFC2, RFO3, RFO3.1, RFC3.2, RFC4, RFC5, RFO6, RFCO7, RFO8, RFC9, RFC10, RFC10.1, RFC10.2, RFC10.3 \\
                \hline
                Vincolo & RVO1, RCV2, RCV3, RCV4, RCV5\\
                \hline
                Qualità & RQC1, RQC2, RQO3, RQC4\\
                \hline
        \end{tabular}
        \caption{Tracciamento Tipologia-Requisito} %descrizone a fine tabella
\end{table}

\newpage
\subsection{Riepilogo}
Di seguito viene riportata una tabella riassuntiva a doppia entrata con la tipologia di requisito per colonna e la priorità per riga. Viene riportato come valore il numero per ogni tipologia e priorità al fine di avere un quadro generale ed immediato della distribuzione degli stessi.

\begin{table}[!htbp] % h! serve per posizionarla relativamente
            \centering
            \renewcommand{\arraystretch}{2} % dimensione verticale delle righe
            \rowcolors{2}{gray!25}{white} %colori alternati, grigio 25% e bianco 100%
            \begin{tabular}{|c|c|c|c|c|c|} % p{dimensione desiderata}
                \rowcolor{orange!50} %colore intestazione
        		\hline & \textbf{Funzionali} & \textbf{Vincolo} & \textbf{Qualità} & \textbf{Prestazionali} & \textbf{Tot. Priorità}\\
                \hline
                \textbf{Compulsory} & 12 & 4 & 3 & 0 & 19\\
                \hline
                \textbf{Optional} & 5 & 1 & 1 & 0 & 7\\
                \hline
                \textbf{Tot. Tipologia} & 17 & 5 & 4 & 0 & 26\\
                \hline
        \end{tabular}
        \caption{Riepilogo Distribuzione Requisiti} %descrizone a fine tabella
\end{table}


%\subsection{SPOSTALA IN PIANO DI QUALIFICA}

%Verificabilità dei requisiti
    %\begin{itemize}
        %\item I requisiti vanno scritti in modo che possano essere
    %oggettivamente verificati nel prodotto finale
    %    \item Occorre quantificare il tasso degli errori.
        %\item “Gli operatori esperti dovrebbero poter controllare le
    %funzioni di sistema dopo due ore di formazione. Dopo tale formazione, il numero %medio di errori degli operatori esperti non dovrebbe superare i due al giorno”.
    %\end{itemize}
    
    %Validazione dei requisiti:
    %\begin{itemize}
        %\item La dimostrazione che i requisiti definiscono il
    %sistema davvero voluto dal cliente.
        %\item La tecnica più importante di validazione dei
    %requisiti è la prototipazione.
    %\end{itemize}
