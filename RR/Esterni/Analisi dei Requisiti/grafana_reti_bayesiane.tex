\section{Grafana e reti Bayesiane}
		\subsection{Grafana}			
Grafana è uno strumento di monitoraggio \pedice dati open-source, tramite delle dashboard è possibile visualizzare e tenere sotto controllo i risultati delle query \pedice loro associate e lanciare un allarme in caso di superamento di valori soglia prestabiliti dall'utente. E' molto utilizzato dalle aziende per via delle sue molteplici funzionalità  espandibili tramite vari plug-in sviluppati dagli utenti.


		\subsection{Reti Bayesiane}
Una Rete Bayesiana è un grafo, ossia un'insieme di nodi e archi. I nodi
indicano le variabili di un problema in gioco, mentre le frecce indicano i
rapporti di causalità tra le variabili e costituiscono un potente mezzo per
modellare un problema ed esprimere i rapporti tra le grandezze in gioco.
Ad una rete bayesiana possono essere fornite delle "evidenze", ossia valori noti di variabili del problema. 
La rete calcola come la conoscenza di queste variabili modifica la probabilità  delle altre variabili.
Anche se non si ha idea di quali siano i rapporti di mutua dipendenza tra
variabili, gli algoritmi di "structure learning" riescono a ricostruire la corretta struttura della rete, sempre che si abbia a disposizione un adeguata base dati.
Sostanzialmente le Reti Bayesiane possono essere utilizzate come potenti mezzi
di "machine learning \pedice". Esse riescono ad individuare i fattori decisivi che
determinano i valori di una variabile, individuare la categoria cui appartengono determinate osservazioni e prevedere comportamenti futuri in base all'esperienza di quelli passati.


\newpage
