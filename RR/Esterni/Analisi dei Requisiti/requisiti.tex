\section{Requisiti}
    Tracciare i requisiti significa che requisiti correlati vengono raggruppati e collegati per facilitare la lettura grazie a tabelle e indicizizzazioni.
    Assegneremo per tanto ad ogni requisito un identificatore univoco, composto da una serie di regole dettate dalla caratteristica stessa del requisiti.
    
    Verificabilità dei requisiti
    • I requisiti vanno scritti in modo che possano essere
    oggettivamente verificati nel prodotto finale
    • Occorre quantificare il tasso degli errori.
    – “Gli operatori esperti dovrebbero poter controllare le
    funzioni di sistema dopo due ore di formazione.
    Dopo tale formazione, il numero medio di errori
    degli operatori esperti non dovrebbe superare i due
    al giorno”
    
    Validazione dei requisiti
    • La dimostrazione che i requisiti definiscono il
    sistema davvero voluto dal cliente
    • La tecnica più importante di validazione dei
    requisiti è la prototipazione


    Analizzeremo nella seguente sezione i vari requisiti suddivisi in categorie:
    \begin{itemize}
        \item Funzionali
        \item Non funzionali:
            \begin{itemize}
                \item Requisiti di vincolo
                \item Requisiti di qualità
                \item Requisiti prestazionali
            \end{itemize}
    \end{itemize}
		\subsection{Requisiti Funzionali}			
        Descrivono in dettaglio i servizi che verranno forniti dal sistema agli attori.
        
        \subsection{Requisiti Non Funzionali}
        Descrivono i vincoli sul sistema e sul suo processo di sviluppo
        \subsubsection{Requisiti di vincolo}
        
        \subsubsection{Requisiti di qualità}
        \subsubsection{Requisiti prestazionali}
