
\section{Requisiti}
    Tracciare i requisiti\pedice significa che requisiti correlati vengono raggruppati e collegati per facilitare la lettura grazie a tabelle e indicizizzazioni.
    Assegneremo per tanto ad ogni requisito un identificatore univoco, composto da una serie di regole dettate dalla caratteristica stessa del requisiti.
    
    \subsubsection{Categorie}
    Analizzeremo nella seguente sezione i vari requisiti suddivisi in categorie:
    \begin{itemize}
        \item Funzionali;
        \item Non funzionali:
            \begin{itemize}
                \item Requisiti di vincolo;
                \item Requisiti di qualità;
                \item Requisiti prestazionali.
            \end{itemize}
    \end{itemize}
    
    \subsubsection{Tracciabilita'}
    Il codice con cui ogni requisito viene univocamente indicizzato e' formato da una regola di composizione definita in questo modo:
    \newline
    
    \begin{center}
        \textbf{R+(F|Q|V|P)+(C|O)+(X(.Y)*)}    
    \end{center}
    
    
    \begin{itemize}
        \item R : Requisito;
        \item F|Q|V|P:
            \begin{itemize}
                \item F: Funzionale;
                \item Q: Qualita';
                \item V: Vincolo;
                \item P: Prestazionale.
            \end{itemize}
        \item C|O:
            \begin{itemize}
                \item C: Compulsory (Obbligatorio);
                \item O: Optional (Opzionale).
            \end{itemize}
        \item X.Y: sono numeri naturali concatenati con un punto per descrivere un sotto requisito.
    \end{itemize}
    
    
    \subsubsection{Fonti}
    Le varie fonti sono:
        \begin{itemize}
            \item Interno: il requisito proviene da una decisone del gruppo "DreamCorp":
                \begin{itemize}
                    \item Verbale: il requisito proviene da una decisione presa durante un incontro ed e' riportata in un verbale.
                \end{itemize}
            \item Capitolato: il requisito proviene dalle richieste del capitolato;
            \item Esterno: il requisito proviene da un incontro con la proponente.
        \end{itemize}
    
		\subsection{Requisiti Funzionali}			
        Descrivono in dettaglio i servizi che verranno forniti dal sistema agli attori.
        
        \begin{table}[]
            \centering
            \renewcommand{\arraystretch}{1.5} % dimensione verticale delle righe
            \rowcolors{2}{gray!25}{white}
            \begin{tabular}{|l|c|p{8cm}|c|}
                \rowcolor{orange!50}
        		\hline
        		\textbf{Codice} & \textbf{Priorità} & \textbf{Descrizione} & \textbf{Fonte}\\
                \hline
                RFC1 & Compulsory & Inserimento della definizione della rete bayesiana & Capitolato\\
                \hline
                RFC1.1 & Compulsory & Inserimento della definizione della rete bayesiana sotto forma di file .json & Interno\\
                \hline
                RFC1.2 & Compulsory & Inserimento definizione rete bayesiana sotto forma di codice json & Interno\\
                \hline
                RFC2 & Compulsory & Associazione di un nodo della rete ad un flusso dati di Grafana & Capitolato\\
                \hline
                RFO3 & Optional & Gestione della duplice associazione di due nodi della rete ad uno stesso flusso dati di Grafana & Interno\\
                \hline
                RFO3.1 & Optional & Annullamento dell'operazione di associazione multipla & Interno\\
                \hline
                RFC3.2 & Compulsory & Rimozione nodo precedentemente associato e associazione nuovo nodo & Interno\\
                \hline
                RFC4 & Compulsory & Visualizzazione del messaggio di conferma di annullamento dell'operazione & Interno\\
                \hline
                RFC5 & Compulsory & Rimozione di un nodo della rete associato ad un flusso dati di Grafana & Interno\\
                \hline
                RFO6 & Optional & Definizione di un alert personalizzato per un flusso dati non monitorato & Capitolato\\
                \hline
                RFO7 & Optional & Modifica di un alert & Interno\\
                \hline
                RFO8 & Optional & Rimozione di un alert & Interno\\
                \hline
                RFC9 & Compulsory & Lancio di un alert & Capitolato\\
                \hline
                RFC10 & Compulsory & Visualizzazione errore & Interno\\
                \hline
                RFC10.1 & Compulsory & Visualizzazione errore di interpretazione della rete bayesiana & Interno\\
                \hline
                RFC10.2 & Compulsory & Visualizzazione errore di invio del messaggio di alert & Interno\\
                \hline
                RFC10.3 & Compulsory & Visualizzazione errore di configurazione dell’alert & Interno\\
                \hline
        \end{tabular}
            \caption{Requisiti Funzionali}
            \label{tab:my_label}
        \end{table}
        
        \newpage

        
        \subsection{Requisiti Non Funzionali}
        Descrivono i vincoli sul sistema e sul suo processo di sviluppo
        \subsubsection{Requisiti di vincolo}
        
        \begin{table}[h!]
            \centering
            \renewcommand{\arraystretch}{1.5} % dimensione verticale delle righe
            \rowcolors{2}{gray!25}{white} %colori alternati, grigio 25% e bianco 100%
            \begin{tabular}{|l|c|p{8cm}|c|} % p{dimensione desiderata}
                \rowcolor{orange!50} %colore intestazione
        		\hline
        		\textbf{Codice} & \textbf{Priorita'} & \textbf{Descrizione} & \textbf{Fonte}\\
                \hline
                RVO1 &  Optional & Il codice sorgente deve essere pubblicato su un repository di Github\pedice & Capitolato/Interno\\
                \hline
                RVC2 & Compulsory & Il plugin deve essere sviluppato in linguaggio Javascript & Capitolato\\
                \hline
                RVC3 & Compulsory & Il plugin deve essere open-source\pedice & Capitolato\\
                \hline
                RVC4 & Compulsory & Il plugin deve essere pubblicato su Grafana.com & Interno\\
                \hline
                RVC5 & Compulsory & La definizione della rete bayesiana deve essere in formato .json & Capitolato\\
                \hline
            \end{tabular}
            \caption{Requisiti di vincolo} %descrizone a fine tabella
            \label{tab:my_label}
        \end{table}
        
        \subsubsection{Requisiti di qualità}
        
        \begin{table}[h!]
            \centering
            \renewcommand{\arraystretch}{1.5} % dimensione verticale delle righe
            \rowcolors{2}{gray!25}{white} %colori alternati, grigio 25% e bianco 100%
            \begin{tabular}{|l|c|p{8cm}|c|} % p{dimensione desiderata}
                \rowcolor{orange!50} %colore intestazione
        		\hline
        		\textbf{Codice} & \textbf{Priorita'} & \textbf{Descrizione} & \textbf{Fonte}\\
                \hline
                RQC1 &  Compulsory & La  progettazione e il codice devono seguire le norme e le metriche riportate nel documento allegato "Piano di qualifica v 1.0.0" & Verbale 1.2.3\\
                \hline
                RQC2 &  Compulsory & Devono essere rispettate le norme definite nel documento "Norme di progetto v1.0.0" & Verbale 1.2.3\\
                \hline
                RQO3 &  Optional & Il codice JavaScript\pedice deve attenersi allo stile Javascript airbnb\pedice & Verbale 1.2.3\\
                \hline
                RQC4 &  Compulsory & Dovrà essere fornito un manuale guida in lingua italiana sull'utilizzo del plugin & Capitolato\\
                \hline
            \end{tabular}
            \caption{Requisiti di qualita'} %descrizone a fine tabella
            \label{tab:my_label}
        \end{table}
        
%- SUCCESSIVI REQUISITI DI QUALITA'
%-L'applicativo deve essere sviluppato usando il framework (cosa useremo per scrivere il codice?)
%-Dovrà essere fornito un file per il tracciamento di tutti i bug presenti nel sistema
%-Dovranno essere consegnati i risultati dei test di unità (li useremo?)
%-Utilizzare il processo di continuous integration durante lo sviluppo del progetto (la useremo?)
        
        \subsubsection{Requisiti prestazionali}
        Le prestazioni del nostro sistema non dipendono dal plugin ma trattandosi di reti bayesiane le prestazioni dipendono dalla velocità di calcolo della libreria jsbayes.js e dalla complessità della rete che si sta osservando. Detto ciò non ha abbiamo osservato requisiti prestazionali.
