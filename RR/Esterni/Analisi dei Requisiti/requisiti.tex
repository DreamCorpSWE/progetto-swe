\section{Requisiti}
    Tracciare i requisiti significa che requisiti correlati vengono raggruppati e collegati per facilitare la lettura grazie a tabelle e indicizizzazioni.
    Assegneremo per tanto ad ogni requisito un identificatore univoco, composto da una serie di regole dettate dalla caratteristica stessa del requisiti.
    
    Verificabilità dei requisiti
    \begin{itemize}
    	\item I requisiti vanno scritti in modo che possano essere
    oggettivamente verificati nel prodotto finale
        \item Occorre quantificare il tasso degli errori.
        \item “Gli operatori esperti dovrebbero poter controllare le
    funzioni di sistema dopo due ore di formazione. Dopo tale formazione, il numero medio di errori degli operatori esperti non dovrebbe superare i due al giorno”.
    \end{itemize}
    
    Validazione dei requisiti:
    \begin{itemize}
        \item La dimostrazione che i requisiti definiscono il
    sistema davvero voluto dal cliente.
        \item La tecnica più importante di validazione dei
    requisiti è la prototipazione.
    \end{itemize}

    Analizzeremo nella seguente sezione i vari requisiti suddivisi in categorie:
    \begin{itemize}
        \item Funzionali
        \item Non funzionali:
            \begin{itemize}
                \item Requisiti di vincolo
                \item Requisiti di qualità
                \item Requisiti prestazionali
            \end{itemize}
    \end{itemize}
		\subsection{Requisiti Funzionali}			
        Descrivono in dettaglio i servizi che verranno forniti dal sistema agli attori.
        \\
        \\
        \begin{tabular}{|l|c|l|l|}
		\hline
		Codice & Priorità & Descrizione & Fonte\\
        \hline
        RC1 & Compulsory & Inserimento della definizione della rete bayesiana & Interno\\
        \hline
        RC1.1 & Compulsory & Inserimento della definizione della rete bayesiana sotto forma di file .json & Interno\\
        \hline
        RC1.2 & Compulsory & Inserimento definizione rete bayesiana sotto forma di codice json & Interno\\
        \hline
        RC2 & Compulsory & Associazione di un nodo della rete ad un flusso dati di Grafana & Interno\\
        \hline
        RO3 & Optional & Gestione della duplice associazione di due nodi della rete ad uno stesso flusso dati di Grafana\\
        \hline
        RO3.1 & Optional & Annullamento dell'operazione di associazione multipla & Interno\\
        \hline
        RC3.2 & Compulsory & Rimozione nodo precedentemente associato e associazione nuovo nodo & Interno\\
        \hline
        RC4 & Compulsory & Visualizzazione del messaggio di conferma di annullamento dell'operazione & Interno\\
        \hline
        RC5 & Compulsory & Rimozione di un nodo della rete associato ad un flusso dati di Grafana & Interno\\
        \hline
        RO6 & Optional & Definizione di un alert personalizzato per un flusso dati non monitorato & Interno\\
        \hline
        RO7 & Optional & Modifica di un alert & Interno\\
        \hline
        RO8 & Optional & Rimozione di un alert & Interno\\
        \hline
        RC9 & Compulsory & Lancio di un alert & Interno\\
        \hline
        RC10 & Compulsory & Visualizzazione errore & Interno\\
        \hline
        RC10.1 & Compulsory & visualizzazione errore di interpretazione della rete bayesiana & Interno\\
        \hline
        RC10.2 & Compulsory & Visualizzazione errore di invio del messaggio di alert & Interno\\
        \hline
        RC10.3 & Compulsory & Visualizzazione errore di configurazione dell’alert & Interno\\
        \hline
        \end{tabular}

        
        \subsection{Requisiti Non Funzionali}
        Descrivono i vincoli sul sistema e sul suo processo di sviluppo
        \subsubsection{Requisiti di vincolo}
        
        \subsubsection{Requisiti di qualità}
        \subsubsection{Requisiti prestazionali}
        Le prestazioni del nostro sistema non dipendono dal plugin ma trattandosi di reti bayesiane le prestazioni dipendono dalla velocità di calcolo della libreria jsbayes.js e dalla complessità della rete che si sta osservando. Detto ciò non ha abbiamo osservato requisiti prestazionali.

