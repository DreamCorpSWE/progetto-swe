\section{Preventivo a finire}
Il preventivo a finire viene presentato come tabella suddivisa in periodi. Per il calcolo del consuntivo di un periodo ancora da avviare viene utilizzato il valore a preventivo.
\begin{table}[h!] % h! serve per posizionarla relativamente
            \centering
            \renewcommand{\arraystretch}{2} % dimensione verticale delle righe
            \rowcolors{2}{gray!25}{white} %colori alternati, grigio 25% e bianco 100%
            \begin{tabular}{|l|c|c|} % p{dimensione desiderata}
                \rowcolor{orange!50} %colore intestazione
        		\hline
        		\textbf{Periodo} & \textbf{Preventivo in \euro} & \textbf{Consuntivo in \euro}\\
                \hline
                Attività preliminari di avvio ed analisi dei requisiti & 2765 & 2770\\
                \hline
                Progettazione architetturale & 3416 & -\\
                \hline
                Progettazione di dettaglio e codifica & 4518 & -\\
                \hline
                Validazione e collaudo & 2565 & -\\
                \hline
                \textbf{Totale} & 13264 \euro & 13269 \euro\\
                \hline
        \end{tabular}
        \caption{Preventivo a finire} %descrizone a fine tabella
        \label{tab:my_label}
\end{table}
