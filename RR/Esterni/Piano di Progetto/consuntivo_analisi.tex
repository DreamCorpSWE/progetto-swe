\subsection{Consuntivo del Periodo di avvio e analisi dei requisiti}

\subsubsection{Consuntivo Orario}
Nel consuntivo orario del periodo di avvio e analisi vengono riportate, per ogni persona, le ore preventivate nella pianificazione (colore nero) e di fianco un numero che può essere di due colori:
\begin{itemize}
    \item \red{rosso}: un numero positivo rappresentante le ore richieste in più rispetto al preventivo;
    \item \green{verde}: un numero negativo rappresentante le ore richieste in meno rispetto al preventivo.
\end{itemize}
\begin{table}[!htbp]
			\centering
			\renewcommand{\arraystretch}{2} 
			\rowcolors{2}{gray!25}{white}
			\begin{tabular}{|l c c c c c c|c| }
				\rowcolor{orange!50}
				\hline
				\multicolumn{8}{|c|}{\textbf{Consuntivo orario del Periodo di avvio e analisi}}\\
				\hline
				\textbf{Nominativo} & RES 	& AMM 	& ANA 	& PRO 	& PRG 	& VER 	& \textbf{Totale} \\
				\hline
				\mat 				& -		& 6		& 10 \red{+1} & -		& -		& 2		& 18 \red{+1}\\
				\hline
				\pie 				& 12 \red{+2}	& -		& 6 \green{-1}		& -		& - 	& -		& 18 \red{+1}\\
				\hline
				\mic  				& -		& -		& 6	\green{-1}	& -		& -		& 12 \red{+1}	& 18\\
				\hline
				\mar  				& -		& 12 \green{-2}	& 2		& -		& - 	& 4 \red{+2}	& 18\\
				\hline
				\daG  				& 6		& -		& 12 \green{-1}	& -		& - 	& -		& 18 \green{-1}\\
				\hline
				\daL 				& -		& 3 \green{-1}		& 11	& -		& -		& 4		& 18 \green{-1}\\
				\hline
				\gia 				& -		& 6	\red{+2}	& -		& -		& -		& 12 \green{-2} & 18\\
				\hline
			\end{tabular}
			\caption{Consuntivo orario del Periodo di avvio e analisi}
		\end{table}
\subsubsection{Consuntivo economico del Periodo di avvio e analisi}
Nel consuntivo economico del periodo di avvio e analisi vengono riportate, per ogni ruolo, le ore e i costi preventivati nella pianificazione (colore nero) e di fianco ad essi un numero che può essere di due colori:
\begin{itemize}
    \item \red{rosso}: un numero positivo rappresentante l'aumento rispetto al preventivo;
    \item \green{verde}: un numero negativo rappresentante la diminuzione rispetto al preventivo.
\end{itemize}
\begin{table}[!htbp]
			\centering
			\renewcommand{\arraystretch}{2} 
			\rowcolors{2}{gray!25}{white}
			\begin{tabular}{| c c c|}
				\rowcolor{orange!50}
				\hline
				\multicolumn{3}{|c|}{\textbf{Consuntivo economico del Periodo di avvio e analisi}}\\
				\hline
				\textbf{Ruolo} 			& Ore 	& Costo in \euro\\
				\hline
				\textbf{Responsabile}	& 18 \red{+2} 	& 540 \red{+60}\\
				\hline
				\textbf{Amministratore}	& 27 \green{-1}	& 540 \green{-20}\\
				\hline
				\textbf{Analista}		& 47 \green{-2}	& 1175 \green{-50}\\
				\hline
				\textbf{Progettista}	& 0 	& 0\\
				\hline
				\textbf{Programmatore}	& 0 	& 0\\
				\hline
				\textbf{Verificatore} 	& 34 \red{+1}	& 510 \red{+15} \\
				\hline
				\textbf{Totale} 		& 126	& 2765 \red{+5} \euro\\
				\hline 
			\end{tabular}
			\caption{Consuntivo economico del Periodo di avvio e analisi}
		\end{table}
\subsubsection{Conclusioni}
Le ore preventivate risultano uguali a quelle consuntivate, e allo stesso modo anche la variazione di costo è trascurabile. 
Tuttavia, come mostrato dai valori delle tabelle, abbiamo assegnato troppe ore ad alcuni ruoli rispetto ad altri. Nel consuntivo è risultato infatti un riequilibrio delle ore:
sono stati sottostimati i ruoli con meno ore assegnate e sovrastimati i ruoli con più ore assegnate. 
L'accaduto può essere spiegato da un pianificazione guidata dall'inesperienza: non siamo stati in grado di stimare il peso di alcuni ruoli.
Queste considerazioni dovranno guidare il seguente preventivo a finire.
