\section{Pianificazione}
	scopo (BB)
	analisi (avvio e analisi)
	-> elenco attività in ordine cronologico
	->divisione dei ruoli
	consolidamento dei requisiti (353)
	-> dettaglio delle varie fasi
	\newline
	\newline
	Con lo scopo di rispettare le scadenze decritte in questo documento è stata redatta la pianificazione del lavoro traendo ispirazione dalle attività previste nei processi di Fornitura e Sviluppo descritti nello standard ISO/IEC 12207:1995, si organizza la suddivisione del lavoro nelle seguenti macro-fasi:
	\begin{itemize}
		\item{\textbf{Attività preliminari di avvio ed analisi dei requisiti;}}
		\item{\textbf{Progettazione architetturale}}
		\item{\textbf{Progettazione di dettaglio e codifica}}
		\item{\textbf{Validazione e collaudo}}
	\end{itemize}

	Queste fasi vengono inserite in un diagramma di Gantt. Dove le milestone vengono assunte come principali scadenze di ogni fase. Verranno inoltre segnalati con colori diversi quelle scadenze che sono bloccanti oppure no ed  ogni attività è rappresentata tramite le sue sotto attività.
	
	\subsection{Attività preliminari di analisi}
	Il preiodo preliminare di analisi si svolge da ... a .... e partono dalla formazione del gruppo terminando poi alla prima milestone ovvero la consegna dei documenti. \newline
 	In ordine di esecuzione le attività principali sono le seguenti:
\begin{itemize}
	\item{\textbf{Norme di progetto:} questo è il primo documento poichè contiene tutte le norme atte a regolare internamente il gruppo DreamCorp che spaziano dall'identatura del codice alle regole per stendere i documenti stessi;}
	\item{\textbf{Studio di fattibilità:} redatta dagli Analisti questa è un' attività bloccante per l'inizio dell'analisi dei reuisiti. Consiste in un analisi dei pro ed i contro dei vari capitolati con il fine di una scelta ponderata del capitolato;}
	\item{\textbf{Analisi dei requisiti:} dopo aver scelto il capitolato è necessario approfondire i requisiti richiesti dalla proponente. Svolto ancora una volta dagli Analisti;}
	\item{\textbf{Piano di progetto:} questa attività è bloccante per la stesura della lettera di presentazione. Viene svolta dal Repsonsabile che ha lo scopo di analizare le attività necessarie e la loro scadenza al fine di garantire la  buona riuscita del progetto mentre L'amministratore analizza i rischi nei quali il gruppo DreamSwe può incombere durate il percorso. Inoltre in questa attività vengono stabilite le risorse dipsonibili per l'intero progetto;}
	\item{\textbf{Piano di qualifica:} in questa attività viene redatto il documento \textit{Piano di Qualifica} che ha lo scopo di individuare dei metodi per garnatire la qualià del prodotto;}
	\item{\textbf{Glossario:} questa attivita` consiste nella redazione del \textit{Glossario}, dove verranno inseriti tutti i termini considerati possibilmente ambigui;} 
	\item{\textbf{Lettera di presentazione:} attività che consiste nello studio autonomo do ogni membro del bruppo e, sopratutto, nella preparazione del materiale di supporto necessario per la stesura della \textit{Lettera di presentazione} necessaria per il gruppo al fine di essere scelto come fornitore.}
\end{itemize}
\paragraph{Diagramma di Gantt}
->Immagine Diagramma
\subsection{Consolidamento dei requisiti}
Lo metto o non lo metto ?
\subsection{Progettazione architetturale}
Il periodo di \textit{Progettazione architetturale} comincia il giorno dopo la presentazione per la \textit{Reveisione dei requisiti} e si conclude con la consegna dei documenti per la \textit{Revisione di Progettazione}. Le attività principali sono:
\begin{itemize}
	\item{\textbf{Incremento e verifica:} all'inizio del periodo vengono svolte attività di incremento e verifica sui vari documenti consegnati alla \textit{Revisione dei requisiti}}
	\item{\textbf{Glossario:} questa attività comprende sia il miglioramento del Glossario che l’aggiunta di nuovi termini;}
	\item{\textbf{Manuale Utente:}  questa attività consiste nella redazione del Manuale Utente, contenente indicazioni sull’utilizzo del programma che sta venendo prodotto;}
	\item{\textbf{Definizione di prodotto:} questa attività consiste nella redazione del documento della Definizione di Prodotto contenente i dettagli della progettazione architetturale;}
	\item{\textbf{Lettera di presentazione:} Lettera di presentazione: questa attivita` prevede la stesura della Lettera di presentazione per la \textit{Revisione di Progettazione}.}
	\item{\textbf{Specifica tecnica:} questa attività consiste nella redazione da parte dei Progettisti del documento della Specifica Tecnica, contenente le scelte progettuali ad alto livello.}
\end{itemize}
\paragraph{Diagramma di Gantt}
->Immagine Diagramma
\subsection{Progettazione di dettaglio e codifica}
Questo periodo inizia il giorno dopo la \textit{Revisione di Progettazione} e si conclude
con la consegna dei documenti per la \textit{Revisione di Qualifica}. Le attività principali sono:
\begin{itemize}
	\item{\textbf{Incremento e Verifica:} all’inizio del periodo vengono svolte attività di Incremento e Verifica su vari documenti;}
	\item{\textbf{Glossario:} questa attività comprende sia il miglioramento del Glossario che
l’aggiunta di nuovi termini;}
	\item{\textbf{Codifica:}Codifica: questa attività consiste nella scrittura del codice e nella sua verifica secondo quanto indicato nella \textit{Definizione di Prodotto};}
	\item{\textbf{Lettera di presentazione:} questa attività prevede la stesura della lettera
di presentazione per la \textit{Revisione di Qualifica}.}
\end{itemize}
\paragraph{Diagramma di Gantt}
->Immagine Diagramma
\subsection{Validazione e collaudo}
Questo periodo inizia il giorno dopo la Revisione di Qualifica e si conclude con la consegna dei documenti per la Revisione di Accettazione. 
\begin{itemize}
\item{\textbf{Incremento e Verifica:} all’inizio del periodo vengono svolte attività di Incremento e Verifica su vari documenti;}
\item{\textbf{Glossario:} questa attività comprende sia il miglioramento del Glossario che
l’aggiunta dei nuovi termini;}
\item{\textbf{Validazione e Collaudo:} questa attività consiste nell'attuare vari ed ulteriori test al fine di assicurare la massima qualità conforme agli obiettivi richiesti;}
\item{\textbf{Manuale Utente:} questa attività consiste nel miglioramento e completamento del \textit{Manuale utente}, contenente indicazioni sull’utilizzo dell’applicazione.}
\end{itemize}
\paragraph{Diagramma di Gantt}
->Immagine Diagramma

	