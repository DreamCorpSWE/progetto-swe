\documentclass[12pt]{article}

%\usepackage{fancyhdr}
\usepackage{titling}
\usepackage{caption}
\usepackage{multirow}
\usepackage{tabularx}
\usepackage[T1]{fontenc}
\usepackage[utf8]{inputenc}
\usepackage[italian]{babel}
\usepackage{lastpage}
\usepackage{nopageno}
\usepackage{graphicx}
\usepackage [colorlinks=true,urlcolor=blue, linkcolor=black]{hyperref}
\newcommand{\subtitle}[1]{%
    \posttitle{%
        \par\end{center}
        \begin{center}\LARGE#1\end{center}

    \vskip0.5em}%
}

\setlength{\oddsidemargin}{0in}
\setlength{\evensidemargin}{0in}
\setlength{\topmargin}{0in}
\iffalse \setlength{\headsep}{-.25in}\fi
\setlength{\textwidth}{6.5in}
\setlength{\textheight}{8.5in}

\font\myfont=cmr12 at 40pt

\pagestyle{fancy}
\fancyhf{}
\rhead{\leftmark}
\lhead{\includegraphics[width = 20mm]{../logo.png}}
\rfoot{Pagina \thepage \space di \pageref{LastPage}}
\lfoot{Studio di fattibilità}
\renewcommand{\footrulewidth}{0.4pt}
\usepackage{fancyhdr}
\usepackage{titling}
\usepackage{caption}
\usepackage{multirow}
\usepackage{tabularx}
\usepackage[T1]{fontenc}
\usepackage[utf8]{inputenc}
\usepackage[italian]{babel}
\usepackage{lastpage}
\usepackage{nopageno}
\usepackage{graphicx}
\usepackage [colorlinks=true,urlcolor=blue, linkcolor=black]{hyperref}
\newcommand{\subtitle}[1]{%
    \posttitle{%
        \par\end{center}
        \begin{center}\LARGE#1\end{center}

    \vskip0.5em}%
}

\setlength{\oddsidemargin}{0in}
\setlength{\evensidemargin}{0in}
\setlength{\topmargin}{0in}
\iffalse \setlength{\headsep}{-.25in}\fi
\setlength{\textwidth}{6.5in}
\setlength{\textheight}{8.5in}

\font\myfont=cmr12 at 40pt

\pagestyle{fancy}
\fancyhf{}
\rhead{\leftmark}
\lhead{\includegraphics[width = 20mm]{../logo.png}}
\rfoot{Pagina \thepage \space di \pageref{LastPage}}
\lfoot{Studio di fattibilità}
\renewcommand{\footrulewidth}{0.4pt}
\title{\myfont{Piano di Progetto}}
\author{ }
\date{ \myfont 05-12-2018}

\begin{document}
	\maketitle
	\begin{center}
	\huge Versione 0.0.3 
	\\G\&B
	\end{center}
	\newpage
		\begin{tabularx}{\textwidth}{|X|X|X|X|X|}
				\hline
						\textbf{Versione} & \textbf{Data} & \textbf{Descrizione} & \textbf{Autore} & \textbf{Ruolo}\\
				\hline
						0.0.3 & 5/12/2018 & Scrittura introduzione e modello di sviluppo & Pietro Casotto & Responsabile \\
				\hline
						0.0.2 & 5/12/2018 & Analisi dei Rischi & Davide Ghiotto & Analista \\
				\hline
						0.0.1 & 4/12/2018 & Creazione struttura del documento & Davide Ghiotto & Analista  \\
				\hline
		\end{tabularx}
	\newpage
	\tableofcontents
	\newpage


	\section{Introduzione}
		\subsection{Scopo del documento}

			Questo documento ha come scopo quello di presentare la pianificazione del gruppo "Dream Corp." per lo sviluppo del progetto \textit{G\&B}.
			Il documento conterrà anche un preventivo dei costi e un'analisi dei rischi previsti per lo svolgimento e realizzazione del progetto.\newline
			In dettaglio il documento coprirà:
			\begin{itemize}
				\item Breve analisi del modello di sviluppo per il progetto; 
				\item Analisi dei rischi relativi al progetto;
				\item Dettagliata pianificazione dei tempi e delle attività;
				\item Stima preventiva dell'utilizzo delle risorse disponibili.
			\end{itemize}

		\subsection{Scopo del progetto}

			Lo scopo del progetto è quello di realizzare un Plugin per Grafana\ped d per integrare metodi di intelligenza artificiale al flusso dei dati raccolti con lo scopo di monitorare lo stato del sistema e migliorare il software utilizzato.

		\subsection{Note esplicative}

			Allo scopo di evitare ambiguità a lettori esterni al gruppo, si specifica che all'interno del documento verranno inseriti dei termini con un carattere g come pedice, questo significa che il significato inteso in quella situazione è stato inserito nel Glossario;

		\newpage
		\subsection{Riferimenti}

			\subsubsection{Riferimenti Normativi}
				\begin{itemize}
					\item Regole organigramma e specifica tecnica-economica:

					\url{https://www.math.unipd.it/~tullio/IS-1/2018//Progetto/RO.html}
						\begin{itemize}
							\item[-] Offerta tecnico-economica;
							\item[-]Struttura organigramma.
						\end{itemize}
					\item Norme di progetto
					\item Capitolato d'appalto C3 - G\&B: monitoraggio intelligente di processi DevOps
					
					\url{https://www.math.unipd.it/~tullio/IS-1/2018/Progetto/C3.pdf}
				\end{itemize}
				
			\subsubsection{Riferimenti Informativi}
				\begin{itemize}
					\item Slide del corso "Ingegneria del Software" - Gestione di progetto:
					
					\url{http://www.math.unipd.it/~tullio/IS-1/2018/Dispense/L06.pdf}
					\item Slide del corso \"Ingegneria del Software" - Regole del Progetto Didattico:

					\url{http://www.math.unipd.it/~tullio/IS-1/2018/Dispense/P01.pdf}

					\item Software Engineering - Ian Sommerville (10th Edition)
				\end{itemize}

		\subsection{Scadenze}
			Il gruppo Dream Corp. ha deciso di rispettare le seguenti scadenze su cui si baserà la pianificazione di tutto il progetto
			\begin{itemize}
				\item \textbf{Revisione dei Requisiti:} 
				\item \textbf{Revisione dei Progettazione:} 
				\item \textbf{Revisione dei Qualifica:} 
				\item \textbf{Revisione dei Accettazione:} 
			\end{itemize}



	\newpage

	\section{Analisi dei rischi}

		Di seguito verranno definiti i rischi individuati dal gruppo Dream Corp. associando ad ognuno di essi un impatto sul progetto e una probabilità che si verifichi. Viene associato inoltre una risposta contenitiva del rischio. 
		\newline \newline
		\begin{tabularx}{\textwidth}{|X|X|X|X|X|}
			\hline
			\textbf{Identificativo} & \textbf{Descrizione rischio} & \textbf{Probabilità} & \textbf{Impatto} & \textbf{Risposta Contenitiva}\\
			\hline
			R1 & Errata stima della durata delle attività da programmare & Alta & Alto & Considerare sempre un certo slack temporale e pianificazione all’indietro\\
			\hline
			R2 & Conflitti interni al gruppo di progetto & Bassa & Basso & Team building sociale e confronti diretti\\
			\hline
			R3 & Sottostima del tempo richiesto per l’apprendimento di nuove tecnologie (GitHub, Grafana,...) & Media & Medio & Confronto con esperti per dubbi e chiarimenti al fine di prevenire ritardi\\
			\hline
			R4 & Problemi Software di terze parti (perdita di dati su GitHub) & Molto  Bassa & Basso &  Backup due volte la settimana\\
			\hline
			R5 & Ridefinizione dei requisiti & Bassa & Molto Alto & ?\\
			\hline
		\end{tabularx}
		\newline \newline
		Per analizzare il fattore di rischio del progetto adottiamo una scala numerica per l’impatto così definita:
		\begin{itemize}
			\item Molto Basso: 1
			\item Basso: 2
			\item Medio: 3
			\item Alto: 4
			\item Molto Alto: 5
		\end{itemize}
		Ed inoltre creiamo una scala di probabilità discreta così definita:
		\begin{itemize}
			\item Molto Bassa: 10%
			\item Bassa: 25%
			\item Media: 50%
			\item Alta: 75%
			\item Molto Alta: 90%
		\end{itemize}
		L’utilizzo di una scala numerica ci permette di analizzare più efficacemente l’impatto probabile all’interno del nostro progetto e quindi anche di pianificare all’indietro tenendo conto di queste analisi.
		Elaborazione dei nostri rischi:
		\newline \newline
		\begin{tabularx}{\textwidth}{|X|X|X|X|}
			\hline
			r1 & 75\% & 4 & 3 \\
			\hline
			r2 &  25\%  & 2  & 0.5 \\
			\hline
			r3  & 50\%  & 3  & 1.5 \\
			\hline
			r4  & 10\% &  2 &  0.2 \\
			\hline
			r5  & 25\%  & 5  & 1.25 \\
			\hline
			\textbf{Media}  & 37\% &  3.2 &  28.66666667\\
			\hline
		\end{tabularx}
		\newline \newline
		Dalla nostra distribuzione di rischi e probabilità otteniamo un profilo di rischio:
		Probabilità media di un rischio = 37\% (Bassa-Media)
		Impatto medio di un rischio = 3.2 (Medio)
		Somma impatto*probabilità = 6.45
		Massima somma impatto*probabilità = 0.9(Molto Alta) * numero\_rischi *5 (Molto Alto) = 22.5
		Profilo di rischio= 6.45/22.5 = 28,7\% (Basso)
		Concludendo possiamo ritenere il nostro profilo di rischio abbastanza basso


	\section{modello di sviluppo}
	->scelta del modello che useremo


	\section{pianificazione}
	scopo (BB)
	analisi (avvio e analisi)
		-> elenco attività in ordine cronologico
		->divisione dei ruoli
	consolidamento dei requisiti (353)
	-> dettaglio delle varie fasi


	\section{preventivo}
	    \subsection{Dettaglio fasi}
		\subsection{prospetto orario}
		\subsection{conteggio ore}


	\section{consuntivo e preventivo a finire}





\end{document}
