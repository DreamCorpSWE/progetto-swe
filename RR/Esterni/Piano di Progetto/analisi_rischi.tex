\section{Analisi dei rischi}

	Di seguito verranno definiti i rischi individuati dal gruppo Dream Corp. associando ad ognuno di essi un impatto sul progetto e una probabilità che si verifichi. Viene associato inoltre una risposta contenitiva del rischio. 
	\newline \newline
	\begin{tabularx}{\textwidth}{|X|X|X|X|X|}
		\hline
		\textbf{Identificativo} & \textbf{Descrizione rischio} & \textbf{Probabilità} & \textbf{Impatto} & \textbf{Risposta Contenitiva}\\
		\hline
		R1 & Errata stima della durata delle attività da programmare & Alta & Alto & Considerare sempre un certo slack temporale e pianificazione all’indietro\\
		\hline
		R2 & Conflitti interni al gruppo di progetto & Bassa & Basso & Team building sociale e confronti diretti\\
		\hline
		R3 & Sottostima del tempo richiesto per l’apprendimento di nuove tecnologie (GitHub, Grafana,...) & Media & Medio & Confronto con esperti per dubbi e chiarimenti al fine di prevenire ritardi\\
		\hline
		R4 & Problemi Software di terze parti (perdita di dati su GitHub) & Molto  Bassa & Basso &  Backup due volte la settimana\\
		\hline
		R5 & Ridefinizione dei requisiti & Bassa & Molto Alto & ?\\
		\hline
	\end{tabularx}
	\newline 
	 \newline
	 \newline
	Per analizzare il fattore di rischio del progetto adottiamo una scala numerica per l’impatto così definita:
	\begin{itemize}
		\item Molto Basso: 1
		\item Basso: 2
		\item Medio: 3
		\item Alto: 4
		\item Molto Alto: 5
	\end{itemize}
	Ed inoltre creiamo una scala di probabilità discreta così definita:
	\begin{itemize}
		\item Molto Bassa: 10%
		\item Bassa: 25%
		\item Media: 50%
		\item Alta: 75%
		\item Molto Alta: 90%
	\end{itemize}
	L’utilizzo di una scala numerica ci permette di analizzare più efficacemente l’impatto probabile all’interno del nostro progetto e quindi anche di pianificare all’indietro tenendo conto di queste analisi.
	Elaborazione dei nostri rischi:
	\newline \newline
	\begin{tabularx}{\textwidth}{|X|X|X|X|}
		\hline
		r1 & 75\% & 4 & 3 \\
		\hline
		r2 &  25\%  & 2  & 0.5 \\
		\hline
		r3  & 50\%  & 3  & 1.5 \\
		\hline
		r4  & 10\% &  2 &  0.2 \\
		\hline
		r5  & 25\%  & 5  & 1.25 \\
		\hline
		\textbf{Media}  & 37\% &  3.2 &  28.66666667\\
		\hline
	\end{tabularx}
	\newline \newline
	Dalla nostra distribuzione di rischi e probabilità otteniamo un profilo di rischio:
	Probabilità media di un rischio = 37\% (Bassa-Media)
	Impatto medio di un rischio = 3.2 (Medio)
	Somma impatto*probabilità = 6.45
	Massima somma impatto*probabilità = 0.9(Molto Alta) * numero\_rischi *5 (Molto Alto) = 22.5
	Profilo di rischio= 6.45/22.5 = 28,7\% (Basso)
	Concludendo possiamo ritenere il nostro profilo di rischio abbastanza basso