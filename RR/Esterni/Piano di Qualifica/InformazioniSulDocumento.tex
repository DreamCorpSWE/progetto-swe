\section{Informazioni sul documento}
\subsection{Scopo del documento}
 Col fine di mantenere alta la qualità del prodotto finale il gruppo \textit{DreamCorp} ha stilato questo documento che descrive i metodi con cui analizzerà e verifichierà i processi attuati. \textbf{FAR CAPIRE CHE IL DOCUMENTO E' INCREMENTALE NEL TEMPO}
 \textbf{DA COMPLETARE}
 \subsection{Scopo del prodotto}
 Lo scopo è quello di creare un plugin per il sistema grafana che, tramite reti bayesiane, monitora in maniera intelligente lo stato dei server.
 \subsection{Glossario e documentazione}
 In questo documento sono presenti termini di non immediata comprensione. Con lo scopo di disambiguare quest'ultimi è stato redatto un glossario, seganalti con una G a pedice.
 Inoltre, per lo stesso motivo, i documenti prodotti dal gruppo saranno segnalati con una D a pedice
 \newpage
 \subsection{Riferimenti}
 \subsubsection{Riferimenti normativi}
 \begin{itemize}
 	\item Norme di progetto
 	\item Standard ISO/IEC 9126 (modello di qualità)
	\item  \href{https://}{Qualità del software} (slide del corso di Ingegneria del Software) (\textbf{secondo me va sui riferimenti normativi perchè sono servite per creare il documento})
	\item  \href{https://it.wikipedia.org/wiki/Qualità_del_software}{Qualità del software} (elenco di metriche utili) (\textbf{secondo me va sui riferimenti normativi perchè sono servite per creare il documento})
 	\item  \href{https://}{Qualità di processo} (slide del corso di Ingegneria del Software) (\textbf{secondo me va sui riferimenti normativi perchè sono servite per creare il documento})
 \end{itemize}
\subsubsection{Riferimenti informativi}
 \begin{itemize}
 	\item  \href{https://it.wikipedia.org/wiki/Indice_Gulpease}{Gulpease index} (per calcolare la metrica)
	\item  \href{https://en.wikipedia.org/wiki/Gunning_fog_index}{Gunning Fog Index} (per calcolare la metrica)
	\item  \href{https://en.wikipedia.org/wiki/SMOG}{Simple Measure of Gobbledygook} (per calcolare la metrica)
	\item  \href{https://en.wikipedia.org/wiki/Code_coverage}{Code coverage} (per definire le metriche)
 \end{itemize}

