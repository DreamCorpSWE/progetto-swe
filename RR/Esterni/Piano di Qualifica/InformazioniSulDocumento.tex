\section{Informazioni sul documento}
\subsection{Scopo del documento}
 Col fine di mantenere alta la qualità del prodotto finale il gruppo Dream Corp. ha stilato questo documento che descrive i metodi con cui analizzerà e verifichierà i processi attuati.
 \subsection{Scopo del progetto}
 Lo scopo è quello di creare un Plugin\pedice per Grafana\pedice per integrare metodi di intelligenza artificale al flusso dei dati raccolti con lo scopo di monitorare lo stato del sistema e migliorare il sfotware utilizzato
 \subsection{Glossario}
 In questo documento sono presenti termini di non immediata comprensione. Con lo scopo di disambiguare quest'ultimi è stato redatto un glossario, segnalati con una G a pedice.

 \newpage
 \subsection{Riferimenti}
 \subsubsection{Riferimenti normativi}
 \begin{itemize}
 	\item \textit{Norme di progetto};
 	\item Standard ISO/IEC 9126:
 		\begin{itemize}
 			\item[-] Modello di qualità;
 		\end{itemize}
	\item Slide del corso di "Ingegneria del Software" - Qualità del Software: \\
		\url{https://www.math.unipd.it/~tullio/IS-1/2018/Dispense/L13.pdf}
	\item  Slide del corso di "Ingegneria del Software" - Qualità di Processo: \\
		\url{https://www.math.unipd.it/~tullio/IS-1/2018/Dispense/L14.pdf}
 \end{itemize}
\subsubsection{Riferimenti informativi}
 \begin{itemize}
 	\item Slide del corso di "Ingegneria del Software" - Qualità del Software: \\
		\url{https://www.math.unipd.it/~tullio/IS-1/2018/Dispense/L13.pdf} 
	\item Qualità del software: \\
		\url{https://it.wikipedia.org/wiki/Qualità_del_software}
		\begin{itemize}
			\item[-] Elenco di metriche utili.
		\end{itemize} 
	\item  Slide del corso di "Ingegneria del Software" - Qualità di Processo: \\
		\url{https://www.math.unipd.it/~tullio/IS-1/2018/Dispense/L14.pdf}
	\item Metriche di progetto \\
 		\url{https://it.wikipedia.org/wiki/Metriche_di_progetto}
 	\item Gulpease index: \\
 		\url{https://it.wikipedia.org/wiki/Indice_Gulpease}
 	\begin{itemize}
 		\item[-] Formula di calcolo.
	\end{itemize}
	\item Gunning Fog Index: \\
		\url{https://en.wikipedia.org/wiki/Gunning_fog_index}
		\begin{itemize}
		\item[-] Formula di calcolo.
		\end{itemize}
	\item Simple Measure of Gobbledygook: \\ 
		 \url{https://en.wikipedia.org/wiki/SMOG}
		 \begin{itemize}
		 	\item[-] Formula di calcolo.
		 \end{itemize} \clearpage
	\item Code coverage: \\ 
		\url{https://en.wikipedia.org/wiki/Code_coverage}
		\begin{itemize}
		\item[-] struttura e definizione metriche.
		\end{itemize}
\end{itemize}