\section{Informazioni sul documento}
\subsection{Scopo del documento}
 Col fine di mantenere alta la qualità del prodotto finale il gruppo \textit{DreamCorp} ha stilato questo documento che descrive i metodi con cui analizzerà e verifichierà i processi attuati. \textbf{FAR CAPIRE CHE IL DOCUMENTO E' INCREMENTALE NEL TEMPO}
 \textbf{DA COMPLETARE}
 \subsection{Scopo del prodotto}
 Lo scopo è quello di creare un plugin per il sistema grafana che, tramite reti bayesiane, monitora in maniera intelligente lo stato dei server.
 \subsection{Glossario e documentazione}
 In questo documento sono presenti termini di non immediata comprensione. Con lo scopo di disambiguare quest'ultimi è stato redatto un glossario, seganalti con una G a pedice.
 Inoltre, per lo stesso motivo, i documenti prodotti dal gruppo saranno segnalati con una D a pedice \textbf{TENGO IL DISCORSO SUL PEDICE ? }
 \newpage
 \subsection{Riferimenti}
 \subsubsection{Riferimenti normativi}
 \begin{itemize}
 	\item Norme di progetto;
 	\item Standard ISO/IEC 9126:
 		\begin{itemize}
 			\item[-] Modello di qualità.
 		\end{itemize}
	\item Slide del corso di "Ingegneria del Software" - Qualità del Software: \\
		\url{https://www.math.unipd.it/~tullio/IS-1/2018/Dispense/L13.pdf} \textbf{secondo me va sui riferimenti normativi perchè sono servite per creare il documento})
	\item Qualità del software: \\
  			\url{https://it.wikipedia.org/wiki/Qualità_del_software} \textbf{secondo me va sui riferimenti normativi perchè sono servite per creare il documento})
  			\begin{itemize}
  				\item[-] Elenco di metriche utili.
  			\end{itemize} 
	\item  Slide del corso di "Ingegneria del Software" - Qualità di Processo: \\
 			\url {https://www.math.unipd.it/~tullio/IS-1/2018/Dispense/L14.pdf}  \textbf{secondo me va sui riferimenti normativi perchè sono servite per creare il documento}
 \end{itemize}
\subsubsection{Riferimenti informativi}
 \begin{itemize}
 	\item Gulpease index: \\
 		\url{https://it.wikipedia.org/wiki/Indice_Gulpease}
 	\begin{itemize}
 		\item[-] Formula di calcolo.
	\end{itemize}
	\item Gunning Fog Index: \\
		\url{https://en.wikipedia.org/wiki/Gunning_fog_index}
		\begin{itemize}
		\item[-] Formula di calcolo.
		\end{itemize}
	\item Simple Measure of Gobbledygook: \\ 
		 \url{https://en.wikipedia.org/wiki/SMOG}
		 \begin{itemize}
		 	\item[-] Formula di calcolo.
		 \end{itemize}
	\item Code coverage: \\ 
		\url{https://en.wikipedia.org/wiki/Code_coverage}
		\begin{itemize}
		\item[-] struttura e definizione metriche.
		\end{itemize}
\end{itemize}