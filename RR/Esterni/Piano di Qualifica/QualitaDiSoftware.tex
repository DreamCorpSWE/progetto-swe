\section{Qualità del Prodotto}
\subsection{Scopo}
Qualità del software effettivo, leggibilità dei documenti (sono un prodotto anche quelli), misura dei test.

\section{Prodotti}
\subsection{Qualità dei documenti}
Ci si prefigge lo scopo di creare dei documenti standardizzati, per questo i nostri obiettivi sono:
\begin{itemize}
	\item{\textbf{Comprensibilià:} devono venire creati dei coumenti di immediata comprensione, per questo si prediligono frasi incisive e si pone l'accento su elementi tecnici presentati da tabelle;}
	\item{\textbf{Correttezza:} non devono contenere errori ortografici;}
	\item{\textbf{Leggibilità:} nonostante lo scopo tecnico i documenti devono essere fruibili alla maggior parte delle persone.}
\end{itemize}
Vengono così utilizzate le seguenti metriche per poter dare un ranking ai documenti:
\paragraph{Metriche utilizzate}
\begin{itemize}
	\item{\textbf{Gulpease Index}}
	\item{\textbf{Errori sintattici}}
	\item{\textbf{Gunning Fog index}}
	\item{\textbf{Coleman Liau index/SMOG}}
\end{itemize}
\begin{table}[!htpb]
	\resizebox{\textwidth}{!}{%
		\begin{tabular}{|l|l|l|}
			\hline
			\rowcolor[HTML]{34CDF9} 
			{\color[HTML]{333333} \textbf{Nome}} & {\color[HTML]{333333} \textbf{Accettazione}} & {\color[HTML]{333333} \textbf{Ottimalità}} \\ \hline
			Gulpease index                       &                                              &                                            \\ \hline
			Errori sintattici                    &                                              &                                            \\ \hline
			Gunning Fog index                    &                                              &                                            \\ \hline
			Coleman Liau index                   &                                              &                                            \\ \hline
			SMOG                                 &                                              &                                            \\ \hline
		\end{tabular}%
	}
\end{table}
\subsection{Qualità del software}
Per qualità del software si intende la misura in cui un prodtto software soddisfa un certo numero di aspettative rispetto sia al suo funzionamento sia alla sua struttura interna. I parametri verranno classificati:
\begin{itemize}
	\item{\textbf{Interni (Int):} qualità percepita dagli sviluppatori;}
	\item{\textbf{Esterni (Ext):} qualità percepita dall'utente finale.}
\end{itemize}
\subsubsection{(Int) Correttezza}
\subsubsection{(Int) Affidibalità}
\subsubsection{(Int) Efficienza}
Rappresenta l'indice di utilizzo delle risorse in maniera proprozionato rispetto ai servizi che svolge. Per questo si sono prefissati i seguenti obiettivi:
\begin{itemize}
	\item{\textbf{Deadline:}  il programma deve svolgere il lavoro entro i tempi stabiliti;}
	\item{\textbf{Prestazioni:} si cerca di mantenere le deadline con il minor utilizzo delle risorse.}
\end{itemize}
