\section{Qualità del Prodotto}
\subsection{Scopo}
Qualità del software effettivo, leggibilità dei documenti (sono un prodotto anche quelli), misura dei test.
Le norme UNI definiscono la qualità come \textit{"l'insieme delle carrateristiche che gli conferiscono la capacità di soddisfare esigenze espresse o implicite"}
\subsection{Prodotti}
\subsection{Qualità dei documenti}
Ci si prefigge lo scopo di creare dei documenti standardizzati, per questo i nostri obiettivi sono:
\begin{itemize}
	\item{\textbf{Comprensibilià:} devono venire creati dei coumenti di immediata comprensione, per questo si prediligono frasi incisive e si pone l'accento su elementi tecnici presentati da tabelle;}
	\item{\textbf{Correttezza:} non devono contenere errori ortografici;}
	\item{\textbf{Leggibilità:} nonostante lo scopo tecnico i documenti devono essere fruibili alla maggior parte delle persone.}
\end{itemize}
Vengono così utilizzate le seguenti metriche per poter dare un ranking ai documenti:
\paragraph{Metriche utilizzate}
\begin{itemize}
	\item{\textbf{Gulpease Index}}
	\item{\textbf{Errori sintattici}}
	\item{\textbf{Gunning Fog index}}
	\item{\textbf{Coleman Liau index/SMOG}}
\end{itemize}
\begin{table}[!htpb]
	\resizebox{\textwidth}{!}{%
		\begin{tabular}{|l|l|l|}
			\hline
			\rowcolor[HTML]{34CDF9} 
			{\color[HTML]{333333} \textbf{Nome}} & {\color[HTML]{333333} \textbf{Accettazione}} & {\color[HTML]{333333} \textbf{Ottimalità}} \\ \hline
			Gulpease index                       &                                              &                                            \\ \hline
			Errori sintattici                    &                                              &                                            \\ \hline
			Gunning Fog index                    &                                              &                                            \\ \hline
			Coleman Liau index                   &                                              &                                            \\ \hline
			SMOG                                 &                                              &                                            \\ \hline
		\end{tabular}%
	}
\end{table}
\subsection{Qualità del software}
Per qualità del software si intende la misura in cui un prodtto software soddisfa un certo numero di aspettative rispetto sia al suo funzionamento sia alla sua struttura interna. I parametri verranno classificati:
\begin{itemize}
	\item{\textbf{Interni (Int):} qualità percepita dagli sviluppatori;}
	\item{\textbf{Esterni (Ext):} qualità percepita dall'utente finale.}
\end{itemize}
\subsubsection{(Int) Correttezza}
\subsubsection{(Int) Affidibalità}
	Un sistema è tanto più affidabile quanto più raramente, durantel 'uso del sistema, si manifestano malfunzionamenti. Pertanto ci siamo posti questi obiettivi:
	\begin{itemize}
	\item \textbf{Adattabilità:} adattarsi al tipo di utente;
	\item \textbf{Tempo medio:} tenere basso il tempo medio che intercorre tra due fallimenti successivi.
\end{itemize}
\paragraph{Metriche utilizzate}
\begin{itemize}
	\item \textbf{MTBF(Mean Time Between Failure)}
\end{itemize}
\begin{table}[!htpb]
	\resizebox{\textwidth}{!}{%
		\begin{tabular}{|l|l|l|}
			\hline
			\rowcolor[HTML]{34CDF9} 
			{\color[HTML]{333333} \textbf{Nome}} & {\color[HTML]{333333} \textbf{Accettazione}} & {\color[HTML]{333333} \textbf{Ottimalità}} \\ \hline
			 MTBF                      &                                              &                                            \\ \hline
			                   &                                              &                                            \\ \hline
			                    &                                              &                                            \\ \hline
			                   &                                              &                                            \\ \hline
			                                 &                                              &                                            \\ \hline
		\end{tabular}%
	}
\end{table}
\subsubsection{(Int) Efficienza (Riferimento ai requisiti prestazionali)}
Rappresenta la capacità di eseguire le proprie funzionalità con un buon rapporto tra tempo d'esecuzione e utilizzo delle risorse. Per questo ci prefiggiamo i seguenti obiettivi:
\begin{itemize}
	\item \textbf{Utilizzo delle risorse:} le fdunzionalità del software devono ponderare l'utilizzo delle risorse a diposizione.
\end{itemize}
\paragraph{Metriche utilizzate}
\begin{itemize}
	\item \textbf{Tempo di risposta}
\end{itemize}
\begin{table}[!htpb]
	\resizebox{\textwidth}{!}{%
		\begin{tabular}{|l|l|l|}
			\hline
			\rowcolor[HTML]{34CDF9} 
			{\color[HTML]{333333} \textbf{Nome}} & {\color[HTML]{333333} \textbf{Accettazione}} & {\color[HTML]{333333} \textbf{Ottimalità}} \\ \hline
			Tempo di risposta                      &                                              &                                            \\ \hline
			&                                              &                                            \\ \hline
			&                                              &                                            \\ \hline

		\end{tabular}%
	}
\end{table}
\subsubsection{Manutenibilità}
Riguarda la facilità di apportare modifiche al sistema realizzato. Sono prefissati i seguenti obiettivi:
\begin{itemize}
	\item \textbf{}
\end{itemize}

\subsubsection{(Ext) Portabilità}
Rappresenta la caratteristica di poter funzionare su ambienti diversi. Pertanto ci siamo prefissati i seguenti obiettivi:
\begin{itemize}
	\item \textbf{Adattabilità:} il prodotto deve adattarsi con il minimo sforzo a tutti gli ambienti di lavoro prefissati;
	\item \textbf{Sostituibilità:} il prodotto deve poter sostiture un altro software che fa la stessa cosa (RIFERITO AL DISCORSO CHE HA FATTO IL TIPO DELLA ZUCCHETTI)
\end{itemize}
\paragraph{Metriche utilizzate}
\begin{itemize}
	\item \textbf{Supporto browser}
	\item \textbf{Funzionalità già esistenti} (intendo cosa sa già fare rispetto al programma che va a sostituire e cosa c'è di nuovo)
\end{itemize}
\begin{table}[!htpb]
	\resizebox{\textwidth}{!}{%
		\begin{tabular}{|l|l|l|}
			\hline
			\rowcolor[HTML]{34CDF9} 
			{\color[HTML]{333333} \textbf{Nome}} & {\color[HTML]{333333} \textbf{Accettazione}} & {\color[HTML]{333333} \textbf{Ottimalità}} \\ \hline
			 Supporto Browser                 &                                   &                                 \\ \hline
			Funzionalità già eistenti             &                                 &                 \\ \hline
		\end{tabular}%
	}
\end{table}
\subsubsection{(Ext) Evolvità}
Rappresenta l'indice di utilizzo delle risorse in maniera proprozionato rispetto ai servizi che svolge. Per questo si sono prefissati i seguenti obiettivi:
\begin{itemize}
	\item{\textbf{Deadline:}  il programma deve svolgere il lavoro entro i tempi stabiliti;}
	\item{\textbf{Prestazioni:} si cerca di mantenere le deadline con il minor utilizzo delle risorse.}
\end{itemize}
\begin{table}[!htpb]
	\resizebox{\textwidth}{!}{%
		\begin{tabular}{|l|l|l|}
			\hline
			\rowcolor[HTML]{34CDF9} 
			{\color[HTML]{333333} \textbf{Nome}} & {\color[HTML]{333333} \textbf{Accettazione}} & {\color[HTML]{333333} \textbf{Ottimalità}} \\ \hline
			                     &                                              &                                            \\ \hline
			&                                              &                                            \\ \hline
			&                                              &                                            \\ \hline
			
		\end{tabular}%
	}
\end{table}