\section{Qualità di Prodotto}
\subsection{Scopo}
Per garantire una buona qualità di prodotto, il gruppo ha individuato dallo standard ISO/IEC 9126\pedice~gli obiettivi di qualità adeguati al contesto ed ha stilato le seguenti metriche coerenti con essi, mantendendo cosí il livello di qualità perseguito.

\subsection{Prodotti}
Le metriche utilizzate per i prodotti sono definite nelle \textit{Norme di progetto} alla §3.5:
\subsubsection{Qualità dei documenti}
Ci si prefigge lo scopo di creare dei documenti standardizzati, per questo i nostri obiettivi sono:
\begin{itemize}
	\item{\textbf{Comprensibilià:} devono venire creati dei documenti di immediata comprensione, per questo si prediligono frasi incisive e si pone l'accento su elementi tecnici presentati da tabelle;}
	\item{\textbf{Correttezza:} non devono contenere errori ortografici;}
	\item{\textbf{Leggibilità:} nonostante lo scopo tecnico i documenti devono essere fruibili alla maggior parte delle persone.}
\end{itemize}
\vspace{0.8cm}
%Vengono così utilizzate le seguenti metriche per poter dare un ranking ai documenti:
\subparagraph{Metriche utilizzate}
\begin{itemize}
	\item{\textbf{Gulpease Index\pedice};}
	\item{\textbf{Errori sintattici};}
	\item{\textbf{Gunning Fog index\pedice};}
	\item{\textbf{Simple measure of Goobledygook (SMOG)\pedice}.}
\end{itemize}
\begin{table}[H]
	\centering
	\renewcommand{\arraystretch}{2} 
	\rowcolors{2}{gray!25}{white}
		\begin{tabular}{|l|l|l|}
			\rowcolor{orange!50}
			\hline
			\textbf{Nome} & \textbf{Accettazione} & \textbf{Ottimalità} \\ \hline
			Gulpease index    &   50-100   &    65-100       \\ \hline
			Errori sintattici     &  0    &   0      \\ \hline
			Gunning Fog index    &     \textless= 13      &   \textless= 10   \\ \hline
			SMOG                  &              \textless= 13                                &                         \textless= 10                   \\ \hline
		\end{tabular}
	\caption{Metriche utilizzate Qualità dei documenti}
\end{table}
\subsubsection{Qualità del software}
Per qualità del software si intende la misura in cui un prodotto software soddisfa un certo numero di aspettative rispetto sia al suo funzionamento che alla struttura interna. 

\myparagraph{E1: Correttezza} 
Un programma o sistema software si dice corretto se si comporta esattamente secondo quanto previsto dalla sua specifica dei requisiti. Per raggiungerla ci siamo posti i seguenti obiettivi:
\begin{itemize}
	\item \textbf{Soddisfazione del cliente:} il cliente è soddisfatto se il software è conforme ai requisiti;
	\item \textbf{Requisiti opzionali soddisfatti:} soddisfare più requisiti opzionali possibili.
\end{itemize}
\subparagraph{Metriche utilizzate}
\begin{itemize}
	\item \textbf{Percentuale requisiti fondamentali soddisfatti};
	\item \textbf{Percentuale requisiti opzionali soddisfatti}.
\end{itemize}
\begin{table}[H]
	\centering
	\renewcommand{\arraystretch}{2} 
	\rowcolors{2}{gray!25}{white}
		\begin{tabular}{|l|l|l|}
			\rowcolor{orange!50}
			\hline
			\textbf{Nome} & \textbf{Accettazione} & \textbf{Ottimalità} \\ \hline
			Requisiti fondamentali soddisfatti  &  95\%  &  100\%      \\ \hline
			Requisiti secondari soddisfatti  &   0\% &   80\%     \\ \hline
		\end{tabular}
	\caption{Metriche utilizzate per la Correttezza}
\end{table}
\myparagraph{E2: Affidabilità}
	Un sistema è tanto più affidabile quanto più raramente, durante l'uso del sistema, si manifestano malfunzionamenti. Pertanto ci siamo posti questi obiettivi:
	\begin{itemize}
	\item \textbf{Adattabilità:} adattarsi al tipo di utente;
	\item \textbf{Tempo medio:} tenere basso il tempo medio che intercorre tra due fallimenti successivi.
\end{itemize}

\subparagraph{Metriche utilizzate}
\begin{itemize}
	\item \textbf{Mean Time Between Failures (MTBF)\pedice;}
	\item \textbf{Blocco operazioni non corrette;}
	\item \textbf{Test conclusi in failure.}
\end{itemize}
\begin{table}[!htpb]
	\centering
	\renewcommand{\arraystretch}{2} 
	\rowcolors{2}{gray!25}{white}
	\begin{tabular}{|l|l|l|}
		\rowcolor{orange!50}
		\hline
		\textbf{Nome} &  \textbf{Accettazione} & \textbf{Ottimalità} \\ \hline
		MTBF     &     \textless= 2 ogni 5 build       &     \textless=1 ogni 5 build          \\ \hline
		Blocco operazioni non corrette   &    0-20\% &  0\%       \\ \hline
		Test conlusi in failure          &     0-10\%          &      0\%        \\ \hline
	\end{tabular}
	\caption{Metriche utilizzate per Affidabilità}
\end{table}
\myparagraph{I1: Efficienza}
Rappresenta la capacità di eseguire le proprie funzionalità con un buon rapporto tra tempo d'esecuzione e utilizzo delle risorse. Per questo ci prefiggiamo i seguenti obiettivi:
\begin{itemize}
	\item \textbf{Utilizzo delle risorse:} le funzionalità del software devono ponderare l'utilizzo delle risorse a diposizione;
	\item \textbf{Rispettare la deadline:} il software deve eseguire le proprie funzionalità entro i tempi prestabiliti.
\end{itemize}

\subparagraph{Metriche utilizzate}
\begin{itemize}
	\item \textbf{Tempo di risposta.}
\end{itemize}
\begin{table}[!htpb]
	\centering
	\renewcommand{\arraystretch}{2} 
	\rowcolors{2}{gray!25}{white}
		\begin{tabular}{|l|l|l|}
			\rowcolor{orange!50}
			\hline
			\textbf{Nome} & \textbf{Accettazione} & \textbf{Ottimalità} \\ \hline
			Tempo di risposta        &           $<$ 300ms            &      $<$ 100ms         \\ \hline                       
		\end{tabular}
	\caption{Metriche utilizzate Efficienza}
\end{table}

\myparagraph{I2: Manutenibilità}
Riguarda la facilità di apportare modifiche al sistema realizzato. Sono prefissati i seguenti obiettivi:
\begin{itemize}
	\item \textbf{Incapsulamento:} ogni parte del software aggiunta deve rispettare il singular responsabilty principle\pedice;
	\item \textbf{Stabilità:} ogni parte del software aggiunta deve essere stabile come le precedenti;
	\item \textbf{Testabilità:} ogni parte deve essere facilmente testabile come le precedenti.
\end{itemize}

\subparagraph{Metriche utilizzate}
\begin{itemize}
	\item \textbf{Impatto nuove aggiunte}
\end{itemize}
\begin{table}[H]
	\centering
	\renewcommand{\arraystretch}{2} 
	\rowcolors{2}{gray!25}{white}
		\begin{tabular}{|l|l|l|}
			\rowcolor{orange!50}
			\hline
			\textbf{Nome} & \textbf{Accettazione} & \textbf{Ottimalità} \\ \hline
			Impatto nuove aggiunte        &           0-30\%            &      0-10\%         \\ \hline                       
		\end{tabular}
	\caption{Metriche utilizzate Manutenibilità}
\end{table}

\subsubsection{Tracciamento Parametri Qualità del Software}
Di seguito viene riportata una tabella riassuntiva che riporta per ogni parametro, su cui viene misurata la qualità del software, il tipo e il numero di metriche scelte. \\ \\ Questa tabella è utile per avere una visione d'insieme dei parametri di qualità del software e per il tracciamento di questi ultimi.

\begin{table}[H]
	\centering
	\renewcommand{\arraystretch}{2} 
	\rowcolors{2}{gray!25}{white}
		\begin{tabular}{|c|c|c|}
			\rowcolor{orange!50}
			\hline
			\textbf{Identificativo} & \textbf{Tipologia} & \textbf{Numero Metriche} \\ \hline
			E1 & Esterno & 2 \\ \hline
			E2 & Esterno & 3 \\ \hline                       
			I1 & Interno & 1 \\ \hline                       
			I2 & Interno & 1 \\ \hline                       
			\textbf{Totale} &  & 7 \\ \hline   
		\end{tabular}
	\caption{Tracciamento Parametri Qualità del Software}
\end{table}

\clearpage
