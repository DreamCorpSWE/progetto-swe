\section{Qualità di processo}
\subsection{Scopo}
\textit{"Da tubi sporchi non esce acqua pulita"}.
\newline
Con questa frase questo documento si prefigge  lo scopo di adottare la qualità di processo come esigenza fondamentale per perseguire la qualità di prodotto. Proprio per questo si è deciso di adottare il PDCA e lo standard ISO/IEC 15504 denominato SPICE. Inoltre si vuole far presente come l'insieme di questi contenuti non sia definitivo ma anzi viene incrementato durante il percorso. \textbf{Questo documento deve rispondere al cosa e non al come.}
%\subsection{Procedure di controllo}
%La qualità di processo viene suddivisa in:
%\begin{itemize}
%	\item{\textbf{Definizione:} per controllarlo e raccontarlo meglio};
%	\item{\textbf{Controllo:} perchè sia conforme alle attese e costi meno};
%	\item{\textbf{Validazione:}  per validare tramite PDCA.}
%\end{itemize}
\subsection{Processi}
Con l'obiettivo di ottenere un miglioramento continuo della qualità in un'ottica a lungo raggio e all'utilizzo ottimale delle risorse è stato adottato il ciclo di Deming o ciclo PDCA. (\textbf{LA SPIEGAZIONE DEL PDCA DOVREBBE ANDARE SULLE NORME IN UN CAPITOLO A PARTE})\newline I processi qui descritti misurano la qualità del lavoro interno, ad esmepio se stimo lavorando secondo i tempi stabiliti

\subsubsection{Definizione e Pianificazione}
Poter controllare al meglio un processo si è scelto il modello incrementale/Agile, inoltre vengono descritte le attività e i compiti da svolgere, la pianificazione del lavoro e dei costi da sostenere . (\textbf{NON DESCRIVO I MODELLI /NON USO LE APPENDICI / IN QUESTO DOCUMENTO SI DEVE PARLARE DI QUANTITA' MISURABILI e NON DI COME CI SI ARRIVA}) Il gruppo inoltre si prefigge di rispettare i seguenti obiettivi:
\begin{itemize}
		\item{\textbf{Calendario:} assicurarsi di organizzare gli obiettivi assicurandosi del loro peso per poter rispettare le scadenze}
		\item{\textbf{Budget:} tramite le metriche descritte si cerca di allineare il budget il più possibile con gli obiettivi prefissati;}
		\item{\textbf{Standard:} definire uno standard per ogni processo al fine di facilitare il lavoro di gruppo e l'incremento continuo di ogni parte.}
\end{itemize} 
\subparagraph{Metriche utlizzate}
\textbf{(RICHIAMARE LE METRICHE CHE SONO STATE DESCRITTE NELLE NORME DI PROGETTO.}
\href{https://it.wikipedia.org/wiki/Metriche_di_progetto}{\textbf{PRESE DA QUI}}
\begin{itemize}
	\item{\textbf{BAC(Budget at Completition)};}
	\item{\textbf{BV(Budget Variance);}}
	\item{\textbf{Function Point;} non sono sicuro vada qui...}
	\item{\textbf{AC(Actual Cost)}.}
\end{itemize}
\begin{table}[!htpb]
	\resizebox{\textwidth}{!}{%
	\begin{tabular}{lll}
		\hline
		\rowcolor[HTML]{34CDF9} 
		{\color[HTML]{333333} \textbf{Nome}}         & {\color[HTML]{333333} \textbf{Accettazione}} & {\color[HTML]{333333} \textbf{Ottimalità}} \\ \hline
		\multicolumn{1}{|l|}{Budget at Completition} & \multicolumn{1}{l|}{}                        & \multicolumn{1}{l|}{}                      \\ \hline
		\multicolumn{1}{|l|}{Budget Variance}        & \multicolumn{1}{l|}{}                        & \multicolumn{1}{l|}{}                      \\ \hline
		\multicolumn{1}{|l|}{Cost Variance}          & \multicolumn{1}{l|}{}                        & \multicolumn{1}{l|}{}                      \\ \hline
	\end{tabular}%
}
\end{table}

\subsubsection{Verifica}
Questo processo ha lo scopo di verificare che tutti gli elementi soddisfino i requisiti necessari. In questa parte ci si prefigge di rispettare i seguenti obiettivi:
\begin{itemize}
	\item{\textbf{Commit brevi ed incisivi:} per facilitare così un'analisi ed un miglior intervento di verifica all'aggiunta di un nuovo bug;}
	\item{\textbf{Commenti al codice:} ogni porzione di codice deve essere commentata così da poter essere verificata.}
\end{itemize}
Viene così utilizzata parte dei criteri fondamentali del \textit{test coverage}
\subparagraph{Metriche utlizzate}.
\begin{itemize}
	\item{\textbf{Function coverage}}
	\item{\textbf{Statement coverage}}
	\item{\textbf{Edge coverage}}
	\item{\textbf{Condition coverage}}
\end{itemize}
\begin{table}[!htpb]
	\resizebox{\textwidth}{!}{%
	\begin{tabular}{|l|l|l|}
		\hline
		\rowcolor[HTML]{34CDF9} 
		{\color[HTML]{333333} \textbf{Nome}} & {\color[HTML]{333333} \textbf{Accettazione}} & {\color[HTML]{333333} \textbf{Ottimalità}} \\ \hline
		Function coverage                &      5-10\%         &  	\textless5\%             \\ \hline
		Statement coverage              &                                              &                                            \\ \hline
		Edge coverage                        &                                              &                                            \\ \hline
		Condition coverage                   &                                              &                                            \\ \hline
	\end{tabular}%
}
\end{table}
\subsubsection{Analisi e gestione dei rischi}
Questo processo ha lo scopo di monitorare ed evitare l'insorgere di nuovi rischi durante tutto il processo di realizzazione. Si prefiggonon quindi i seguenti obiettivi:
\begin{itemize}
	\item{\textbf{Analisi:} ad ogni fase è necessario analizzare i possibili rischi;}
	\item{\textbf{Categorizzazione:}  definire il tipo di rischio ad esempio se è di tipo noto, prevedibile o imprevedibile per raffinare gli strumenti con cui agire;}
	\item{\textbf{Catalogo dei rischi:} al fine di individuare i rischi è utile stilare un catalogo dei rischi, suddiviso in rischi noti e prevedibili.}
\end{itemize}
Vengono così utilizzate le seguenti metriche: 
\begin{itemize}
	\item{\textbf{Indisponibilità servizi esterni}}
	\item{\textbf{AGGIUNGERNE ALTRI...}}
\end{itemize}
\begin{table}[!htpb]
	\resizebox{\textwidth}{!}{%
		\begin{tabular}{|l|l|l|}
			\hline
			\rowcolor[HTML]{34CDF9} 
			{\color[HTML]{333333} \textbf{Nome}} & {\color[HTML]{333333} \textbf{Accettazione}} & {\color[HTML]{333333} \textbf{Ottimalità}} \\ \hline
			Indisponibilità \\ servizi esterni                    &                                              
		\end{tabular}%
	}
\end{table}
\newpage
\subsubsection{Gestione Test}
\textbf{LO FACCIAMO ? 353 LO HA MESSO NELLA V2. BISOGNA PENSARCI}
Prendendo nota che il processo di sviluppo è ancora in fase embrionale il gruppo non è ancora in grado di fornire delle metriche per la gestione dei test.
\subsubsection{Versionamento}
\textbf{LO FACCIAMO ? 353 LO HA MESSO NELLA V2. BISOGNA PENSARCI}
Come per la gestione dei test vale anche per il processo di versionamento