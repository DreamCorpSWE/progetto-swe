\section{Qualità di processo}
\subsection{Scopo}
\textit{"Da tubi sporchi non esce acqua pulita"}.
\newline
Con questa frase questo documento si prefigge  lo scopo di adottare la qualità di processo come esigenza fondamentale per perseguire la qualità di prodotto. Proprio per questo si è deciso di adottare il PDCA e lo standard ISO/IEC 15504 denominato SPICE. Inoltre si vuole far presente come l'insieme di questi contenuti non sia definitivo ma anzi viene incrementato durante il percorso. \textbf{Questo documento deve rispondere al cosa e non al come.}
%\subsection{Procedure di controllo}
%La qualità di processo viene suddivisa in:
%\begin{itemize}
%	\item{\textbf{Definizione:} per controllarlo e raccontarlo meglio};
%	\item{\textbf{Controllo:} perchè sia conforme alle attese e costi meno};
%	\item{\textbf{Validazione:}  per validare tramite PDCA.}
%\end{itemize}
\subsection{Processi}
Con l'obiettivo di ottenere un miglioramento continuo della qualità in un'ottica a lungo raggio e all'utilizzo ottimale delle risorse è stato adottato il ciclo di Deming o ciclo PDCA. (\textbf{LA SPIEGAZIONE DEL PDCA DOVREBBE ANDARE SULLE NORME IN UN CAPITOLO A PARTE})\newline I processi qui descritti misurano la qualità del lavoro interno, ad esmepio se stimo lavorando secondo i tempi stabiliti

\subsubsection{Definizione e Pianificazione (prc1)}
Poter controllare al meglio un processo si è scelto il modello incrementale/Agile, inoltre vengono descritte le attività e i compiti da svolgere, la pianificazione del lavoro e dei costi da sostenere . (\textbf{NON DESCRIVO I MODELLI /NON USO LE APPENDICI / IN QUESTO DOCUMENTO SI DEVE PARLARE DI QUANTITA' MISURABILI e NON DI COME CI SI ARRIVA}) Il gruppo inoltre si prefigge di rispettare i seguenti obiettivi:
\begin{itemize}
		\item{\textbf{Calendario:} assicurarsi di organizzare gli obiettivi assicurandosi del loro peso per poter rispettare le scadenze}
		\item{\textbf{Budget:} tramite le metriche descritte si cerca di allineare il budget il più possibile con gli obiettivi prefissati;}
		\item{\textbf{Standard:} definire uno standard per ogni processo al fine di facilitare il lavoro di gruppo e l'incremento continuo di ogni parte.}
\end{itemize} 
\subparagraph{Metriche utlizzate}
\href{https://it.wikipedia.org/wiki/Metriche_di_progetto}{\textbf{PRESE DA QUI}}
\begin{itemize}
	\item{\textbf{SV(Schedule Variance);}}
	\item{\textbf{BV(Budget Variance);}}
	%\item{\textbf{AC(Actual Cost)}.} serve già a calcolare budget variance
	\item{\textbf{Function Points.} }
\end{itemize}
\begin{table}[!htpb]
	\centering
	\renewcommand{\arraystretch}{2} 
	\rowcolors{2}{gray!25}{white}
	\resizebox{\textwidth}{!}{%
	\begin{tabular}{|l|l|l|}
		\hline
		\rowcolor{orange!50} 
		\textbf{Nome} & \textbf{Accettazione} & \textbf{Ottimalità} \\
		\hline
		Schedule Variance & \textgreater = -3 giorni &0 giorni \\
		\hline
		Budget Variance & \textgreater = -15\% & \textgreater = 0 \\ 
		\hline
		Function Points & - & - \\
		\hline
	\end{tabular}
	}
	\caption{TBD}
\end{table}

\subsubsection{Verifica (prc2)}
Questo processo ha lo scopo di verificare che tutti gli elementi soddisfino i requisiti necessari. In questa parte ci si prefigge di rispettare i seguenti obiettivi:
\begin{itemize}
	\item{\textbf{Commit brevi ed incisivi:} per facilitare così un'analisi ed un miglior intervento di verifica all'aggiunta di un nuovo bug;}
	\item{\textbf{Commenti al codice:} ogni porzione di codice deve essere commentata così da poter essere verificata.}
	\item{\textbf{Parlare id integrazione continua...forse}}
\end{itemize}
Viene così utilizzata parte dei criteri fondamentali del \textit{code coverage}
\subparagraph{Metriche utilizzate}.
\begin{itemize}
	\item{\textbf{Line coverage}(primitiva rispetto alle successive, fornisce un'idea generale)}
	\item{\textbf{Functional coverage}}
	\item{\textbf{Path coverage}}
	\item{\textbf{Condition coverage}}
	\item{\textbf{Branch coverage}}
\end{itemize}
\begin{table}[!htpb]
	%\resizebox{\textwidth}{!}{%
	\centering
	\renewcommand{\arraystretch}{2} 
	\rowcolors{2}{gray!25}{white}
	\resizebox{\textwidth}{!}{%
	\begin{tabular}{|l|l|l|}
		\rowcolor{orange!50}
		\hline
		\textbf{Nome} & \textbf{Accettazione} & \textbf{Ottimalità} \\
		\hline
		Line coverage & 90\% & 100\% \\
		\hline
		Functional coverage & 93\%(non più alto per evitare ridondanza) & 100\% \\
		\hline
		Path coverage & 96\% & 100\% \\
		\hline
		Condition coverage & 98\% & 100\% \\
		\hline
		Branch coverage & 95\% & 100\% \\
		\hline
	\end{tabular}
	}
	\caption{TBD}
%}
\end{table}
\subsubsection{Analisi e gestione dei rischi (prc3)}
Questo processo ha lo scopo di monitorare ed evitare l'insorgere di nuovi rischi durante tutto il processo di realizzazione. Si prefiggonon quindi i seguenti obiettivi:
\begin{itemize}
	\item{\textbf{Analisi:} ad ogni fase è necessario analizzare i possibili rischi;}
	\item{\textbf{Categorizzazione:}  definire il tipo di rischio ad esempio se è di tipo noto, prevedibile o imprevedibile per raffinare gli strumenti con cui agire;}
	\item{\textbf{Catalogo dei rischi:} al fine di individuare i rischi è utile stilare un catalogo dei rischi utilizzando la suddivisione del punto precedente.}
\end{itemize}
Vengono così utilizzate le seguenti metriche: 
\begin{itemize}
	\item{\textbf{Servizi esterni non raggiungibili}}
	\item{\textbf{Rischi non calcolati}}
	\subparagraph{nota:} per le metriche descritte in questa tabella viene utilizzato un indice numerico positivo che indica il numero di servizi esterni non raggiungibili o rischi non previsti che hanno minato temporaneamente il processo. (\textbf{non so se le note vadano qui o vadano spiegate in qualche altro documento})
\end{itemize}
\begin{table}[!htbp]
	\centering
	\renewcommand{\arraystretch}{2} 
	\rowcolors{2}{gray!25}{white}
	\resizebox{\textwidth}{!}{%
		\begin{tabular}{|l|l|l|}
			\rowcolor{orange!50}
			\hline
			Nome & Accettazione & Ottimalità \\
			\hline
			Servizi esterni non raggiungibili & 0 & 0 \\
			\hline
			Rischi non previsti & 0 & 0 \\
			\hline
		\end{tabular}
	}
	\caption{TBD}
\end{table}
\newpage
\subsubsection{Gestione Test (prc4)}
Prendendo nota che il processo di sviluppo è ancora in fase embrionale il gruppo non è ancora in grado di fornire delle metriche per la gestione dei test.
\subsubsection{Versionamento (prc5)}
Come per la gestione dei test vale anche per il processo di versionamento