\section{Qualità di processo}
\subsection{Scopo}
\textit{"Da tubi sporchi non esce acqua pulita"}.
\newline
Questa frase ha lo scopo di far vedere la qualità di processo come esigenza fondamentale per perseguire la qualità di prodotto. Proprio per questo si è deciso di adottare il PDCA e lo standard ISO/IEC 15504 denominato SPICE
\subsection{Procedure di controllo}
La qualità di processo viene suddivisa in:

\begin{itemize}
	\item{\textbf{Definizione:} per controllarlo e raccontarlo meglio};
	\item{\textbf{Controllo:} perchè sia conforme alle attese e costi meno};
	\item{\textbf{Validazione:}  per validare tramite PDCA.}
\end{itemize}
Con l'obiettivo di ottenere un miglioramento continuo della qualità in un'ottica a lungo raggio e all'utilizzo ottimale delle risorse è stato adottato il ciclo di Deming o ciclo PDCA.
\subsection{Processi}
\subsection{Definizione e Pianificazione}
Come precedentemente detto per poter controllare al meglio un processo si è scelto il modello incrementale/Agile (\textbf{descrivo i modelli ?}) con i seguienti obiettivi:
\begin{itemize}
		\item{\textbf{Task:}}
		\item{\textbf{Budget:}}
		\item{\textbf{Standard:}}
\end{itemize} 