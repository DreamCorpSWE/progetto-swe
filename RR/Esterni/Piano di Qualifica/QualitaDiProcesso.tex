\section{Qualità di processo}
\subsection{Scopo}
\textit{"Da tubi sporchi non esce acqua pulita"}.
\newline
Con questa frase questo documento si prefigge  lo scopo di adottare la qualità di processo come esigenza fondamentale per perseguire la qualità di prodotto. Proprio per questo si è deciso di adottare il PDCA e lo standard ISO/IEC 15504 denominato SPICE. Inoltre si vuole far presente come l'insieme di questi contenuti non sia definitivo ma anzi viene incrementato durante il percorso. \textbf{Questo documento deve rispondere al cosa e non al come.}
%\subsection{Procedure di controllo}
%La qualità di processo viene suddivisa in:
%\begin{itemize}
%	\item{\textbf{Definizione:} per controllarlo e raccontarlo meglio};
%	\item{\textbf{Controllo:} perchè sia conforme alle attese e costi meno};
%	\item{\textbf{Validazione:}  per validare tramite PDCA.}
%\end{itemize}
\subsection{Processi}
Con l'obiettivo di ottenere un miglioramento continuo della qualità in un'ottica a lungo raggio e all'utilizzo ottimale delle risorse è stato adottato il ciclo di Deming o ciclo PDCA. (\textbf{LA SPIEGAZIONE DEL PDCA DOVREBBE ANDARE SULLE NORME IN UN CAPITOLO A PARTE})
\subsubsection{Definizione e Pianificazione}
Poter controllare al meglio un processo si è scelto il modello incrementale/Agile, inoltre vengono descritte le attività e i compiti da svolgere, la pianificazione del lavoro e dei costi da sostenere . (\textbf{NON DESCRIVO I MODELLI /NON USO LE APPENDICI / IN QUESTO DOCUMENTO SI DEVE PARLARE DI QUANTITA' MISURABILI e NON DI COME CI SI ARRIVA}) Il gruppo inoltre si prefigge di rispettare i seguenti obiettivi:
\begin{itemize}
		\item{\textbf{Calendario:} assicurarsi di organizzare gli obiettivi assicurandosi del loro peso per poter rispettare le scadenze}
		\item{\textbf{Budget:} tramite le metriche descritte si cerca di allineare il budget il più possibile con gli obiettivi prefissati;}
		\item{\textbf{Standard:} definire uno standard per ogni processo al fine di facilitare il lavoro di gruppo e l'incremento continuo di ogni parte.}
\end{itemize} 
\subsubsection{Metriche utlizzate}
\textbf{RICHIAMARE LE METRICHE CHE SONO STATE DESCRITTE NELLE NORME DI PROGETTO}
\paragraph{tabella delle metriche utlizzate in questo passo}