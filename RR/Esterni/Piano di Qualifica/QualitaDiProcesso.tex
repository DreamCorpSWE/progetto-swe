\section{Qualità di processo}
\subsection{Scopo}
Il documento si prefigge di adottare la qualità di processo come esigenza fondamentale per perseguire la qualità di prodotto. Proprio per questo si è deciso di adottare il PDCA\pedice~e lo standard ISO/IEC 15504\pedice~denominato SPICE. Inoltre si vuole far presente come l'insieme di questi contenuti non sia definitivo ma anzi viene incrementato durante il percorso.
\subsection{Processi}
Con l'obiettivo di ottenere un miglioramento continuo della qualità in un'ottica a lungo termine e all'utilizzo ottimale delle risorse è stato adottato il ciclo di Deming o ciclo PDCA.\newline
I processi descritti di seguito misurano la qualità del lavoro interno (ad esempio se stiamo lavorando secondo i tempi prestabiliti).
Ogni processo verrà inizialmente descritto e il gruppo identificherà per esso degli obiettivi. Conseguentemente verranno elencate le metriche scelte per l'analisi dell'andamento di questi ultimi. \\
Le metriche utilizzate per i processi sono definite nelle \textit{Norme di progetto} alla §3.4:

\subsubsection{P1: Definizione e Pianificazione}
Poter controllare al meglio un processo si è scelto il modello incrementale, inoltre vengono descritte le attività e i compiti da svolgere, la pianificazione del lavoro e dei costi da sostenere. Il gruppo inoltre si prefigge di rispettare i seguenti obiettivi:
\begin{itemize}
		\item{\textbf{Scadenze:} assicurarsi di organizzare gli obiettivi assicurandosi del loro peso per poter rispettare le scadenze;}
		\item{\textbf{Budget:} tramite le metriche descritte si cerca di allineare il budget il più possibile agli obiettivi prefissati;}
		\item{\textbf{Standard:} definire uno standard per ogni processo al fine di facilitare il lavoro di gruppo e l'incremento continuo di ogni parte.}
\end{itemize} 
\subparagraph{Metriche utlizzate}
\begin{itemize}
	\item{\textbf{SV(Schedule Variance);}}
	\item{\textbf{BV(Budget Variance).}}
\end{itemize}
\begin{table}[!htpb]
	\centering
	\renewcommand{\arraystretch}{2} 
	\rowcolors{2}{gray!25}{white}
	\begin{tabular}{|l|l|l|}
		\hline
		\rowcolor{orange!50} 
		\textbf{Nome} & \textbf{Accettazione} & \textbf{Ottimalità} \\
		\hline
		Schedule Variance & \textgreater = -3 giorni &0 giorni \\
		\hline
		Budget Variance & -15 \% $<$ x $<$ +15 \% & -1\% $<$ x $<$ +1\% \\ 
		\hline
	\end{tabular}
	\caption{Metriche utilizzate per la Definizione e Pianificazione}
\end{table}

\subsubsection{P2: Verifica}
Questo processo ha lo scopo di verificare che tutti gli elementi soddisfino i requisiti necessari. In questa parte ci si prefigge di rispettare i seguenti obiettivi:
\begin{itemize}
	\item{\textbf{Commit brevi ed incisivi:} per facilitare così un'analisi ed un miglior intervento di verifica alla comparsa di un nuovo bug;}
	\item{\textbf{Commenti al codice:} ogni porzione di codice deve essere commentata così da poter essere compresa e condivisa da collaboratori diversi dall'autore.}
\end{itemize}
Viene così utilizzata parte dei criteri fondamentali del code coverage\pedice.
\subparagraph{Metriche utilizzate}
\begin{itemize}
	\item{\textbf{Line coverage};}
	\item{\textbf{Functional coverage};}
	\item{\textbf{Path coverage};}
	\item{\textbf{Condition coverage};}
	\item{\textbf{Branch coverage}.}
\end{itemize}
\begin{table}[H]
	\centering
	\renewcommand{\arraystretch}{2} 
	\rowcolors{2}{gray!25}{white}
	\begin{tabular}{|l|l|l|}
		\rowcolor{orange!50}
		\hline
		\textbf{Nome} & \textbf{Accettazione} & \textbf{Ottimalità} \\
		\hline
		Line coverage & 90\% & 100\% \\
		\hline
		Functional coverage & 93\%(non più alto per evitare ridondanza) & 100\% \\
		\hline
		Path coverage & 96\% & 100\% \\
		\hline
		Condition coverage & 98\% & 100\% \\
		\hline
		Branch coverage & 95\% & 100\% \\
		\hline
	\end{tabular}
	\caption{Metriche utilizzate per la Verifica}
%}
\end{table}
\subsubsection{P3: Analisi e gestione dei rischi}
Questo processo ha lo scopo di monitorare ed evitare l'insorgere di nuovi rischi durante tutto il processo di realizzazione. Ci prefiggiamo quindi i seguenti obiettivi:
\begin{itemize}
	\item{\textbf{Analisi:} analizzare i possibili rischi ad ogni fase;}
	\item{\textbf{Categorizzazione:}  definire il tipo di rischio per raffinare gli strumenti con cui agire (se è di tipo noto, prevedibile o imprevedibile);}
	\item{\textbf{Catalogo dei rischi:} al fine di individuare i rischi è utile stilarne un catalogo utilizzando la suddivisione del punto precedente.}
\end{itemize}
\subparagraph{Metriche utlizzate}
\begin{itemize}
	\item{\textbf{Servizi esterni non raggiungibili};}
	\item{\textbf{Rischi non calcolati}.}
\end{itemize}
\begin{table}[!htbp]
	\centering
	\renewcommand{\arraystretch}{2} 
	\rowcolors{2}{gray!25}{white}
		\begin{tabular}{|l|l|l|}
			\rowcolor{orange!50}
			\hline
			Nome & Accettazione & Ottimalità \\
			\hline
			Servizi esterni non raggiungibili & 2 & 0 \\
			\hline
			Rischi non calcolati & 2 & 0 \\
			\hline
		\end{tabular}
	\caption{Metriche Analisi e Gestione dei rischi}
\end{table}
\subparagraph{\textbf{NB}:} per le metriche descritte in questa tabella viene utilizzato un indice intero non negativo che indica il numero di servizi esterni non raggiungibili o rischi non previsti che hanno minato temporaneamente il processo.
\newpage
\subsubsection{P4: Gestione Test}
Prendendo nota che il processo di sviluppo è ancora in fase embrionale il gruppo non è ancora in grado di fornire delle metriche per la gestione dei test.
\subsubsection{P5: Versionamento}
Come per la gestione dei test vale anche per il processo di versionamento.
