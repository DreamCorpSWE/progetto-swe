\medskip
~\newline
\begin{table}[!h] % h! serve per posizionarla relativamente
            \centering
            \renewcommand{\arraystretch}{2}
            \rowcolors{2}{gray!25}{white} %colori alternati, grigio 25% e bianco 100%
            \begin{tabular}{|c|c|p{6cm}|l|l|} % p{dimensione desiderata}
                \rowcolor{orange!50} %colore intestazione
        		\hline
        		\textbf{Versione} & \textbf{Data} & \textbf{Descrizione} & \textbf{Autore} & \textbf{Ruolo} \\
                \hline
                1.0.0 & 09/01/2019 & Approvazione documento per rilascio RR & \pie & Responsabile \\
                \hline
                0.5.2 & 08/01/2019 & Superamento verifica & \daL & Verificatore \\
                \hline
                0.5.1 & 08/01/2019 & Aggiunta di nuovi termini per il documento \textit{Piano di Qualifica} & \gia & Verificatore \\ 
                \hline
                0.5.0 & 06/01/2019 & Aggiunta di nuovi termini per il documento \textit{Piano di Qualifica} & \gia & Verificatore \\
                \hline
                0.4.0 & 26/12/2018 & Aggiunta di nuovi termini per il documento \textit{Norme di Progetto} & \daG & Analista \\
                \hline
                0.3.0 & 25/12/2018 & Aggiunta di nuovi termini per il documento \textit{Piano di Progetto} & \daL & Analista \\
                \hline
                0.2.1 & 23/12/2018 & Aggiunta di nuovi termini per il documento \textit{Analisi dei requisiti} & \daL & Analista \\
                \hline
                0.2.0 & 22/12/2018 & Aggiunta di nuovi termini per il documento \textit{Analisi dei requisiti} & \daL & Analista \\
                \hline
                0.1.0 & 21/12/2018 & Stesura iniziale del documento & \daG & Responsabile \\
                \hline
                
        \end{tabular}
        \caption{Log delle Modifiche} %descrizone a fine tabella
        \label{tab:Log delle modifiche}
\end{table}
