\newglossaryentry{Capitolato}
{
    name=Capitolato,
    description={Il capitolato è un documento tecnico in linguaggio naturale al quale si fa riferimento per definire le specifiche tecniche di un problema che si vuole risolvere e le caratteristiche richieste dal prodotto}
}
\newglossaryentry{DevOps}
{
    name=DevOps,
    description={(Development and Operations) è un insieme di pratiche finalizzate ad aumentare l'automatismo tra processi per lo sviluppo di software in modo da poter effettuare le build, implementare i test e rilasciare software in modo più veloce e sicuro}
}
\newglossaryentry{Zucchetti}
{
    name=Zucchetti,
    description={Zucchetti è un'azienda italiana che produce soluzioni software, hardware e servizi per aziende, banche, assicurazioni, professionisti e associazioni di categoria}
}
\newglossaryentry{Plug-in}
{
    name=Plug-in,
    description={Il plugin in campo informatico è un programma non autonomo che interagisce con un altro programma per ampliarne o estenderne le funzionalità originarie}
}
\newglossaryentry{Grafana}
{
    name=Grafana,
    description={Grafana è un' applicazione web open-source che permette il monitoraggio di flussi di dati provenienti da diversi tipi di database tramite dashboard e grafici}
}

\newglossaryentry{Javascript}
{
    name=Javascript,
    description={E' un linguaggio dinamico di alto livello, lato-client, interpretato dal browser e spesso utilizzato nel web per descrivere il comportamento delle pagine}
}
\newglossaryentry{Json}
{
    name=Json,
    description={(JavaScript Object Notation) Viene usato per lo scambio di dati tra il browser e il server, è basato sul modello della programmazione ad oggetti di Javascript}
}
\newglossaryentry{Rete Bayesiana}
{
    name=Rete Bayesiana,
    description={E' un grafo aciclico orientato in cui i nodi rappresentano le variabili e gli archi rappresentano le relazioni di dipendenza statistica tra le variabili e le distribuzioni locali di probabilità dei nodi figlio rispetto ai valori dei nodi padre. Una rete bayesiana rappresenta la distribuzione della probabilità congiunta di un insieme di variabili
}
}
\newglossaryentry{Flusso di Monitoraggio}
{
    name=Flusso di monitoraggio,
    description={Flusso di dati provenienti da un database interrogato da una query effettuata ad intervalli temporali prestabiliti}
}
\newglossaryentry{Intelligenza artificiale}
{
    name=Intelligenza artificiale,
    description={Abilità di un sistema tecnologico di risolvere problemi o svolgere compiti e attività tipici della mente e dell’abilità umana}
}
\newglossaryentry{Dashboard}
{
    name=Dashboard,
    description={Insieme di Panels organizzati e disposti in una o più righe}
}
\newglossaryentry{Alert}
{
    name=Alert,
    description={Segnale che avvisa, tramite una notifica, una possibile situazione critica da tenere sottocontrollo}
}
\newglossaryentry{Casi d'uso}
{
    name=Casi d'uso,
    description={Descrive una sequenza di interazioni tra il sistema e un utente che può essere un umano o un altro software, chiamato attore, che devono essere svolte per ottenere un risultato. Rappresentano quindi i vari scenari che si possono incontrare nell’utilizzo di un prodotto software}
}
\newglossaryentry{Attori}
{
    name=Attori,
    description={Chi o cosa interagisce con il sistema, possono essere sia umani che altri software}
}
\newglossaryentry{Panel}
{
    name=Panel,
    description={Pannello che mostra il risultato di una query sottoforma di grafico, tabella o testo}
}
\newglossaryentry{Strumento di monitoraggio}
{
    name=Strumento di monitoraggio,
    description={Strumento che monitora costantemente lo stato di un sistema notificando l'amministratore se uno o più valori superano una fissata soglia critica}
}
\newglossaryentry{Query}
{
    name=Query,
    description={Interrogazione da parte di un utente di un database per compiere determinate operazioni sui dati come selezione, inserimento, cancellazione e aggiornamento}
}
\newglossaryentry{Machine Learning}
{
    name=Machine Learning,
    description={Rappresenta un insieme di metodi statistici per migliorare progressivamente la performance di un algoritmo nell'identificare pattern nei dati}
}
\newglossaryentry{Requisiti}
{
    name=Requisiti,
    description={Requisiti che un sistema deve soddisfare, possono essere obbligatori se stabiliti dal cliente oppure opzionali se implementati a discrezione dello sviluppatore (da rivedere)}
}
\newglossaryentry{GitHub}
{
    name=GitHub,
    description={Software di controllo di versione, permette di aggiornare un file senza dover sovrascrivere le versioni precedenti.}
}
\newglossaryentry{AirBnb}
{
    name=AirBnb,
    description={Insieme di regole che indicano come un codice Javascript dovrebbe essere organizzato e scritto in modo da avere un codice più leggibile e corretto}
}
\newglossaryentry{Open-Source}
{
    name=Open-Source,
    description={Software di cui gli autori rendono pubblico il codice sorgente, favorendone il libero studio e permettendo a programmatori indipendenti di apportarvi modifiche ed estensioni.}
}


