\newglossaryentry{Capitolato}
{
    name=Capitolato,
    description={Il capitolato è un documento tecnico in linguaggio naturale al quale si fa riferimento per definire le specifiche tecniche di un problema che si vuole risolvere e le caratteristiche richieste dal prodotto}
}
\newglossaryentry{DevOps}
{
    name=DevOps,
    description={(Development and Operations) è un insieme di pratiche finalizzate ad aumentare l'automatismo tra processi per lo sviluppo di software in modo da poter effettuare le build, implementare i test e rilasciare software in modo più veloce e sicuro}
}
\newglossaryentry{Zucchetti}
{
    name=Zucchetti,
    description={Zucchetti è un'azienda italiana che produce soluzioni software, hardware e servizi per aziende, banche, assicurazioni, professionisti e associazioni di categoria}
}
\newglossaryentry{Plugin}
{
    name=Plug-in,
    description={Il plugin in campo informatico è un programma non autonomo che interagisce con un altro programma per ampliarne o estenderne le funzionalità originarie}
}
\newglossaryentry{Grafana}
{
    name=Grafana,
    description={Grafana è un' applicazione web open-source che permette il monitoraggio di flussi di dati provenienti da diversi tipi di database tramite dashboard e grafici}
}

\newglossaryentry{Javascript}
{
    name=Javascript,
    description={E' un linguaggio dinamico di alto livello, lato-client, interpretato dal browser e spesso utilizzato nel web per descrivere il comportamento delle pagine}
}
\newglossaryentry{Json}
{
    name=Json,
    description={(JavaScript Object Notation) Viene usato per lo scambio di dati tra il browser e il server, è basato sul modello della programmazione ad oggetti di Javascript}
}
\newglossaryentry{Rete Bayesiana}
{
    name=Rete Bayesiana,
    description={E' un grafo aciclico orientato in cui i nodi rappresentano le variabili e gli archi rappresentano le relazioni di dipendenza statistica tra le variabili e le distribuzioni locali di probabilità dei nodi figlio rispetto ai valori dei nodi padre. Una rete bayesiana rappresenta la distribuzione della probabilità congiunta di un insieme di variabili
}
}
\newglossaryentry{Flusso di Monitoraggio}
{
    name=Flusso di monitoraggio,
    description={Flusso di dati provenienti da un database interrogato da una query effettuata ad intervalli temporali prestabiliti}
}
\newglossaryentry{Intelligenza artificiale}
{
    name=Intelligenza artificiale,
    description={Abilità di un sistema tecnologico di risolvere problemi o svolgere compiti e attività tipici della mente e dell’abilità umana}
}
\newglossaryentry{Dashboard}
{
    name=Dashboard,
    description={Insieme di Panels organizzati e disposti in una o più righe}
}
\newglossaryentry{Alert}
{
    name=Alert,
    description={Segnale che avvisa, tramite una notifica, una possibile situazione critica da tenere sottocontrollo}
}
\newglossaryentry{Casi d'uso}
{
    name=Casi d'uso,
    description={Descrive una sequenza di interazioni tra il sistema e un utente che può essere un umano o un altro software, chiamato attore, che devono essere svolte per ottenere un risultato. Rappresentano quindi i vari scenari che si possono incontrare nell’utilizzo di un prodotto software}
}
\newglossaryentry{Attori}
{
    name=Attori,
    description={Chi o cosa interagisce con il sistema, possono essere sia umani che altri software}
}
\newglossaryentry{Panel}
{
    name=Panel,
    description={Pannello che mostra il risultato di una query sottoforma di grafico, tabella o testo}
}
\newglossaryentry{Strumento di monitoraggio}
{
    name=Strumento di monitoraggio,
    description={Strumento che monitora costantemente lo stato di un sistema notificando l'amministratore se uno o più valori superano una fissata soglia critica}
}
\newglossaryentry{Query}
{
    name=Query,
    description={Interrogazione da parte di un utente di un database per compiere determinate operazioni sui dati come selezione, inserimento, cancellazione e aggiornamento}
}
\newglossaryentry{Machine Learning}
{
    name=Machine Learning,
    description={Rappresenta un insieme di metodi statistici per migliorare progressivamente la performance di un algoritmo nell'identificare pattern nei dati}
}
\newglossaryentry{Requisiti}
{
    name=Requisiti,
    description={Requisiti che un sistema deve soddisfare, possono essere obbligatori se stabiliti dal cliente oppure opzionali se implementati a discrezione dello sviluppatore (da rivedere)}
}
\newglossaryentry{GitHub}
{
    name=GitHub,
    description={Software di controllo di versione, permette di aggiornare un file senza dover sovrascrivere le versioni precedenti}
}
\newglossaryentry{AirBnb}
{
    name=AirBnb,
    description={Insieme di regole che indicano come un codice Javascript dovrebbe essere organizzato e scritto in modo da avere un codice più leggibile e corretto}
}
\newglossaryentry{Open-Source}
{
    name=Open-Source,
    description={Software di cui gli autori rendono pubblico il codice sorgente, favorendone il libero studio e permettendo a programmatori indipendenti di apportarvi modifiche ed estensioni}
}
\newglossaryentry{Efficienza}
{
    name=Efficienza,
    description={Indica la capacità di raggiungere l'obiettivo prefissato impiegando le risorse minime indispensabili}
}
\newglossaryentry{Stakeholders}
{
    name=Stakeholders,
    description={Tutti i soggetti, individui od organizzazioni, attivamente coinvolti in un progetto il cui interesse è influenzato dal risultato e la cui azione o reazione a sua volta influenza le fasi o il completamento del progetto. Ad esempio lo possono essere il cliente, il fornitore, i membri del team di progetto, i fruitori dei risultati in uscita dal progetto e i finanziatori}
}
\newglossaryentry{UML}
{
    name=UML,
    description={UML (Unified Modeling Language) è un linguaggio di modellazione in ambito di progettazione di software object oriented. Permette di rappresentare i diagrammi delle classi, dei casi d'uso, diagrammi di stato e di sequenza}
}
\newglossaryentry{Slack}
{
    name=Slack,
    description={Applicazione di messaggistica istantanea pensata per la collaborazione tra i membri di uno o piu' gruppi di lavoro. Offre la possibilita' di creare canali privati, gruppi privati e chat dirette. Tutto cio' che e' presente in Slack puo' essere cercato, che sia un file, una conversazione o le persone stesse. Inoltre puo' essere esteso mediante l' uso di applicazioni di terze parti}
}
\newglossaryentry{Gmail}
{
    name=Gmail,
    description={Gmail è un servizio di email gratuito sviluppato da Google nel 2004, è possibile accedervi sia tramite web che tramite un' apposita applicazione}
}
\newglossaryentry{IssueTrackingSystem}
{
    name=IssueTrackingSystem,
    description={E' un software che registra e mantiene una lista delle issues, ovvero un' insieme di passi da compiere per migliorare un sistema}
}
\newglossaryentry{TexStudio}
{
    name=TexStudio,
    description={E' un editor open-source per fogli di tipo Latex, tra le sue caratteristiche include un correttore dello spelling, un sistema autocompletamento, ed un visualizzatore di PDF integrato}
}
\newglossaryentry{Google Drive}
{
    name=Google Drive,
    description={Google Drive è un servizio web, in ambiente cloud computing, di memorizzazione e sincronizzazione online introdotto da Google nel 2012. Comprende il file hosting, il file sharing e la modifica collaborativa di documenti, da a disposizione fino a 15 GB di spazio gratutito estendibili fino a 30 TB in totale}
}
\newglossaryentry{Repository}
{
    name=Repository,
    description={Ambiente di un sistema informativo in cui vengono  gestiti i metadati attraverso tabelle relazionali. L’insieme di tabelle, regole e motori di calcolo tramite cui si gestiscono i metadati prende il nome di metabase}
}
\newglossaryentry{Branch}
{
    name=Branch,
    description={Ramo di lavoro che favorisce uno sviluppo non lineare e incrementale, offrendo la possibilità di sviluppare degli incrementi sui branch che verranno poi uniti al branch principale chiamato master}
}
\newglossaryentry{Commit}
{
    name=Commit,
    description={Un insieme di modifiche o incrementi effettuati ai file contenuti in un repository, richiede una breve descrizione su quanto aggiunto o modificato}
}
\newglossaryentry{Framework}
{
    name=Framework,
    description={E' un' architettura logica di supporto (spesso un' implementazione logica di un particolare design pattern) su cui un software può essere progettato e realizzato, spesso facilitandone lo sviluppo da parte del programmatore}
}
\newglossaryentry{Tecnology Baseline}
{
    name=Tecnology Baseline,
    description={Configurazione di software, hardware e processi stabiliti e documentati da utilizzare come riferimento. E' una base di appoggio alla quale non si può retrocedere}
}
\newglossaryentry{Diagramma di Gantt}
{
    name=Diagramma di Gantt,
    description={Permette la rappresentazione grafica di un calendario di attività, utile al fine di pianificare, coordinare e tracciare specifiche attività in un progetto dando una chiara illustrazione dello stato d'avanzamento del progetto rappresentato}
}
\newglossaryentry{Milestone}
{
    name=Milestone,
    description={Il termine milestone viene tipicamente utilizzato nella pianificazione e gestione di progetti complessi per indicare il raggiungimento di obiettivi stabiliti in fase di definizione del progetto stesso}
}
\newglossaryentry{Magnitudo}
{
    name=Magnitudo,
    description={Nell'analisi dei rischi è calcolata come il prodotto dell'impatto di un rischio e della probabilità che esso accada. Descrive quindi il rischio reale}
}
\newglossaryentry{Latex}
{
    name=Latex,
    description={Latex è un linguaggio di markup per la preparazione di testi, basato sul programma di composizione tipografica TEX}
}
\newglossaryentry{TexStudio}
{
    name=TexStudio,
    description={TeXstudio è un ambiente di scrittura integrata per scrivere documenti Latex}
}
\newglossaryentry{Overleaf}
{
    name=Overleaf,
    description={Overleaf è un editor online Latex che permette la collaborazione real-time e la compilazione online}
}



